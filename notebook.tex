
% Default to the notebook output style

    


% Inherit from the specified cell style.




    
\documentclass[11pt]{article}

    
    
    \usepackage[T1]{fontenc}
    % Nicer default font (+ math font) than Computer Modern for most use cases
    \usepackage{mathpazo}

    % Basic figure setup, for now with no caption control since it's done
    % automatically by Pandoc (which extracts ![](path) syntax from Markdown).
    \usepackage{graphicx}
    % We will generate all images so they have a width \maxwidth. This means
    % that they will get their normal width if they fit onto the page, but
    % are scaled down if they would overflow the margins.
    \makeatletter
    \def\maxwidth{\ifdim\Gin@nat@width>\linewidth\linewidth
    \else\Gin@nat@width\fi}
    \makeatother
    \let\Oldincludegraphics\includegraphics
    % Set max figure width to be 80% of text width, for now hardcoded.
    \renewcommand{\includegraphics}[1]{\Oldincludegraphics[width=.8\maxwidth]{#1}}
    % Ensure that by default, figures have no caption (until we provide a
    % proper Figure object with a Caption API and a way to capture that
    % in the conversion process - todo).
    \usepackage{caption}
    \DeclareCaptionLabelFormat{nolabel}{}
    \captionsetup{labelformat=nolabel}

    \usepackage{adjustbox} % Used to constrain images to a maximum size 
    \usepackage{xcolor} % Allow colors to be defined
    \usepackage{enumerate} % Needed for markdown enumerations to work
    \usepackage{geometry} % Used to adjust the document margins
    \usepackage{amsmath} % Equations
    \usepackage{amssymb} % Equations
    \usepackage{textcomp} % defines textquotesingle
    % Hack from http://tex.stackexchange.com/a/47451/13684:
    \AtBeginDocument{%
        \def\PYZsq{\textquotesingle}% Upright quotes in Pygmentized code
    }
    \usepackage{upquote} % Upright quotes for verbatim code
    \usepackage{eurosym} % defines \euro
    \usepackage[mathletters]{ucs} % Extended unicode (utf-8) support
    \usepackage[utf8x]{inputenc} % Allow utf-8 characters in the tex document
    \usepackage{fancyvrb} % verbatim replacement that allows latex
    \usepackage{grffile} % extends the file name processing of package graphics 
                         % to support a larger range 
    % The hyperref package gives us a pdf with properly built
    % internal navigation ('pdf bookmarks' for the table of contents,
    % internal cross-reference links, web links for URLs, etc.)
    \usepackage{hyperref}
    \usepackage{longtable} % longtable support required by pandoc >1.10
    \usepackage{booktabs}  % table support for pandoc > 1.12.2
    \usepackage[inline]{enumitem} % IRkernel/repr support (it uses the enumerate* environment)
    \usepackage[normalem]{ulem} % ulem is needed to support strikethroughs (\sout)
                                % normalem makes italics be italics, not underlines
    

    
    
    % Colors for the hyperref package
    \definecolor{urlcolor}{rgb}{0,.145,.698}
    \definecolor{linkcolor}{rgb}{.71,0.21,0.01}
    \definecolor{citecolor}{rgb}{.12,.54,.11}

    % ANSI colors
    \definecolor{ansi-black}{HTML}{3E424D}
    \definecolor{ansi-black-intense}{HTML}{282C36}
    \definecolor{ansi-red}{HTML}{E75C58}
    \definecolor{ansi-red-intense}{HTML}{B22B31}
    \definecolor{ansi-green}{HTML}{00A250}
    \definecolor{ansi-green-intense}{HTML}{007427}
    \definecolor{ansi-yellow}{HTML}{DDB62B}
    \definecolor{ansi-yellow-intense}{HTML}{B27D12}
    \definecolor{ansi-blue}{HTML}{208FFB}
    \definecolor{ansi-blue-intense}{HTML}{0065CA}
    \definecolor{ansi-magenta}{HTML}{D160C4}
    \definecolor{ansi-magenta-intense}{HTML}{A03196}
    \definecolor{ansi-cyan}{HTML}{60C6C8}
    \definecolor{ansi-cyan-intense}{HTML}{258F8F}
    \definecolor{ansi-white}{HTML}{C5C1B4}
    \definecolor{ansi-white-intense}{HTML}{A1A6B2}

    % commands and environments needed by pandoc snippets
    % extracted from the output of `pandoc -s`
    \providecommand{\tightlist}{%
      \setlength{\itemsep}{0pt}\setlength{\parskip}{0pt}}
    \DefineVerbatimEnvironment{Highlighting}{Verbatim}{commandchars=\\\{\}}
    % Add ',fontsize=\small' for more characters per line
    \newenvironment{Shaded}{}{}
    \newcommand{\KeywordTok}[1]{\textcolor[rgb]{0.00,0.44,0.13}{\textbf{{#1}}}}
    \newcommand{\DataTypeTok}[1]{\textcolor[rgb]{0.56,0.13,0.00}{{#1}}}
    \newcommand{\DecValTok}[1]{\textcolor[rgb]{0.25,0.63,0.44}{{#1}}}
    \newcommand{\BaseNTok}[1]{\textcolor[rgb]{0.25,0.63,0.44}{{#1}}}
    \newcommand{\FloatTok}[1]{\textcolor[rgb]{0.25,0.63,0.44}{{#1}}}
    \newcommand{\CharTok}[1]{\textcolor[rgb]{0.25,0.44,0.63}{{#1}}}
    \newcommand{\StringTok}[1]{\textcolor[rgb]{0.25,0.44,0.63}{{#1}}}
    \newcommand{\CommentTok}[1]{\textcolor[rgb]{0.38,0.63,0.69}{\textit{{#1}}}}
    \newcommand{\OtherTok}[1]{\textcolor[rgb]{0.00,0.44,0.13}{{#1}}}
    \newcommand{\AlertTok}[1]{\textcolor[rgb]{1.00,0.00,0.00}{\textbf{{#1}}}}
    \newcommand{\FunctionTok}[1]{\textcolor[rgb]{0.02,0.16,0.49}{{#1}}}
    \newcommand{\RegionMarkerTok}[1]{{#1}}
    \newcommand{\ErrorTok}[1]{\textcolor[rgb]{1.00,0.00,0.00}{\textbf{{#1}}}}
    \newcommand{\NormalTok}[1]{{#1}}
    
    % Additional commands for more recent versions of Pandoc
    \newcommand{\ConstantTok}[1]{\textcolor[rgb]{0.53,0.00,0.00}{{#1}}}
    \newcommand{\SpecialCharTok}[1]{\textcolor[rgb]{0.25,0.44,0.63}{{#1}}}
    \newcommand{\VerbatimStringTok}[1]{\textcolor[rgb]{0.25,0.44,0.63}{{#1}}}
    \newcommand{\SpecialStringTok}[1]{\textcolor[rgb]{0.73,0.40,0.53}{{#1}}}
    \newcommand{\ImportTok}[1]{{#1}}
    \newcommand{\DocumentationTok}[1]{\textcolor[rgb]{0.73,0.13,0.13}{\textit{{#1}}}}
    \newcommand{\AnnotationTok}[1]{\textcolor[rgb]{0.38,0.63,0.69}{\textbf{\textit{{#1}}}}}
    \newcommand{\CommentVarTok}[1]{\textcolor[rgb]{0.38,0.63,0.69}{\textbf{\textit{{#1}}}}}
    \newcommand{\VariableTok}[1]{\textcolor[rgb]{0.10,0.09,0.49}{{#1}}}
    \newcommand{\ControlFlowTok}[1]{\textcolor[rgb]{0.00,0.44,0.13}{\textbf{{#1}}}}
    \newcommand{\OperatorTok}[1]{\textcolor[rgb]{0.40,0.40,0.40}{{#1}}}
    \newcommand{\BuiltInTok}[1]{{#1}}
    \newcommand{\ExtensionTok}[1]{{#1}}
    \newcommand{\PreprocessorTok}[1]{\textcolor[rgb]{0.74,0.48,0.00}{{#1}}}
    \newcommand{\AttributeTok}[1]{\textcolor[rgb]{0.49,0.56,0.16}{{#1}}}
    \newcommand{\InformationTok}[1]{\textcolor[rgb]{0.38,0.63,0.69}{\textbf{\textit{{#1}}}}}
    \newcommand{\WarningTok}[1]{\textcolor[rgb]{0.38,0.63,0.69}{\textbf{\textit{{#1}}}}}
    
    
    % Define a nice break command that doesn't care if a line doesn't already
    % exist.
    \def\br{\hspace*{\fill} \\* }
    % Math Jax compatability definitions
    \def\gt{>}
    \def\lt{<}
    % Document parameters
    \title{Chapter\_10\_Data\_Arregation\_and\_Grouping\_Operations}
    
    
    

    % Pygments definitions
    
\makeatletter
\def\PY@reset{\let\PY@it=\relax \let\PY@bf=\relax%
    \let\PY@ul=\relax \let\PY@tc=\relax%
    \let\PY@bc=\relax \let\PY@ff=\relax}
\def\PY@tok#1{\csname PY@tok@#1\endcsname}
\def\PY@toks#1+{\ifx\relax#1\empty\else%
    \PY@tok{#1}\expandafter\PY@toks\fi}
\def\PY@do#1{\PY@bc{\PY@tc{\PY@ul{%
    \PY@it{\PY@bf{\PY@ff{#1}}}}}}}
\def\PY#1#2{\PY@reset\PY@toks#1+\relax+\PY@do{#2}}

\expandafter\def\csname PY@tok@w\endcsname{\def\PY@tc##1{\textcolor[rgb]{0.73,0.73,0.73}{##1}}}
\expandafter\def\csname PY@tok@c\endcsname{\let\PY@it=\textit\def\PY@tc##1{\textcolor[rgb]{0.25,0.50,0.50}{##1}}}
\expandafter\def\csname PY@tok@cp\endcsname{\def\PY@tc##1{\textcolor[rgb]{0.74,0.48,0.00}{##1}}}
\expandafter\def\csname PY@tok@k\endcsname{\let\PY@bf=\textbf\def\PY@tc##1{\textcolor[rgb]{0.00,0.50,0.00}{##1}}}
\expandafter\def\csname PY@tok@kp\endcsname{\def\PY@tc##1{\textcolor[rgb]{0.00,0.50,0.00}{##1}}}
\expandafter\def\csname PY@tok@kt\endcsname{\def\PY@tc##1{\textcolor[rgb]{0.69,0.00,0.25}{##1}}}
\expandafter\def\csname PY@tok@o\endcsname{\def\PY@tc##1{\textcolor[rgb]{0.40,0.40,0.40}{##1}}}
\expandafter\def\csname PY@tok@ow\endcsname{\let\PY@bf=\textbf\def\PY@tc##1{\textcolor[rgb]{0.67,0.13,1.00}{##1}}}
\expandafter\def\csname PY@tok@nb\endcsname{\def\PY@tc##1{\textcolor[rgb]{0.00,0.50,0.00}{##1}}}
\expandafter\def\csname PY@tok@nf\endcsname{\def\PY@tc##1{\textcolor[rgb]{0.00,0.00,1.00}{##1}}}
\expandafter\def\csname PY@tok@nc\endcsname{\let\PY@bf=\textbf\def\PY@tc##1{\textcolor[rgb]{0.00,0.00,1.00}{##1}}}
\expandafter\def\csname PY@tok@nn\endcsname{\let\PY@bf=\textbf\def\PY@tc##1{\textcolor[rgb]{0.00,0.00,1.00}{##1}}}
\expandafter\def\csname PY@tok@ne\endcsname{\let\PY@bf=\textbf\def\PY@tc##1{\textcolor[rgb]{0.82,0.25,0.23}{##1}}}
\expandafter\def\csname PY@tok@nv\endcsname{\def\PY@tc##1{\textcolor[rgb]{0.10,0.09,0.49}{##1}}}
\expandafter\def\csname PY@tok@no\endcsname{\def\PY@tc##1{\textcolor[rgb]{0.53,0.00,0.00}{##1}}}
\expandafter\def\csname PY@tok@nl\endcsname{\def\PY@tc##1{\textcolor[rgb]{0.63,0.63,0.00}{##1}}}
\expandafter\def\csname PY@tok@ni\endcsname{\let\PY@bf=\textbf\def\PY@tc##1{\textcolor[rgb]{0.60,0.60,0.60}{##1}}}
\expandafter\def\csname PY@tok@na\endcsname{\def\PY@tc##1{\textcolor[rgb]{0.49,0.56,0.16}{##1}}}
\expandafter\def\csname PY@tok@nt\endcsname{\let\PY@bf=\textbf\def\PY@tc##1{\textcolor[rgb]{0.00,0.50,0.00}{##1}}}
\expandafter\def\csname PY@tok@nd\endcsname{\def\PY@tc##1{\textcolor[rgb]{0.67,0.13,1.00}{##1}}}
\expandafter\def\csname PY@tok@s\endcsname{\def\PY@tc##1{\textcolor[rgb]{0.73,0.13,0.13}{##1}}}
\expandafter\def\csname PY@tok@sd\endcsname{\let\PY@it=\textit\def\PY@tc##1{\textcolor[rgb]{0.73,0.13,0.13}{##1}}}
\expandafter\def\csname PY@tok@si\endcsname{\let\PY@bf=\textbf\def\PY@tc##1{\textcolor[rgb]{0.73,0.40,0.53}{##1}}}
\expandafter\def\csname PY@tok@se\endcsname{\let\PY@bf=\textbf\def\PY@tc##1{\textcolor[rgb]{0.73,0.40,0.13}{##1}}}
\expandafter\def\csname PY@tok@sr\endcsname{\def\PY@tc##1{\textcolor[rgb]{0.73,0.40,0.53}{##1}}}
\expandafter\def\csname PY@tok@ss\endcsname{\def\PY@tc##1{\textcolor[rgb]{0.10,0.09,0.49}{##1}}}
\expandafter\def\csname PY@tok@sx\endcsname{\def\PY@tc##1{\textcolor[rgb]{0.00,0.50,0.00}{##1}}}
\expandafter\def\csname PY@tok@m\endcsname{\def\PY@tc##1{\textcolor[rgb]{0.40,0.40,0.40}{##1}}}
\expandafter\def\csname PY@tok@gh\endcsname{\let\PY@bf=\textbf\def\PY@tc##1{\textcolor[rgb]{0.00,0.00,0.50}{##1}}}
\expandafter\def\csname PY@tok@gu\endcsname{\let\PY@bf=\textbf\def\PY@tc##1{\textcolor[rgb]{0.50,0.00,0.50}{##1}}}
\expandafter\def\csname PY@tok@gd\endcsname{\def\PY@tc##1{\textcolor[rgb]{0.63,0.00,0.00}{##1}}}
\expandafter\def\csname PY@tok@gi\endcsname{\def\PY@tc##1{\textcolor[rgb]{0.00,0.63,0.00}{##1}}}
\expandafter\def\csname PY@tok@gr\endcsname{\def\PY@tc##1{\textcolor[rgb]{1.00,0.00,0.00}{##1}}}
\expandafter\def\csname PY@tok@ge\endcsname{\let\PY@it=\textit}
\expandafter\def\csname PY@tok@gs\endcsname{\let\PY@bf=\textbf}
\expandafter\def\csname PY@tok@gp\endcsname{\let\PY@bf=\textbf\def\PY@tc##1{\textcolor[rgb]{0.00,0.00,0.50}{##1}}}
\expandafter\def\csname PY@tok@go\endcsname{\def\PY@tc##1{\textcolor[rgb]{0.53,0.53,0.53}{##1}}}
\expandafter\def\csname PY@tok@gt\endcsname{\def\PY@tc##1{\textcolor[rgb]{0.00,0.27,0.87}{##1}}}
\expandafter\def\csname PY@tok@err\endcsname{\def\PY@bc##1{\setlength{\fboxsep}{0pt}\fcolorbox[rgb]{1.00,0.00,0.00}{1,1,1}{\strut ##1}}}
\expandafter\def\csname PY@tok@kc\endcsname{\let\PY@bf=\textbf\def\PY@tc##1{\textcolor[rgb]{0.00,0.50,0.00}{##1}}}
\expandafter\def\csname PY@tok@kd\endcsname{\let\PY@bf=\textbf\def\PY@tc##1{\textcolor[rgb]{0.00,0.50,0.00}{##1}}}
\expandafter\def\csname PY@tok@kn\endcsname{\let\PY@bf=\textbf\def\PY@tc##1{\textcolor[rgb]{0.00,0.50,0.00}{##1}}}
\expandafter\def\csname PY@tok@kr\endcsname{\let\PY@bf=\textbf\def\PY@tc##1{\textcolor[rgb]{0.00,0.50,0.00}{##1}}}
\expandafter\def\csname PY@tok@bp\endcsname{\def\PY@tc##1{\textcolor[rgb]{0.00,0.50,0.00}{##1}}}
\expandafter\def\csname PY@tok@fm\endcsname{\def\PY@tc##1{\textcolor[rgb]{0.00,0.00,1.00}{##1}}}
\expandafter\def\csname PY@tok@vc\endcsname{\def\PY@tc##1{\textcolor[rgb]{0.10,0.09,0.49}{##1}}}
\expandafter\def\csname PY@tok@vg\endcsname{\def\PY@tc##1{\textcolor[rgb]{0.10,0.09,0.49}{##1}}}
\expandafter\def\csname PY@tok@vi\endcsname{\def\PY@tc##1{\textcolor[rgb]{0.10,0.09,0.49}{##1}}}
\expandafter\def\csname PY@tok@vm\endcsname{\def\PY@tc##1{\textcolor[rgb]{0.10,0.09,0.49}{##1}}}
\expandafter\def\csname PY@tok@sa\endcsname{\def\PY@tc##1{\textcolor[rgb]{0.73,0.13,0.13}{##1}}}
\expandafter\def\csname PY@tok@sb\endcsname{\def\PY@tc##1{\textcolor[rgb]{0.73,0.13,0.13}{##1}}}
\expandafter\def\csname PY@tok@sc\endcsname{\def\PY@tc##1{\textcolor[rgb]{0.73,0.13,0.13}{##1}}}
\expandafter\def\csname PY@tok@dl\endcsname{\def\PY@tc##1{\textcolor[rgb]{0.73,0.13,0.13}{##1}}}
\expandafter\def\csname PY@tok@s2\endcsname{\def\PY@tc##1{\textcolor[rgb]{0.73,0.13,0.13}{##1}}}
\expandafter\def\csname PY@tok@sh\endcsname{\def\PY@tc##1{\textcolor[rgb]{0.73,0.13,0.13}{##1}}}
\expandafter\def\csname PY@tok@s1\endcsname{\def\PY@tc##1{\textcolor[rgb]{0.73,0.13,0.13}{##1}}}
\expandafter\def\csname PY@tok@mb\endcsname{\def\PY@tc##1{\textcolor[rgb]{0.40,0.40,0.40}{##1}}}
\expandafter\def\csname PY@tok@mf\endcsname{\def\PY@tc##1{\textcolor[rgb]{0.40,0.40,0.40}{##1}}}
\expandafter\def\csname PY@tok@mh\endcsname{\def\PY@tc##1{\textcolor[rgb]{0.40,0.40,0.40}{##1}}}
\expandafter\def\csname PY@tok@mi\endcsname{\def\PY@tc##1{\textcolor[rgb]{0.40,0.40,0.40}{##1}}}
\expandafter\def\csname PY@tok@il\endcsname{\def\PY@tc##1{\textcolor[rgb]{0.40,0.40,0.40}{##1}}}
\expandafter\def\csname PY@tok@mo\endcsname{\def\PY@tc##1{\textcolor[rgb]{0.40,0.40,0.40}{##1}}}
\expandafter\def\csname PY@tok@ch\endcsname{\let\PY@it=\textit\def\PY@tc##1{\textcolor[rgb]{0.25,0.50,0.50}{##1}}}
\expandafter\def\csname PY@tok@cm\endcsname{\let\PY@it=\textit\def\PY@tc##1{\textcolor[rgb]{0.25,0.50,0.50}{##1}}}
\expandafter\def\csname PY@tok@cpf\endcsname{\let\PY@it=\textit\def\PY@tc##1{\textcolor[rgb]{0.25,0.50,0.50}{##1}}}
\expandafter\def\csname PY@tok@c1\endcsname{\let\PY@it=\textit\def\PY@tc##1{\textcolor[rgb]{0.25,0.50,0.50}{##1}}}
\expandafter\def\csname PY@tok@cs\endcsname{\let\PY@it=\textit\def\PY@tc##1{\textcolor[rgb]{0.25,0.50,0.50}{##1}}}

\def\PYZbs{\char`\\}
\def\PYZus{\char`\_}
\def\PYZob{\char`\{}
\def\PYZcb{\char`\}}
\def\PYZca{\char`\^}
\def\PYZam{\char`\&}
\def\PYZlt{\char`\<}
\def\PYZgt{\char`\>}
\def\PYZsh{\char`\#}
\def\PYZpc{\char`\%}
\def\PYZdl{\char`\$}
\def\PYZhy{\char`\-}
\def\PYZsq{\char`\'}
\def\PYZdq{\char`\"}
\def\PYZti{\char`\~}
% for compatibility with earlier versions
\def\PYZat{@}
\def\PYZlb{[}
\def\PYZrb{]}
\makeatother


    % Exact colors from NB
    \definecolor{incolor}{rgb}{0.0, 0.0, 0.5}
    \definecolor{outcolor}{rgb}{0.545, 0.0, 0.0}



    
    % Prevent overflowing lines due to hard-to-break entities
    \sloppy 
    % Setup hyperref package
    \hypersetup{
      breaklinks=true,  % so long urls are correctly broken across lines
      colorlinks=true,
      urlcolor=urlcolor,
      linkcolor=linkcolor,
      citecolor=citecolor,
      }
    % Slightly bigger margins than the latex defaults
    
    \geometry{verbose,tmargin=1in,bmargin=1in,lmargin=1in,rmargin=1in}
    
    

    \begin{document}
    
    
    \maketitle
    
    

    
    \begin{Verbatim}[commandchars=\\\{\}]
{\color{incolor}In [{\color{incolor}1}]:} \PY{c+c1}{\PYZsh{} \PYZhy{}*\PYZhy{} coding: utf\PYZhy{}8 \PYZhy{}*\PYZhy{}}
        \PY{c+c1}{\PYZsh{} @author: tongzi}
        \PY{c+c1}{\PYZsh{} @created date: 2019/07/24}
        \PY{c+c1}{\PYZsh{} @description: Data Aggregation and Gouping Operations}
        \PY{c+c1}{\PYZsh{} @last modification:2019/07/14}
\end{Verbatim}


    \begin{Verbatim}[commandchars=\\\{\}]
{\color{incolor}In [{\color{incolor}2}]:} \PY{k+kn}{import} \PY{n+nn}{numpy}  \PY{k}{as} \PY{n+nn}{np}
        \PY{k+kn}{import} \PY{n+nn}{pandas} \PY{k}{as} \PY{n+nn}{pd}
        \PY{k+kn}{import} \PY{n+nn}{matplotlib}\PY{n+nn}{.}\PY{n+nn}{pyplot} \PY{k}{as} \PY{n+nn}{plt}
        \PY{k+kn}{import} \PY{n+nn}{seaborn} \PY{k}{as} \PY{n+nn}{sns}
        \PY{o}{\PYZpc{}}\PY{k}{matplotlib} notebook
\end{Verbatim}


    Categorizing a dataset and applying a function to each group, whether
aggregation or transformation, is often a critical component of a data
analysis workflow. After loading, merging, and prepraring a dataset, we
may need to compute group statistics or possibly \emph{pivot tables} for
reporting or visualization purposes. \emph{pandas} provides a flexible
\emph{groupby}() interface, enabling us to slice, dice, and summary
dataset in a natural way.

    One reason for the popularity of relational databases and SQL (which
stands for structured query language) is the ease with whcih data can be
joined, filtered, transformed, and aggregated. However, query languages
like SQL are somewhat constrained in the kind of group operations that
can be performed. As we will see, with the expressiveness of Python and
pandas, we can perform quit compplex group operations by utilizing any
function that accepts a pandas object or NumPy array. In this chapter,we
will learn how to:\\
(1)Splits a pandas object into pieces using one or more keys (in the
form of functions, arrays, or DataFrame columns)

(2)Calculate group summary statistics, like count, mean, standard
deviation or user-defined function

(3)Apply within-group transformation or other manipulations, like
normalization, linear regression, rank, or subset selection.

(4)Compute pivot table and cross tabulations

(5)Perform quantile analysis and other statistical analysis.

    \begin{quote}
Aggregation for time series data, a special case of groupby, is referred
to as \emph{resampling}() in this book and will receive treatment in
Chapter 11.
\end{quote}

    \hypertarget{groupby-mechanics}{%
\subsubsection{10.1 GroupBy Mechanics}\label{groupby-mechanics}}

    Hadly
Wickham,很多R编程语言流行包的作者,为描述分组操作创造一个术语``split-apply-combine''(分隔,应用,组合)。该过程的第一个阶段中,一个pandas对象中的数据不论是Series,DataFrame或者其它类型的数据会根据一个或者多个提供的键被分割成一个个组。分割操作是在一个对象的某个特定轴进行。比如说,一个DataFrame可以按照行或者列进行分割。分割完成后,每一个组应用一个函数并产生一个新的值。最后,所有应用到每个组的函数的结果组合到一起。组合的结果其形式取决于对数据执行了怎样的操作。如图所示:\\
\includegraphics{attachment:image.png}

    每一个分组的键可以是多种形式的,不一定都是相同的类型:\\
(1)一个列表或者数组,其中的值和要被分组的轴在长度是相等的\\
(2)DataFrame中的一个列名\\
(3)一个字典或者Series,它给出了要给分组的轴的值与该组名字的对应关系\\
(4)一个函数,该函数在轴索引或某索引的单个标签上进行调用

    \begin{Verbatim}[commandchars=\\\{\}]
{\color{incolor}In [{\color{incolor}3}]:} \PY{n}{df} \PY{o}{=} \PY{n}{pd}\PY{o}{.}\PY{n}{DataFrame}\PY{p}{(}\PY{p}{\PYZob{}}\PY{l+s+s1}{\PYZsq{}}\PY{l+s+s1}{key1}\PY{l+s+s1}{\PYZsq{}} \PY{p}{:} \PY{p}{[}\PY{l+s+s1}{\PYZsq{}}\PY{l+s+s1}{a}\PY{l+s+s1}{\PYZsq{}}\PY{p}{,} \PY{l+s+s1}{\PYZsq{}}\PY{l+s+s1}{a}\PY{l+s+s1}{\PYZsq{}}\PY{p}{,} \PY{l+s+s1}{\PYZsq{}}\PY{l+s+s1}{b}\PY{l+s+s1}{\PYZsq{}}\PY{p}{,} \PY{l+s+s1}{\PYZsq{}}\PY{l+s+s1}{b}\PY{l+s+s1}{\PYZsq{}}\PY{p}{,} \PY{l+s+s1}{\PYZsq{}}\PY{l+s+s1}{a}\PY{l+s+s1}{\PYZsq{}}\PY{p}{]}\PY{p}{,}
        \PY{o}{.}\PY{o}{.}\PY{o}{.}\PY{o}{.}\PY{p}{:} \PY{l+s+s1}{\PYZsq{}}\PY{l+s+s1}{key2}\PY{l+s+s1}{\PYZsq{}} \PY{p}{:} \PY{p}{[}\PY{l+s+s1}{\PYZsq{}}\PY{l+s+s1}{one}\PY{l+s+s1}{\PYZsq{}}\PY{p}{,} \PY{l+s+s1}{\PYZsq{}}\PY{l+s+s1}{two}\PY{l+s+s1}{\PYZsq{}}\PY{p}{,} \PY{l+s+s1}{\PYZsq{}}\PY{l+s+s1}{one}\PY{l+s+s1}{\PYZsq{}}\PY{p}{,} \PY{l+s+s1}{\PYZsq{}}\PY{l+s+s1}{two}\PY{l+s+s1}{\PYZsq{}}\PY{p}{,} \PY{l+s+s1}{\PYZsq{}}\PY{l+s+s1}{one}\PY{l+s+s1}{\PYZsq{}}\PY{p}{]}\PY{p}{,}
        \PY{o}{.}\PY{o}{.}\PY{o}{.}\PY{o}{.}\PY{p}{:} \PY{l+s+s1}{\PYZsq{}}\PY{l+s+s1}{data1}\PY{l+s+s1}{\PYZsq{}} \PY{p}{:} \PY{n}{np}\PY{o}{.}\PY{n}{random}\PY{o}{.}\PY{n}{randn}\PY{p}{(}\PY{l+m+mi}{5}\PY{p}{)}\PY{p}{,}
        \PY{o}{.}\PY{o}{.}\PY{o}{.}\PY{o}{.}\PY{p}{:} \PY{l+s+s1}{\PYZsq{}}\PY{l+s+s1}{data2}\PY{l+s+s1}{\PYZsq{}} \PY{p}{:} \PY{n}{np}\PY{o}{.}\PY{n}{random}\PY{o}{.}\PY{n}{randn}\PY{p}{(}\PY{l+m+mi}{5}\PY{p}{)}\PY{p}{\PYZcb{}}\PY{p}{)}
\end{Verbatim}


    \begin{Verbatim}[commandchars=\\\{\}]
{\color{incolor}In [{\color{incolor}4}]:} \PY{n}{df}
\end{Verbatim}


\begin{Verbatim}[commandchars=\\\{\}]
{\color{outcolor}Out[{\color{outcolor}4}]:}       data1     data2 key1 key2
        0  0.714162 -0.917054    a  one
        1 -0.251975 -0.335887    a  two
        2  0.056946 -2.165745    b  one
        3  0.718219  2.431233    b  two
        4 -0.388676 -1.043849    a  one
\end{Verbatim}
            
    Suppose we want to compute the mean of the \emph{data1} using the labels
frin \emph{key1}. There are a number of ways to do this, One is to
access \emph{data1} and call \emph{groupby}() with the column (a Series)
at \emph{key1}:

    \begin{Verbatim}[commandchars=\\\{\}]
{\color{incolor}In [{\color{incolor}5}]:} \PY{n}{grouped} \PY{o}{=} \PY{n}{df}\PY{p}{[}\PY{l+s+s1}{\PYZsq{}}\PY{l+s+s1}{data1}\PY{l+s+s1}{\PYZsq{}}\PY{p}{]}\PY{o}{.}\PY{n}{groupby}\PY{p}{(}\PY{n}{df}\PY{p}{[}\PY{l+s+s1}{\PYZsq{}}\PY{l+s+s1}{key1}\PY{l+s+s1}{\PYZsq{}}\PY{p}{]}\PY{p}{)}
\end{Verbatim}


    \begin{Verbatim}[commandchars=\\\{\}]
{\color{incolor}In [{\color{incolor}6}]:} \PY{n}{grouped}
\end{Verbatim}


\begin{Verbatim}[commandchars=\\\{\}]
{\color{outcolor}Out[{\color{outcolor}6}]:} <pandas.core.groupby.SeriesGroupBy object at 0x000000000AD8F898>
\end{Verbatim}
            
    This \emph{grouped} variable is now a GroupBy object. It has not actualy
computed anything except for some intermediate data about the group key
df{[}`key1'{]}. The idea is that this object has all of the information
needed to apply some group operation to each of the groups. For example,
to compute group means we can the GoupBy's \emph{mean}() method:

    \begin{Verbatim}[commandchars=\\\{\}]
{\color{incolor}In [{\color{incolor}7}]:} \PY{n}{grouped}\PY{o}{.}\PY{n}{mean}\PY{p}{(}\PY{p}{)}
\end{Verbatim}


\begin{Verbatim}[commandchars=\\\{\}]
{\color{outcolor}Out[{\color{outcolor}7}]:} key1
        a    0.024504
        b    0.387583
        Name: data1, dtype: float64
\end{Verbatim}
            
    Later, we will explain more about what happens when we call
\emph{.mean}(). The important thing here is that the data (Series) has
been aggregated according to the group key, producing a new Series that
is now indexed by the unique values in the \emph{key1} column.

The result index has the name `key1' because the DataFrame column
df{[}`key1'{]} did.

If instead we had passed multiple arrays as a list, we'd get something
different:

    \begin{Verbatim}[commandchars=\\\{\}]
{\color{incolor}In [{\color{incolor}8}]:} \PY{n}{means} \PY{o}{=} \PY{n}{df}\PY{p}{[}\PY{l+s+s1}{\PYZsq{}}\PY{l+s+s1}{data1}\PY{l+s+s1}{\PYZsq{}}\PY{p}{]}\PY{o}{.}\PY{n}{groupby}\PY{p}{(}\PY{p}{[}\PY{n}{df}\PY{p}{[}\PY{l+s+s1}{\PYZsq{}}\PY{l+s+s1}{key1}\PY{l+s+s1}{\PYZsq{}}\PY{p}{]}\PY{p}{,} \PY{n}{df}\PY{p}{[}\PY{l+s+s1}{\PYZsq{}}\PY{l+s+s1}{key2}\PY{l+s+s1}{\PYZsq{}}\PY{p}{]}\PY{p}{]}\PY{p}{)}\PY{o}{.}\PY{n}{mean}\PY{p}{(}\PY{p}{)}
\end{Verbatim}


    \begin{Verbatim}[commandchars=\\\{\}]
{\color{incolor}In [{\color{incolor}9}]:} \PY{n}{means}
\end{Verbatim}


\begin{Verbatim}[commandchars=\\\{\}]
{\color{outcolor}Out[{\color{outcolor}9}]:} key1  key2
        a     one     0.162743
              two    -0.251975
        b     one     0.056946
              two     0.718219
        Name: data1, dtype: float64
\end{Verbatim}
            
    Here we grouped the data using two keys, and the resulting Series has a
hierarchical index consisting of the unique pairs of keys observed.

    \begin{Verbatim}[commandchars=\\\{\}]
{\color{incolor}In [{\color{incolor}10}]:} \PY{n}{means}\PY{o}{.}\PY{n}{unstack}\PY{p}{(}\PY{p}{)}
\end{Verbatim}


\begin{Verbatim}[commandchars=\\\{\}]
{\color{outcolor}Out[{\color{outcolor}10}]:} key2       one       two
         key1                    
         a     0.162743 -0.251975
         b     0.056946  0.718219
\end{Verbatim}
            
    In this example, the group keys are all Series, though they can be any
arrays of the right length:

    \begin{Verbatim}[commandchars=\\\{\}]
{\color{incolor}In [{\color{incolor}11}]:} \PY{n}{states} \PY{o}{=} \PY{n}{np}\PY{o}{.}\PY{n}{array}\PY{p}{(}\PY{p}{[}\PY{l+s+s1}{\PYZsq{}}\PY{l+s+s1}{Ohio}\PY{l+s+s1}{\PYZsq{}}\PY{p}{,} \PY{l+s+s1}{\PYZsq{}}\PY{l+s+s1}{California}\PY{l+s+s1}{\PYZsq{}}\PY{p}{,} \PY{l+s+s1}{\PYZsq{}}\PY{l+s+s1}{California}\PY{l+s+s1}{\PYZsq{}}\PY{p}{,} \PY{l+s+s1}{\PYZsq{}}\PY{l+s+s1}{Ohio}\PY{l+s+s1}{\PYZsq{}}\PY{p}{,} \PY{l+s+s1}{\PYZsq{}}\PY{l+s+s1}{Ohio}\PY{l+s+s1}{\PYZsq{}}\PY{p}{]}\PY{p}{)}
\end{Verbatim}


    \begin{Verbatim}[commandchars=\\\{\}]
{\color{incolor}In [{\color{incolor}12}]:} \PY{n}{years} \PY{o}{=} \PY{n}{np}\PY{o}{.}\PY{n}{array}\PY{p}{(}\PY{p}{[}\PY{l+m+mi}{2005}\PY{p}{,} \PY{l+m+mi}{2005}\PY{p}{,} \PY{l+m+mi}{2006}\PY{p}{,} \PY{l+m+mi}{2005}\PY{p}{,} \PY{l+m+mi}{2006}\PY{p}{]}\PY{p}{)}
\end{Verbatim}


    \begin{Verbatim}[commandchars=\\\{\}]
{\color{incolor}In [{\color{incolor}13}]:} \PY{n}{df}\PY{p}{[}\PY{l+s+s1}{\PYZsq{}}\PY{l+s+s1}{data1}\PY{l+s+s1}{\PYZsq{}}\PY{p}{]}\PY{o}{.}\PY{n}{groupby}\PY{p}{(}\PY{p}{[}\PY{n}{states}\PY{p}{,} \PY{n}{years}\PY{p}{]}\PY{p}{)}\PY{o}{.}\PY{n}{mean}\PY{p}{(}\PY{p}{)}
\end{Verbatim}


\begin{Verbatim}[commandchars=\\\{\}]
{\color{outcolor}Out[{\color{outcolor}13}]:} California  2005   -0.251975
                     2006    0.056946
         Ohio        2005    0.716191
                     2006   -0.388676
         Name: data1, dtype: float64
\end{Verbatim}
            
    Frequently the grouping information is found in the same DataFrame as
the data you work on. In that case, we can pass column names (whether
those are strings, numbers, or other Python object) as the group keys:

    \begin{Verbatim}[commandchars=\\\{\}]
{\color{incolor}In [{\color{incolor}14}]:} \PY{n}{df}\PY{o}{.}\PY{n}{groupby}\PY{p}{(}\PY{l+s+s1}{\PYZsq{}}\PY{l+s+s1}{key1}\PY{l+s+s1}{\PYZsq{}}\PY{p}{)}\PY{o}{.}\PY{n}{mean}\PY{p}{(}\PY{p}{)}
\end{Verbatim}


\begin{Verbatim}[commandchars=\\\{\}]
{\color{outcolor}Out[{\color{outcolor}14}]:}          data1     data2
         key1                    
         a     0.024504 -0.765597
         b     0.387583  0.132744
\end{Verbatim}
            
    \begin{Verbatim}[commandchars=\\\{\}]
{\color{incolor}In [{\color{incolor}15}]:} \PY{n}{df}\PY{o}{.}\PY{n}{groupby}\PY{p}{(}\PY{p}{[}\PY{l+s+s1}{\PYZsq{}}\PY{l+s+s1}{key1}\PY{l+s+s1}{\PYZsq{}}\PY{p}{,} \PY{l+s+s1}{\PYZsq{}}\PY{l+s+s1}{key2}\PY{l+s+s1}{\PYZsq{}}\PY{p}{]}\PY{p}{)}\PY{o}{.}\PY{n}{mean}\PY{p}{(}\PY{p}{)}
\end{Verbatim}


\begin{Verbatim}[commandchars=\\\{\}]
{\color{outcolor}Out[{\color{outcolor}15}]:}               data1     data2
         key1 key2                    
         a    one   0.162743 -0.980452
              two  -0.251975 -0.335887
         b    one   0.056946 -2.165745
              two   0.718219  2.431233
\end{Verbatim}
            
    You may have noticed in the first case \emph{df{[}`key1'{]}.mean}() that
there is no \emph{key2} column in the result. Because
\emph{df{[}`key2'{]}} is not numeric data, it is said to a
\emph{nuisance column} (讨厌的列), which is excluded from the result. By
default, all of the numeric columns are aggregated, though it is
possible to filter down to a subset, as we'll see soon.

    \begin{Verbatim}[commandchars=\\\{\}]
{\color{incolor}In [{\color{incolor}16}]:} \PY{n}{df}
\end{Verbatim}


\begin{Verbatim}[commandchars=\\\{\}]
{\color{outcolor}Out[{\color{outcolor}16}]:}       data1     data2 key1 key2
         0  0.714162 -0.917054    a  one
         1 -0.251975 -0.335887    a  two
         2  0.056946 -2.165745    b  one
         3  0.718219  2.431233    b  two
         4 -0.388676 -1.043849    a  one
\end{Verbatim}
            
    Regardless of the objective in using \emph{groupby}(), a generally
useful GroupBy method is \emph{size}(), which returns a Series
containing group sizes:

    \begin{Verbatim}[commandchars=\\\{\}]
{\color{incolor}In [{\color{incolor}17}]:} \PY{n}{df}\PY{o}{.}\PY{n}{groupby}\PY{p}{(}\PY{p}{[}\PY{l+s+s1}{\PYZsq{}}\PY{l+s+s1}{key1}\PY{l+s+s1}{\PYZsq{}}\PY{p}{,} \PY{l+s+s1}{\PYZsq{}}\PY{l+s+s1}{key2}\PY{l+s+s1}{\PYZsq{}}\PY{p}{]}\PY{p}{)}\PY{o}{.}\PY{n}{size}\PY{p}{(}\PY{p}{)}
\end{Verbatim}


\begin{Verbatim}[commandchars=\\\{\}]
{\color{outcolor}Out[{\color{outcolor}17}]:} key1  key2
         a     one     2
               two     1
         b     one     1
               two     1
         dtype: int64
\end{Verbatim}
            
    \begin{quote}
Take note that any missing values in a group key will be excluded from
the result.
\end{quote}

    \hypertarget{iterating-over-groups}{%
\paragraph{Iterating Over Groups}\label{iterating-over-groups}}

    The GroupBy object supports iteration, generating a sequence of 2-tuples
(2元素元组) containing the group name along with the chunk of data.
Consider the following:

    \begin{Verbatim}[commandchars=\\\{\}]
{\color{incolor}In [{\color{incolor}18}]:} \PY{n+nb}{list}\PY{p}{(}\PY{n}{df}\PY{o}{.}\PY{n}{groupby}\PY{p}{(}\PY{l+s+s1}{\PYZsq{}}\PY{l+s+s1}{key1}\PY{l+s+s1}{\PYZsq{}}\PY{p}{)}\PY{p}{)}
\end{Verbatim}


\begin{Verbatim}[commandchars=\\\{\}]
{\color{outcolor}Out[{\color{outcolor}18}]:} [('a',       data1     data2 key1 key2
           0  0.714162 -0.917054    a  one
           1 -0.251975 -0.335887    a  two
           4 -0.388676 -1.043849    a  one), ('b',       data1     data2 key1 key2
           2  0.056946 -2.165745    b  one
           3  0.718219  2.431233    b  two)]
\end{Verbatim}
            
    \begin{Verbatim}[commandchars=\\\{\}]
{\color{incolor}In [{\color{incolor}19}]:} \PY{k}{for} \PY{n}{name}\PY{p}{,} \PY{n}{group} \PY{o+ow}{in} \PY{n}{df}\PY{o}{.}\PY{n}{groupby}\PY{p}{(}\PY{l+s+s1}{\PYZsq{}}\PY{l+s+s1}{key1}\PY{l+s+s1}{\PYZsq{}}\PY{p}{)}\PY{p}{:}
             \PY{n+nb}{print}\PY{p}{(}\PY{n}{name}\PY{p}{)}
             \PY{n+nb}{print}\PY{p}{(}\PY{n}{group}\PY{p}{)}
\end{Verbatim}


    \begin{Verbatim}[commandchars=\\\{\}]
a
      data1     data2 key1 key2
0  0.714162 -0.917054    a  one
1 -0.251975 -0.335887    a  two
4 -0.388676 -1.043849    a  one
b
      data1     data2 key1 key2
2  0.056946 -2.165745    b  one
3  0.718219  2.431233    b  two

    \end{Verbatim}

    In the case of multiple keys, the first element in the tuple will be a
tuple of key values:

    \begin{Verbatim}[commandchars=\\\{\}]
{\color{incolor}In [{\color{incolor}20}]:} \PY{k}{for} \PY{p}{(}\PY{n}{k1}\PY{p}{,} \PY{n}{k2}\PY{p}{)}\PY{p}{,} \PY{n}{group} \PY{o+ow}{in} \PY{n}{df}\PY{o}{.}\PY{n}{groupby}\PY{p}{(}\PY{p}{[}\PY{l+s+s1}{\PYZsq{}}\PY{l+s+s1}{key1}\PY{l+s+s1}{\PYZsq{}}\PY{p}{,} \PY{l+s+s1}{\PYZsq{}}\PY{l+s+s1}{key2}\PY{l+s+s1}{\PYZsq{}}\PY{p}{]}\PY{p}{)}\PY{p}{:}
             \PY{n+nb}{print}\PY{p}{(}\PY{p}{(}\PY{n}{k1}\PY{p}{,} \PY{n}{k2}\PY{p}{)}\PY{p}{)}
             \PY{n+nb}{print}\PY{p}{(}\PY{n}{group}\PY{p}{)}
\end{Verbatim}


    \begin{Verbatim}[commandchars=\\\{\}]
('a', 'one')
      data1     data2 key1 key2
0  0.714162 -0.917054    a  one
4 -0.388676 -1.043849    a  one
('a', 'two')
      data1     data2 key1 key2
1 -0.251975 -0.335887    a  two
('b', 'one')
      data1     data2 key1 key2
2  0.056946 -2.165745    b  one
('b', 'two')
      data1     data2 key1 key2
3  0.718219  2.431233    b  two

    \end{Verbatim}

    Of course, we can choose to whateve we want with the pieces of data, A
recipe you may find useful is computing a dict of the pieces as
one-liner (一行):

    \begin{Verbatim}[commandchars=\\\{\}]
{\color{incolor}In [{\color{incolor}21}]:} \PY{n}{pieces} \PY{o}{=} \PY{n+nb}{dict}\PY{p}{(}\PY{n+nb}{list}\PY{p}{(}\PY{n}{df}\PY{o}{.}\PY{n}{groupby}\PY{p}{(}\PY{l+s+s1}{\PYZsq{}}\PY{l+s+s1}{key1}\PY{l+s+s1}{\PYZsq{}}\PY{p}{)}\PY{p}{)}\PY{p}{)}
\end{Verbatim}


    \begin{Verbatim}[commandchars=\\\{\}]
{\color{incolor}In [{\color{incolor}22}]:} \PY{n}{pieces}\PY{p}{[}\PY{l+s+s1}{\PYZsq{}}\PY{l+s+s1}{a}\PY{l+s+s1}{\PYZsq{}}\PY{p}{]}
\end{Verbatim}


\begin{Verbatim}[commandchars=\\\{\}]
{\color{outcolor}Out[{\color{outcolor}22}]:}       data1     data2 key1 key2
         0  0.714162 -0.917054    a  one
         1 -0.251975 -0.335887    a  two
         4 -0.388676 -1.043849    a  one
\end{Verbatim}
            
    By default, \emph{groupby}() groups on axis=0, but we can group on any
of the other axes. For example, we could group the columns of our
example \emph{df} by \textbf{dtype} like so:

    \begin{Verbatim}[commandchars=\\\{\}]
{\color{incolor}In [{\color{incolor}23}]:} \PY{n}{df}\PY{o}{.}\PY{n}{dtypes}
\end{Verbatim}


\begin{Verbatim}[commandchars=\\\{\}]
{\color{outcolor}Out[{\color{outcolor}23}]:} data1    float64
         data2    float64
         key1      object
         key2      object
         dtype: object
\end{Verbatim}
            
    \begin{Verbatim}[commandchars=\\\{\}]
{\color{incolor}In [{\color{incolor}24}]:} \PY{n}{grouped} \PY{o}{=} \PY{n}{df}\PY{o}{.}\PY{n}{groupby}\PY{p}{(}\PY{n}{df}\PY{o}{.}\PY{n}{dtypes}\PY{p}{,} \PY{n}{axis}\PY{o}{=}\PY{l+m+mi}{1}\PY{p}{)}
\end{Verbatim}


    We can print out the groups like so:

    \begin{Verbatim}[commandchars=\\\{\}]
{\color{incolor}In [{\color{incolor}25}]:} \PY{k}{for} \PY{n}{dtype}\PY{p}{,} \PY{n}{group} \PY{o+ow}{in} \PY{n}{grouped}\PY{p}{:}
             \PY{n+nb}{print}\PY{p}{(}\PY{n}{dtype}\PY{p}{)}
             \PY{n+nb}{print}\PY{p}{(}\PY{n}{group}\PY{p}{)}
\end{Verbatim}


    \begin{Verbatim}[commandchars=\\\{\}]
float64
      data1     data2
0  0.714162 -0.917054
1 -0.251975 -0.335887
2  0.056946 -2.165745
3  0.718219  2.431233
4 -0.388676 -1.043849
object
  key1 key2
0    a  one
1    a  two
2    b  one
3    b  two
4    a  one

    \end{Verbatim}

    \hypertarget{selecting-a-column-or-subset-of-columns}{%
\paragraph{Selecting a Column or Subset of
Columns}\label{selecting-a-column-or-subset-of-columns}}

    Indexing a GroupBy object created from a DataFrame with a column name or
array of column names has the effect of column subsetting for
aggregation. This means that:

      \emph{df.groupby(`key1'){[}`data1'{]}}\\
  \emph{df.groupby(`key1'){[}{[}`data2'{]}{]}}

    are syntactic sugar for:

      \emph{df{[}`data1'{]}.groupby(df{[}`key1'{]})}\\
  \emph{df{[}{[}`data2'{]}{]}.groupby(df{[}`key2'{]})}

    \begin{Verbatim}[commandchars=\\\{\}]
{\color{incolor}In [{\color{incolor}26}]:} \PY{n}{df}\PY{p}{[}\PY{l+s+s1}{\PYZsq{}}\PY{l+s+s1}{data2}\PY{l+s+s1}{\PYZsq{}}\PY{p}{]} \PY{c+c1}{\PYZsh{} 这种方式会返回一个Series对象}
\end{Verbatim}


\begin{Verbatim}[commandchars=\\\{\}]
{\color{outcolor}Out[{\color{outcolor}26}]:} 0   -0.917054
         1   -0.335887
         2   -2.165745
         3    2.431233
         4   -1.043849
         Name: data2, dtype: float64
\end{Verbatim}
            
    \begin{Verbatim}[commandchars=\\\{\}]
{\color{incolor}In [{\color{incolor}27}]:} \PY{n}{df}\PY{p}{[}\PY{p}{[}\PY{l+s+s1}{\PYZsq{}}\PY{l+s+s1}{data2}\PY{l+s+s1}{\PYZsq{}}\PY{p}{]}\PY{p}{]} \PY{c+c1}{\PYZsh{} 这种方式会返回一个DataFrame对象}
\end{Verbatim}


\begin{Verbatim}[commandchars=\\\{\}]
{\color{outcolor}Out[{\color{outcolor}27}]:}       data2
         0 -0.917054
         1 -0.335887
         2 -2.165745
         3  2.431233
         4 -1.043849
\end{Verbatim}
            
    对于大的数据集,可能指向聚合一列数据,比方说,在前面的数据集中,只想计算\emph{data2}列的均值,并且其结果作为一个DataFrame返回,可以写:

    \begin{Verbatim}[commandchars=\\\{\}]
{\color{incolor}In [{\color{incolor}28}]:} \PY{n}{df}\PY{o}{.}\PY{n}{groupby}\PY{p}{(}\PY{p}{[}\PY{l+s+s1}{\PYZsq{}}\PY{l+s+s1}{key1}\PY{l+s+s1}{\PYZsq{}}\PY{p}{,} \PY{l+s+s1}{\PYZsq{}}\PY{l+s+s1}{key2}\PY{l+s+s1}{\PYZsq{}}\PY{p}{]}\PY{p}{)}\PY{p}{[}\PY{p}{[}\PY{l+s+s1}{\PYZsq{}}\PY{l+s+s1}{data2}\PY{l+s+s1}{\PYZsq{}}\PY{p}{]}\PY{p}{]}\PY{o}{.}\PY{n}{mean}\PY{p}{(}\PY{p}{)}
\end{Verbatim}


\begin{Verbatim}[commandchars=\\\{\}]
{\color{outcolor}Out[{\color{outcolor}28}]:}               data2
         key1 key2          
         a    one  -0.980452
              two  -0.335887
         b    one  -2.165745
              two   2.431233
\end{Verbatim}
            
    The object returned by this indexing operation is a grouped DataFrame if
a list or array is passed or a grouped Series if only a single column
name is passed as a scalar:

    \begin{Verbatim}[commandchars=\\\{\}]
{\color{incolor}In [{\color{incolor}29}]:} \PY{n}{s\PYZus{}grouped} \PY{o}{=} \PY{n}{df}\PY{o}{.}\PY{n}{groupby}\PY{p}{(}\PY{p}{[}\PY{l+s+s1}{\PYZsq{}}\PY{l+s+s1}{key1}\PY{l+s+s1}{\PYZsq{}}\PY{p}{,} \PY{l+s+s1}{\PYZsq{}}\PY{l+s+s1}{key2}\PY{l+s+s1}{\PYZsq{}}\PY{p}{]}\PY{p}{)}\PY{p}{[}\PY{l+s+s1}{\PYZsq{}}\PY{l+s+s1}{data2}\PY{l+s+s1}{\PYZsq{}}\PY{p}{]}
\end{Verbatim}


    \begin{Verbatim}[commandchars=\\\{\}]
{\color{incolor}In [{\color{incolor}30}]:} \PY{n}{s\PYZus{}grouped}
\end{Verbatim}


\begin{Verbatim}[commandchars=\\\{\}]
{\color{outcolor}Out[{\color{outcolor}30}]:} <pandas.core.groupby.SeriesGroupBy object at 0x000000000AE00390>
\end{Verbatim}
            
    \begin{Verbatim}[commandchars=\\\{\}]
{\color{incolor}In [{\color{incolor}31}]:} \PY{n}{s\PYZus{}grouped}\PY{o}{.}\PY{n}{mean}\PY{p}{(}\PY{p}{)}
\end{Verbatim}


\begin{Verbatim}[commandchars=\\\{\}]
{\color{outcolor}Out[{\color{outcolor}31}]:} key1  key2
         a     one    -0.980452
               two    -0.335887
         b     one    -2.165745
               two     2.431233
         Name: data2, dtype: float64
\end{Verbatim}
            
    \hypertarget{grouping-with-dicts-and-series}{%
\paragraph{Grouping with Dicts and
Series}\label{grouping-with-dicts-and-series}}
Grouping information may exist in a form other than an array. Let's consider another example DataFrame:
    \begin{Verbatim}[commandchars=\\\{\}]
{\color{incolor}In [{\color{incolor}32}]:} \PY{n}{people} \PY{o}{=} \PY{n}{pd}\PY{o}{.}\PY{n}{DataFrame}\PY{p}{(}\PY{n}{np}\PY{o}{.}\PY{n}{random}\PY{o}{.}\PY{n}{randn}\PY{p}{(}\PY{l+m+mi}{5}\PY{p}{,} \PY{l+m+mi}{5}\PY{p}{)}\PY{p}{,}
         \PY{o}{.}\PY{o}{.}\PY{o}{.}\PY{o}{.}\PY{p}{:} \PY{n}{columns}\PY{o}{=}\PY{p}{[}\PY{l+s+s1}{\PYZsq{}}\PY{l+s+s1}{a}\PY{l+s+s1}{\PYZsq{}}\PY{p}{,} \PY{l+s+s1}{\PYZsq{}}\PY{l+s+s1}{b}\PY{l+s+s1}{\PYZsq{}}\PY{p}{,} \PY{l+s+s1}{\PYZsq{}}\PY{l+s+s1}{c}\PY{l+s+s1}{\PYZsq{}}\PY{p}{,} \PY{l+s+s1}{\PYZsq{}}\PY{l+s+s1}{d}\PY{l+s+s1}{\PYZsq{}}\PY{p}{,} \PY{l+s+s1}{\PYZsq{}}\PY{l+s+s1}{e}\PY{l+s+s1}{\PYZsq{}}\PY{p}{]}\PY{p}{,}
         \PY{o}{.}\PY{o}{.}\PY{o}{.}\PY{o}{.}\PY{p}{:} \PY{n}{index}\PY{o}{=}\PY{p}{[}\PY{l+s+s1}{\PYZsq{}}\PY{l+s+s1}{Joe}\PY{l+s+s1}{\PYZsq{}}\PY{p}{,} \PY{l+s+s1}{\PYZsq{}}\PY{l+s+s1}{Steve}\PY{l+s+s1}{\PYZsq{}}\PY{p}{,} \PY{l+s+s1}{\PYZsq{}}\PY{l+s+s1}{Wes}\PY{l+s+s1}{\PYZsq{}}\PY{p}{,} \PY{l+s+s1}{\PYZsq{}}\PY{l+s+s1}{Jim}\PY{l+s+s1}{\PYZsq{}}\PY{p}{,} \PY{l+s+s1}{\PYZsq{}}\PY{l+s+s1}{Travis}\PY{l+s+s1}{\PYZsq{}}\PY{p}{]}\PY{p}{)}
\end{Verbatim}


    \begin{Verbatim}[commandchars=\\\{\}]
{\color{incolor}In [{\color{incolor}33}]:} \PY{n}{people}\PY{o}{.}\PY{n}{iloc}\PY{p}{[}\PY{l+m+mi}{2}\PY{p}{:}\PY{l+m+mi}{3}\PY{p}{,} \PY{p}{[}\PY{l+m+mi}{1}\PY{p}{,}\PY{l+m+mi}{2}\PY{p}{]}\PY{p}{]} \PY{o}{=} \PY{n}{np}\PY{o}{.}\PY{n}{nan}
\end{Verbatim}


    \begin{Verbatim}[commandchars=\\\{\}]
{\color{incolor}In [{\color{incolor}34}]:} \PY{n}{people}
\end{Verbatim}


\begin{Verbatim}[commandchars=\\\{\}]
{\color{outcolor}Out[{\color{outcolor}34}]:}                a         b         c         d         e
         Joe    -0.840949 -0.712837  0.385332  0.843450 -0.938942
         Steve   0.393219  0.398256  0.401803 -1.538065  1.365792
         Wes    -0.126382       NaN       NaN -0.426163  0.657845
         Jim     0.873312 -0.739668 -0.121883 -0.450816  0.770976
         Travis -0.397205 -0.346472  0.656310 -0.438620  0.205482
\end{Verbatim}
            
    Now, suppose I have a group correspondence for the columns and want to
sum together the columns by group:

    \begin{Verbatim}[commandchars=\\\{\}]
{\color{incolor}In [{\color{incolor}35}]:} \PY{n}{mapping} \PY{o}{=} \PY{p}{\PYZob{}}\PY{l+s+s1}{\PYZsq{}}\PY{l+s+s1}{a}\PY{l+s+s1}{\PYZsq{}}\PY{p}{:} \PY{l+s+s1}{\PYZsq{}}\PY{l+s+s1}{red}\PY{l+s+s1}{\PYZsq{}}\PY{p}{,} \PY{l+s+s1}{\PYZsq{}}\PY{l+s+s1}{b}\PY{l+s+s1}{\PYZsq{}}\PY{p}{:} \PY{l+s+s1}{\PYZsq{}}\PY{l+s+s1}{red}\PY{l+s+s1}{\PYZsq{}}\PY{p}{,} \PY{l+s+s1}{\PYZsq{}}\PY{l+s+s1}{c}\PY{l+s+s1}{\PYZsq{}}\PY{p}{:} \PY{l+s+s1}{\PYZsq{}}\PY{l+s+s1}{blue}\PY{l+s+s1}{\PYZsq{}}\PY{p}{,}
         \PY{o}{.}\PY{o}{.}\PY{o}{.}\PY{o}{.}\PY{p}{:} \PY{l+s+s1}{\PYZsq{}}\PY{l+s+s1}{d}\PY{l+s+s1}{\PYZsq{}}\PY{p}{:} \PY{l+s+s1}{\PYZsq{}}\PY{l+s+s1}{blue}\PY{l+s+s1}{\PYZsq{}}\PY{p}{,} \PY{l+s+s1}{\PYZsq{}}\PY{l+s+s1}{e}\PY{l+s+s1}{\PYZsq{}}\PY{p}{:} \PY{l+s+s1}{\PYZsq{}}\PY{l+s+s1}{red}\PY{l+s+s1}{\PYZsq{}}\PY{p}{,} \PY{l+s+s1}{\PYZsq{}}\PY{l+s+s1}{f}\PY{l+s+s1}{\PYZsq{}} \PY{p}{:} \PY{l+s+s1}{\PYZsq{}}\PY{l+s+s1}{orange}\PY{l+s+s1}{\PYZsq{}}\PY{p}{\PYZcb{}}
\end{Verbatim}


    Now we can construct an array from this dict to pass to
\emph{groupby}(), but instead we can just the dict (I included the key
\emph{`f'} to highlight that unused grouping keys are OK):

    \begin{Verbatim}[commandchars=\\\{\}]
{\color{incolor}In [{\color{incolor}36}]:} \PY{n}{by\PYZus{}column} \PY{o}{=} \PY{n}{people}\PY{o}{.}\PY{n}{groupby}\PY{p}{(}\PY{n}{mapping}\PY{p}{,} \PY{n}{axis}\PY{o}{=}\PY{l+m+mi}{1}\PY{p}{)}
\end{Verbatim}


    \begin{Verbatim}[commandchars=\\\{\}]
{\color{incolor}In [{\color{incolor}37}]:} \PY{n}{by\PYZus{}column}\PY{o}{.}\PY{n}{sum}\PY{p}{(}\PY{p}{)}
\end{Verbatim}


\begin{Verbatim}[commandchars=\\\{\}]
{\color{outcolor}Out[{\color{outcolor}37}]:}             blue       red
         Joe     1.228782 -2.492729
         Steve  -1.136261  2.157267
         Wes    -0.426163  0.531463
         Jim    -0.572699  0.904620
         Travis  0.217690 -0.538195
\end{Verbatim}
            
    The same functionality holds for Series, which can be viewed as a
fix-size mapping:

    \begin{Verbatim}[commandchars=\\\{\}]
{\color{incolor}In [{\color{incolor}38}]:} \PY{n}{map\PYZus{}series} \PY{o}{=} \PY{n}{pd}\PY{o}{.}\PY{n}{Series}\PY{p}{(}\PY{n}{mapping}\PY{p}{)}
\end{Verbatim}


    \begin{Verbatim}[commandchars=\\\{\}]
{\color{incolor}In [{\color{incolor}39}]:} \PY{n}{map\PYZus{}series}
\end{Verbatim}


\begin{Verbatim}[commandchars=\\\{\}]
{\color{outcolor}Out[{\color{outcolor}39}]:} a       red
         b       red
         c      blue
         d      blue
         e       red
         f    orange
         dtype: object
\end{Verbatim}
            
    \begin{Verbatim}[commandchars=\\\{\}]
{\color{incolor}In [{\color{incolor}40}]:} \PY{n}{people}\PY{o}{.}\PY{n}{groupby}\PY{p}{(}\PY{n}{map\PYZus{}series}\PY{p}{,} \PY{n}{axis}\PY{o}{=}\PY{l+m+mi}{1}\PY{p}{)}\PY{o}{.}\PY{n}{count}\PY{p}{(}\PY{p}{)}
\end{Verbatim}


\begin{Verbatim}[commandchars=\\\{\}]
{\color{outcolor}Out[{\color{outcolor}40}]:}         blue  red
         Joe        2    3
         Steve      2    3
         Wes        1    2
         Jim        2    3
         Travis     2    3
\end{Verbatim}
            
    Using Python function is a more generic way of defining a group mapping
compared with a dict or Series. \textbf{Any function passed as group key
will be called once per index value, with the returned values being used
as the group names.} More concretely, consider the example DataFrame
from the previous section, which has people's first names as index
values. Suppose we want to group by the lenght of the names; while we
could compute an array of string lenghts, it's simpler to just pass the
\emph{len}() function:

    \begin{Verbatim}[commandchars=\\\{\}]
{\color{incolor}In [{\color{incolor}41}]:} \PY{n}{people}\PY{o}{.}\PY{n}{groupby}\PY{p}{(}\PY{n+nb}{len}\PY{p}{)}\PY{o}{.}\PY{n}{sum}\PY{p}{(}\PY{p}{)}
\end{Verbatim}


\begin{Verbatim}[commandchars=\\\{\}]
{\color{outcolor}Out[{\color{outcolor}41}]:}           a         b         c         d         e
         3 -0.094019 -1.452505  0.263449 -0.033529  0.489879
         5  0.393219  0.398256  0.401803 -1.538065  1.365792
         6 -0.397205 -0.346472  0.656310 -0.438620  0.205482
\end{Verbatim}
            
    \begin{Verbatim}[commandchars=\\\{\}]
{\color{incolor}In [{\color{incolor}42}]:} \PY{n+nb}{list}\PY{p}{(}\PY{n}{people}\PY{o}{.}\PY{n}{groupby}\PY{p}{(}\PY{n+nb}{len}\PY{p}{)}\PY{p}{)}
\end{Verbatim}


\begin{Verbatim}[commandchars=\\\{\}]
{\color{outcolor}Out[{\color{outcolor}42}]:} [(3,             a         b         c         d         e
           Joe -0.840949 -0.712837  0.385332  0.843450 -0.938942
           Wes -0.126382       NaN       NaN -0.426163  0.657845
           Jim  0.873312 -0.739668 -0.121883 -0.450816  0.770976),
          (5,               a         b         c         d         e
           Steve  0.393219  0.398256  0.401803 -1.538065  1.365792),
          (6,                a         b        c        d         e
           Travis -0.397205 -0.346472  0.65631 -0.43862  0.205482)]
\end{Verbatim}
            
    Mixing function with dicts, arrays or Series is not a problem as
evething gets converted to array internally:

    \begin{Verbatim}[commandchars=\\\{\}]
{\color{incolor}In [{\color{incolor}43}]:} \PY{n}{key\PYZus{}list} \PY{o}{=} \PY{p}{[}\PY{l+s+s1}{\PYZsq{}}\PY{l+s+s1}{one}\PY{l+s+s1}{\PYZsq{}}\PY{p}{,} \PY{l+s+s1}{\PYZsq{}}\PY{l+s+s1}{one}\PY{l+s+s1}{\PYZsq{}}\PY{p}{,} \PY{l+s+s1}{\PYZsq{}}\PY{l+s+s1}{one}\PY{l+s+s1}{\PYZsq{}}\PY{p}{,} \PY{l+s+s1}{\PYZsq{}}\PY{l+s+s1}{two}\PY{l+s+s1}{\PYZsq{}}\PY{p}{,} \PY{l+s+s1}{\PYZsq{}}\PY{l+s+s1}{two}\PY{l+s+s1}{\PYZsq{}}\PY{p}{]}
\end{Verbatim}


    \begin{Verbatim}[commandchars=\\\{\}]
{\color{incolor}In [{\color{incolor}44}]:} \PY{n}{people}\PY{o}{.}\PY{n}{groupby}\PY{p}{(}\PY{p}{[}\PY{n+nb}{len}\PY{p}{,} \PY{n}{key\PYZus{}list}\PY{p}{]}\PY{p}{)}\PY{o}{.}\PY{n}{min}\PY{p}{(}\PY{p}{)}
\end{Verbatim}


\begin{Verbatim}[commandchars=\\\{\}]
{\color{outcolor}Out[{\color{outcolor}44}]:}               a         b         c         d         e
         3 one -0.840949 -0.712837  0.385332 -0.426163 -0.938942
           two  0.873312 -0.739668 -0.121883 -0.450816  0.770976
         5 one  0.393219  0.398256  0.401803 -1.538065  1.365792
         6 two -0.397205 -0.346472  0.656310 -0.438620  0.205482
\end{Verbatim}
            
    \begin{Verbatim}[commandchars=\\\{\}]
{\color{incolor}In [{\color{incolor}45}]:} \PY{n+nb}{list}\PY{p}{(}\PY{n}{people}\PY{o}{.}\PY{n}{groupby}\PY{p}{(}\PY{n}{key\PYZus{}list}\PY{p}{)}\PY{p}{)}
\end{Verbatim}


\begin{Verbatim}[commandchars=\\\{\}]
{\color{outcolor}Out[{\color{outcolor}45}]:} [('one',               a         b         c         d         e
           Joe   -0.840949 -0.712837  0.385332  0.843450 -0.938942
           Steve  0.393219  0.398256  0.401803 -1.538065  1.365792
           Wes   -0.126382       NaN       NaN -0.426163  0.657845),
          ('two',                a         b         c         d         e
           Jim     0.873312 -0.739668 -0.121883 -0.450816  0.770976
           Travis -0.397205 -0.346472  0.656310 -0.438620  0.205482)]
\end{Verbatim}
            
    \begin{Verbatim}[commandchars=\\\{\}]
{\color{incolor}In [{\color{incolor}46}]:} \PY{n}{people}
\end{Verbatim}


\begin{Verbatim}[commandchars=\\\{\}]
{\color{outcolor}Out[{\color{outcolor}46}]:}                a         b         c         d         e
         Joe    -0.840949 -0.712837  0.385332  0.843450 -0.938942
         Steve   0.393219  0.398256  0.401803 -1.538065  1.365792
         Wes    -0.126382       NaN       NaN -0.426163  0.657845
         Jim     0.873312 -0.739668 -0.121883 -0.450816  0.770976
         Travis -0.397205 -0.346472  0.656310 -0.438620  0.205482
\end{Verbatim}
            
    \hypertarget{grouping-by-index-levels}{%
\paragraph{Grouping by Index Levels}\label{grouping-by-index-levels}}

    A final convenience for hierarchically indexex datasets is the ability
to aggregate using one of the levels of an axis index. Let's look at an
example:

    \begin{Verbatim}[commandchars=\\\{\}]
{\color{incolor}In [{\color{incolor}47}]:} \PY{n}{columns} \PY{o}{=} \PY{n}{pd}\PY{o}{.}\PY{n}{MultiIndex}\PY{o}{.}\PY{n}{from\PYZus{}arrays}\PY{p}{(}\PY{p}{[}\PY{p}{[}\PY{l+s+s1}{\PYZsq{}}\PY{l+s+s1}{US}\PY{l+s+s1}{\PYZsq{}}\PY{p}{,} \PY{l+s+s1}{\PYZsq{}}\PY{l+s+s1}{US}\PY{l+s+s1}{\PYZsq{}}\PY{p}{,} \PY{l+s+s1}{\PYZsq{}}\PY{l+s+s1}{US}\PY{l+s+s1}{\PYZsq{}}\PY{p}{,} \PY{l+s+s1}{\PYZsq{}}\PY{l+s+s1}{JP}\PY{l+s+s1}{\PYZsq{}}\PY{p}{,} \PY{l+s+s1}{\PYZsq{}}\PY{l+s+s1}{JP}\PY{l+s+s1}{\PYZsq{}}\PY{p}{]}\PY{p}{,}
         \PY{o}{.}\PY{o}{.}\PY{o}{.}\PY{o}{.}\PY{p}{:} \PY{p}{[}\PY{l+m+mi}{1}\PY{p}{,} \PY{l+m+mi}{3}\PY{p}{,} \PY{l+m+mi}{5}\PY{p}{,} \PY{l+m+mi}{1}\PY{p}{,} \PY{l+m+mi}{3}\PY{p}{]}\PY{p}{]}\PY{p}{,}
         \PY{o}{.}\PY{o}{.}\PY{o}{.}\PY{o}{.}\PY{p}{:} \PY{n}{names}\PY{o}{=}\PY{p}{[}\PY{l+s+s1}{\PYZsq{}}\PY{l+s+s1}{cty}\PY{l+s+s1}{\PYZsq{}}\PY{p}{,} \PY{l+s+s1}{\PYZsq{}}\PY{l+s+s1}{tenor}\PY{l+s+s1}{\PYZsq{}}\PY{p}{]}\PY{p}{)}
\end{Verbatim}


    \begin{Verbatim}[commandchars=\\\{\}]
{\color{incolor}In [{\color{incolor}48}]:} \PY{n}{hier\PYZus{}df} \PY{o}{=} \PY{n}{pd}\PY{o}{.}\PY{n}{DataFrame}\PY{p}{(}\PY{n}{np}\PY{o}{.}\PY{n}{random}\PY{o}{.}\PY{n}{randn}\PY{p}{(}\PY{l+m+mi}{4}\PY{p}{,} \PY{l+m+mi}{5}\PY{p}{)}\PY{p}{,} \PY{n}{columns}\PY{o}{=}\PY{n}{columns}\PY{p}{)}
\end{Verbatim}


    \begin{Verbatim}[commandchars=\\\{\}]
{\color{incolor}In [{\color{incolor}49}]:} \PY{n}{hier\PYZus{}df}
\end{Verbatim}


\begin{Verbatim}[commandchars=\\\{\}]
{\color{outcolor}Out[{\color{outcolor}49}]:} cty          US                            JP          
         tenor         1         3         5         1         3
         0     -0.265282  1.101794 -1.100412 -0.334982 -1.931610
         1     -1.071204  0.763616  0.841652  0.437332 -0.122405
         2     -0.983273  0.593081  0.648976 -0.414249  0.327547
         3     -1.779847 -0.507833  0.398431 -1.744622  1.616412
\end{Verbatim}
            
    To group by level, pass the level number or name using the \emph{level}
keyword:

    \begin{Verbatim}[commandchars=\\\{\}]
{\color{incolor}In [{\color{incolor}50}]:} \PY{n}{hier\PYZus{}df}\PY{o}{.}\PY{n}{groupby}\PY{p}{(}\PY{n}{level}\PY{o}{=}\PY{l+s+s1}{\PYZsq{}}\PY{l+s+s1}{cty}\PY{l+s+s1}{\PYZsq{}}\PY{p}{,} \PY{n}{axis}\PY{o}{=}\PY{l+m+mi}{1}\PY{p}{)}\PY{o}{.}\PY{n}{count}\PY{p}{(}\PY{p}{)}
\end{Verbatim}


\begin{Verbatim}[commandchars=\\\{\}]
{\color{outcolor}Out[{\color{outcolor}50}]:} cty  JP  US
         0     2   3
         1     2   3
         2     2   3
         3     2   3
\end{Verbatim}
            
    Data Aggregation

    Aggregations refer to any data transformation that produces scalar
values from arrays. The preceding examples have used several of them,
including \emph{mean}, \emph{std}, \emph{count}, \emph{min}, \emph{max},
and \emph{sum}. We may wonder what's going on when we invoke
\emph{mean}() method on a GroupBy object. Many commmon aggragations,
sucn as those found in table below, have optimized impplementations.
However, we are not limited to only this set of methods.

    Table 10-1 \includegraphics{attachment:image.png}

    We can use aggregations of our own devising and additionally call any
method that is also defined on the grouped object. For example, we might
recall that \emph{quantile}() computes the sample quantiles of a Series
or a DataFrame's column.

While \emph{quantile}() is not explicitly implemented for GroupBy, it is
a Series method and thus available for use. Internally, GroupBy
efficiently slice up the Series, calls \emph{piece.quantile}(0.9) for
each piece, and then assembles those results together into the result
object:

    \begin{Verbatim}[commandchars=\\\{\}]
{\color{incolor}In [{\color{incolor}51}]:} \PY{n}{df}
\end{Verbatim}


\begin{Verbatim}[commandchars=\\\{\}]
{\color{outcolor}Out[{\color{outcolor}51}]:}       data1     data2 key1 key2
         0  0.714162 -0.917054    a  one
         1 -0.251975 -0.335887    a  two
         2  0.056946 -2.165745    b  one
         3  0.718219  2.431233    b  two
         4 -0.388676 -1.043849    a  one
\end{Verbatim}
            
    \begin{Verbatim}[commandchars=\\\{\}]
{\color{incolor}In [{\color{incolor}52}]:} \PY{n}{grouped} \PY{o}{=} \PY{n}{df}\PY{o}{.}\PY{n}{groupby}\PY{p}{(}\PY{l+s+s1}{\PYZsq{}}\PY{l+s+s1}{key1}\PY{l+s+s1}{\PYZsq{}}\PY{p}{)}
\end{Verbatim}


    \begin{Verbatim}[commandchars=\\\{\}]
{\color{incolor}In [{\color{incolor}53}]:} \PY{n+nb}{list}\PY{p}{(}\PY{n}{grouped}\PY{p}{)}
\end{Verbatim}


\begin{Verbatim}[commandchars=\\\{\}]
{\color{outcolor}Out[{\color{outcolor}53}]:} [('a',       data1     data2 key1 key2
           0  0.714162 -0.917054    a  one
           1 -0.251975 -0.335887    a  two
           4 -0.388676 -1.043849    a  one), ('b',       data1     data2 key1 key2
           2  0.056946 -2.165745    b  one
           3  0.718219  2.431233    b  two)]
\end{Verbatim}
            
    \begin{Verbatim}[commandchars=\\\{\}]
{\color{incolor}In [{\color{incolor}54}]:} \PY{n+nb}{list}\PY{p}{(}\PY{n}{grouped}\PY{p}{[}\PY{l+s+s1}{\PYZsq{}}\PY{l+s+s1}{data1}\PY{l+s+s1}{\PYZsq{}}\PY{p}{]}\PY{p}{)}
\end{Verbatim}


\begin{Verbatim}[commandchars=\\\{\}]
{\color{outcolor}Out[{\color{outcolor}54}]:} [('a', 0    0.714162
           1   -0.251975
           4   -0.388676
           Name: data1, dtype: float64), ('b', 2    0.056946
           3    0.718219
           Name: data1, dtype: float64)]
\end{Verbatim}
            
    \begin{Verbatim}[commandchars=\\\{\}]
{\color{incolor}In [{\color{incolor}55}]:} \PY{n}{grouped}\PY{p}{[}\PY{l+s+s1}{\PYZsq{}}\PY{l+s+s1}{data1}\PY{l+s+s1}{\PYZsq{}}\PY{p}{]}\PY{o}{.}\PY{n}{quantile}\PY{p}{(}\PY{p}{)}
\end{Verbatim}


\begin{Verbatim}[commandchars=\\\{\}]
{\color{outcolor}Out[{\color{outcolor}55}]:} key1
         a   -0.251975
         b    0.387583
         Name: data1, dtype: float64
\end{Verbatim}
            
    \begin{Verbatim}[commandchars=\\\{\}]
{\color{incolor}In [{\color{incolor}56}]:} \PY{n}{grouped}\PY{p}{[}\PY{l+s+s1}{\PYZsq{}}\PY{l+s+s1}{data1}\PY{l+s+s1}{\PYZsq{}}\PY{p}{]}\PY{o}{.}\PY{n}{quantile}\PY{p}{(}\PY{l+m+mf}{0.9}\PY{p}{)}
\end{Verbatim}


\begin{Verbatim}[commandchars=\\\{\}]
{\color{outcolor}Out[{\color{outcolor}56}]:} key1
         a    0.520934
         b    0.652092
         Name: data1, dtype: float64
\end{Verbatim}
            
    Let's consider am example to understand quantile:

    \begin{Verbatim}[commandchars=\\\{\}]
{\color{incolor}In [{\color{incolor}57}]:} \PY{n}{ser} \PY{o}{=} \PY{n}{pd}\PY{o}{.}\PY{n}{Series}\PY{p}{(}\PY{p}{[}\PY{l+m+mi}{1}\PY{p}{,} \PY{l+m+mi}{2}\PY{p}{,} \PY{l+m+mi}{3}\PY{p}{,} \PY{l+m+mi}{4}\PY{p}{,} \PY{l+m+mi}{5}\PY{p}{]}\PY{p}{)}
\end{Verbatim}


    \begin{Verbatim}[commandchars=\\\{\}]
{\color{incolor}In [{\color{incolor}58}]:} \PY{n}{ser}
\end{Verbatim}


\begin{Verbatim}[commandchars=\\\{\}]
{\color{outcolor}Out[{\color{outcolor}58}]:} 0    1
         1    2
         2    3
         3    4
         4    5
         dtype: int64
\end{Verbatim}
            
    \begin{Verbatim}[commandchars=\\\{\}]
{\color{incolor}In [{\color{incolor}59}]:} \PY{n}{ser}\PY{o}{.}\PY{n}{quantile}\PY{p}{(}\PY{l+m+mf}{0.5}\PY{p}{)}
\end{Verbatim}


\begin{Verbatim}[commandchars=\\\{\}]
{\color{outcolor}Out[{\color{outcolor}59}]:} 3.0
\end{Verbatim}
            
    \begin{Verbatim}[commandchars=\\\{\}]
{\color{incolor}In [{\color{incolor}60}]:} \PY{n}{ser}\PY{o}{.}\PY{n}{quantile}\PY{p}{(}\PY{l+m+mf}{0.9}\PY{p}{)}
\end{Verbatim}


\begin{Verbatim}[commandchars=\\\{\}]
{\color{outcolor}Out[{\color{outcolor}60}]:} 4.6
\end{Verbatim}
            
    \begin{Verbatim}[commandchars=\\\{\}]
{\color{incolor}In [{\color{incolor}61}]:} \PY{n}{ser}\PY{o}{.}\PY{n}{quantile}\PY{p}{(}\PY{p}{[}\PY{l+m+mf}{0.25}\PY{p}{,} \PY{l+m+mf}{0.5}\PY{p}{,} \PY{l+m+mf}{0.75}\PY{p}{,} \PY{l+m+mf}{1.0}\PY{p}{]}\PY{p}{)}
\end{Verbatim}


\begin{Verbatim}[commandchars=\\\{\}]
{\color{outcolor}Out[{\color{outcolor}61}]:} 0.25    2.0
         0.50    3.0
         0.75    4.0
         1.00    5.0
         dtype: float64
\end{Verbatim}
            
    To use our own aggregation functions, pass any function that aggregates
an array to the \emph{aggregate}() or \emph{agg}() method:

    \begin{Verbatim}[commandchars=\\\{\}]
{\color{incolor}In [{\color{incolor}62}]:} \PY{c+c1}{\PYZsh{} define a function for aggregating}
         \PY{k}{def} \PY{n+nf}{peak\PYZus{}to\PYZus{}peak}\PY{p}{(}\PY{n}{arr}\PY{p}{)}\PY{p}{:}
             \PY{k}{return} \PY{n}{arr}\PY{o}{.}\PY{n}{max}\PY{p}{(}\PY{p}{)} \PY{o}{\PYZhy{}} \PY{n}{arr}\PY{o}{.}\PY{n}{min}\PY{p}{(}\PY{p}{)}
\end{Verbatim}


    \begin{Verbatim}[commandchars=\\\{\}]
{\color{incolor}In [{\color{incolor}63}]:} \PY{n}{grouped}\PY{o}{.}\PY{n}{agg}\PY{p}{(}\PY{n}{peak\PYZus{}to\PYZus{}peak}\PY{p}{)}
\end{Verbatim}


\begin{Verbatim}[commandchars=\\\{\}]
{\color{outcolor}Out[{\color{outcolor}63}]:}          data1     data2
         key1                    
         a     1.102838  0.707962
         b     0.661273  4.596978
\end{Verbatim}
            
    We may notice that some methods like \emph{describe}() are also work,
even though they are not aggregations, strictly speaking:

    \begin{Verbatim}[commandchars=\\\{\}]
{\color{incolor}In [{\color{incolor}64}]:} \PY{n}{grouped}\PY{o}{.}\PY{n}{describe}\PY{p}{(}\PY{p}{)}
\end{Verbatim}


\begin{Verbatim}[commandchars=\\\{\}]
{\color{outcolor}Out[{\color{outcolor}64}]:}      data1                                                              \textbackslash{}
              count      mean       std       min       25\%       50\%       75\%   
         key1                                                                     
         a      3.0  0.024504  0.601160 -0.388676 -0.320326 -0.251975  0.231093   
         b      2.0  0.387583  0.467591  0.056946  0.222265  0.387583  0.552901   
         
                        data2                                                    \textbackslash{}
                    max count      mean       std       min       25\%       50\%   
         key1                                                                     
         a     0.714162   3.0 -0.765597  0.377501 -1.043849 -0.980452 -0.917054   
         b     0.718219   2.0  0.132744  3.250554 -2.165745 -1.016500  0.132744   
         
                                   
                    75\%       max  
         key1                      
         a    -0.626470 -0.335887  
         b     1.281989  2.431233  
\end{Verbatim}
            
    \begin{quote}
Custom aggregation function are generally much slower than the optimized
functions. This is because there is some extra overhead (function calls,
data rearrangement) in constructing the intermediate group data chuncks.
\end{quote}

    Column-Wise and Multiple Function Application

    Let's return to the tipping dataset from the earlier examples. After
loading it with \emph{read\_csv}(), we add a tipping percentage column
\emph{tip\_pct}:

    \begin{Verbatim}[commandchars=\\\{\}]
{\color{incolor}In [{\color{incolor}65}]:} \PY{n}{tips} \PY{o}{=} \PY{n}{pd}\PY{o}{.}\PY{n}{read\PYZus{}csv}\PY{p}{(}\PY{l+s+s2}{\PYZdq{}}\PY{l+s+s2}{./examples/tips.csv}\PY{l+s+s2}{\PYZdq{}}\PY{p}{)}
\end{Verbatim}


    \begin{Verbatim}[commandchars=\\\{\}]
{\color{incolor}In [{\color{incolor}66}]:} \PY{n}{tips}\PY{p}{[}\PY{l+s+s1}{\PYZsq{}}\PY{l+s+s1}{tip\PYZus{}pct}\PY{l+s+s1}{\PYZsq{}}\PY{p}{]} \PY{o}{=} \PY{n}{tips}\PY{p}{[}\PY{l+s+s1}{\PYZsq{}}\PY{l+s+s1}{tip}\PY{l+s+s1}{\PYZsq{}}\PY{p}{]} \PY{o}{/} \PY{n}{tips}\PY{p}{[}\PY{l+s+s1}{\PYZsq{}}\PY{l+s+s1}{total\PYZus{}bill}\PY{l+s+s1}{\PYZsq{}}\PY{p}{]}
\end{Verbatim}


    \begin{Verbatim}[commandchars=\\\{\}]
{\color{incolor}In [{\color{incolor}67}]:} \PY{n}{tips}\PY{p}{[}\PY{p}{:}\PY{l+m+mi}{5}\PY{p}{]}
\end{Verbatim}


\begin{Verbatim}[commandchars=\\\{\}]
{\color{outcolor}Out[{\color{outcolor}67}]:}    total\_bill   tip     sex smoker  day    time  size   tip\_pct
         0       16.99  1.01  Female     No  Sun  Dinner     2  0.059447
         1       10.34  1.66    Male     No  Sun  Dinner     3  0.160542
         2       21.01  3.50    Male     No  Sun  Dinner     3  0.166587
         3       23.68  3.31    Male     No  Sun  Dinner     2  0.139780
         4       24.59  3.61  Female     No  Sun  Dinner     4  0.146808
\end{Verbatim}
            
    As we've already seen, aggregating a Series or all of the columns in a
DataFrame is a matter of using \emph{aggregate}() with the desired
function or calling a method like \emph{mean}() or \emph{std}. However,
we may want to aggregate using different function depending on the
column, or multiple functions at once. Fortunately, this is possible to
do, which we'll illustrate through a number of examples. First, I'll
group the \emph{tips} by day and \emph{smoker}:

    \begin{Verbatim}[commandchars=\\\{\}]
{\color{incolor}In [{\color{incolor}68}]:} \PY{n}{grouped} \PY{o}{=} \PY{n}{tips}\PY{o}{.}\PY{n}{groupby}\PY{p}{(}\PY{p}{[}\PY{l+s+s1}{\PYZsq{}}\PY{l+s+s1}{day}\PY{l+s+s1}{\PYZsq{}}\PY{p}{,} \PY{l+s+s1}{\PYZsq{}}\PY{l+s+s1}{smoker}\PY{l+s+s1}{\PYZsq{}}\PY{p}{]}\PY{p}{)}
\end{Verbatim}


    Note that for descriptive statistics like those in Table 10-1, we can
pass the name of the function as string:

    \begin{Verbatim}[commandchars=\\\{\}]
{\color{incolor}In [{\color{incolor}69}]:} \PY{n}{grouped\PYZus{}pct} \PY{o}{=} \PY{n}{grouped}\PY{p}{[}\PY{l+s+s1}{\PYZsq{}}\PY{l+s+s1}{tip\PYZus{}pct}\PY{l+s+s1}{\PYZsq{}}\PY{p}{]}
\end{Verbatim}


    \begin{Verbatim}[commandchars=\\\{\}]
{\color{incolor}In [{\color{incolor}70}]:} \PY{n}{grouped\PYZus{}pct}\PY{o}{.}\PY{n}{agg}\PY{p}{(}\PY{l+s+s1}{\PYZsq{}}\PY{l+s+s1}{mean}\PY{l+s+s1}{\PYZsq{}}\PY{p}{)}
\end{Verbatim}


\begin{Verbatim}[commandchars=\\\{\}]
{\color{outcolor}Out[{\color{outcolor}70}]:} day   smoker
         Fri   No        0.151650
               Yes       0.174783
         Sat   No        0.158048
               Yes       0.147906
         Sun   No        0.160113
               Yes       0.187250
         Thur  No        0.160298
               Yes       0.163863
         Name: tip\_pct, dtype: float64
\end{Verbatim}
            
    If we pass a list of functions or function names instead, we get back a
DataFrame with the column names taken from the functions:

    \begin{Verbatim}[commandchars=\\\{\}]
{\color{incolor}In [{\color{incolor}71}]:} \PY{n}{grouped\PYZus{}pct}\PY{o}{.}\PY{n}{agg}\PY{p}{(}\PY{p}{[}\PY{l+s+s1}{\PYZsq{}}\PY{l+s+s1}{std}\PY{l+s+s1}{\PYZsq{}}\PY{p}{,} \PY{l+s+s1}{\PYZsq{}}\PY{l+s+s1}{mean}\PY{l+s+s1}{\PYZsq{}}\PY{p}{,} \PY{n}{peak\PYZus{}to\PYZus{}peak}\PY{p}{]}\PY{p}{)}
\end{Verbatim}


\begin{Verbatim}[commandchars=\\\{\}]
{\color{outcolor}Out[{\color{outcolor}71}]:}                   std      mean  peak\_to\_peak
         day  smoker                                  
         Fri  No      0.028123  0.151650      0.067349
              Yes     0.051293  0.174783      0.159925
         Sat  No      0.039767  0.158048      0.235193
              Yes     0.061375  0.147906      0.290095
         Sun  No      0.042347  0.160113      0.193226
              Yes     0.154134  0.187250      0.644685
         Thur No      0.038774  0.160298      0.193350
              Yes     0.039389  0.163863      0.151240
\end{Verbatim}
            
    Here we passed a list of aggregation function to evaluate indenpendently
on the data groups.

We don't need to accept the names that GroupBy gives to the columns;
notably, \emph{lambda} functions have the name ``'', which makes them
hard to identify. Thus, if we pass a list of (name, function) tuples,
the first element of each tuple will be used as the DataFrame column
names, for example:

    \begin{Verbatim}[commandchars=\\\{\}]
{\color{incolor}In [{\color{incolor}72}]:} \PY{n}{grouped\PYZus{}pct}\PY{o}{.}\PY{n}{agg}\PY{p}{(}\PY{p}{[}\PY{p}{(}\PY{l+s+s1}{\PYZsq{}}\PY{l+s+s1}{foo}\PY{l+s+s1}{\PYZsq{}}\PY{p}{,} \PY{l+s+s1}{\PYZsq{}}\PY{l+s+s1}{mean}\PY{l+s+s1}{\PYZsq{}}\PY{p}{)}\PY{p}{,} \PY{p}{(}\PY{l+s+s1}{\PYZsq{}}\PY{l+s+s1}{bar}\PY{l+s+s1}{\PYZsq{}}\PY{p}{,} \PY{n}{np}\PY{o}{.}\PY{n}{std}\PY{p}{)}\PY{p}{]}\PY{p}{)}
\end{Verbatim}


\begin{Verbatim}[commandchars=\\\{\}]
{\color{outcolor}Out[{\color{outcolor}72}]:}                   foo       bar
         day  smoker                    
         Fri  No      0.151650  0.028123
              Yes     0.174783  0.051293
         Sat  No      0.158048  0.039767
              Yes     0.147906  0.061375
         Sun  No      0.160113  0.042347
              Yes     0.187250  0.154134
         Thur No      0.160298  0.038774
              Yes     0.163863  0.039389
\end{Verbatim}
            
    With a DataFrame we have more options, as we can specify a list of
functions to apply to all of columns or defferent function per column.
To start, suppose we want to compute the same three statistics for the
\emph{tip\_pct} and \emph{total\_bill} columns:

    \begin{Verbatim}[commandchars=\\\{\}]
{\color{incolor}In [{\color{incolor}73}]:} \PY{n}{functions} \PY{o}{=} \PY{p}{[}\PY{l+s+s1}{\PYZsq{}}\PY{l+s+s1}{count}\PY{l+s+s1}{\PYZsq{}}\PY{p}{,} \PY{l+s+s1}{\PYZsq{}}\PY{l+s+s1}{mean}\PY{l+s+s1}{\PYZsq{}}\PY{p}{,} \PY{l+s+s1}{\PYZsq{}}\PY{l+s+s1}{max}\PY{l+s+s1}{\PYZsq{}}\PY{p}{]}
\end{Verbatim}


    \begin{Verbatim}[commandchars=\\\{\}]
{\color{incolor}In [{\color{incolor}74}]:} \PY{n}{result} \PY{o}{=} \PY{n}{grouped}\PY{p}{[}\PY{l+s+s1}{\PYZsq{}}\PY{l+s+s1}{tip\PYZus{}pct}\PY{l+s+s1}{\PYZsq{}}\PY{p}{,} \PY{l+s+s1}{\PYZsq{}}\PY{l+s+s1}{total\PYZus{}bill}\PY{l+s+s1}{\PYZsq{}}\PY{p}{]}\PY{o}{.}\PY{n}{agg}\PY{p}{(}\PY{n}{functions}\PY{p}{)}
\end{Verbatim}


    \begin{Verbatim}[commandchars=\\\{\}]
{\color{incolor}In [{\color{incolor}75}]:} \PY{n}{result}
\end{Verbatim}


\begin{Verbatim}[commandchars=\\\{\}]
{\color{outcolor}Out[{\color{outcolor}75}]:}             tip\_pct                     total\_bill                  
                       count      mean       max      count       mean    max
         day  smoker                                                         
         Fri  No           4  0.151650  0.187735          4  18.420000  22.75
              Yes         15  0.174783  0.263480         15  16.813333  40.17
         Sat  No          45  0.158048  0.291990         45  19.661778  48.33
              Yes         42  0.147906  0.325733         42  21.276667  50.81
         Sun  No          57  0.160113  0.252672         57  20.506667  48.17
              Yes         19  0.187250  0.710345         19  24.120000  45.35
         Thur No          45  0.160298  0.266312         45  17.113111  41.19
              Yes         17  0.163863  0.241255         17  19.190588  43.11
\end{Verbatim}
            
    As we can see, the resulting DataFrame has a hierarchical columns, the
same as we would get the aggregating each column separately and using
\emph{concat}() to glue the results together using column names as the
keys argument:

    \begin{Verbatim}[commandchars=\\\{\}]
{\color{incolor}In [{\color{incolor}76}]:} \PY{n}{result}\PY{p}{[}\PY{l+s+s1}{\PYZsq{}}\PY{l+s+s1}{tip\PYZus{}pct}\PY{l+s+s1}{\PYZsq{}}\PY{p}{]}
\end{Verbatim}


\begin{Verbatim}[commandchars=\\\{\}]
{\color{outcolor}Out[{\color{outcolor}76}]:}              count      mean       max
         day  smoker                           
         Fri  No          4  0.151650  0.187735
              Yes        15  0.174783  0.263480
         Sat  No         45  0.158048  0.291990
              Yes        42  0.147906  0.325733
         Sun  No         57  0.160113  0.252672
              Yes        19  0.187250  0.710345
         Thur No         45  0.160298  0.266312
              Yes        17  0.163863  0.241255
\end{Verbatim}
            
    As before, a list of tuples with custom names can be passed:

    \begin{Verbatim}[commandchars=\\\{\}]
{\color{incolor}In [{\color{incolor}77}]:} \PY{n}{ftuples} \PY{o}{=} \PY{p}{[}\PY{p}{(}\PY{l+s+s1}{\PYZsq{}}\PY{l+s+s1}{Durchschnitt}\PY{l+s+s1}{\PYZsq{}}\PY{p}{,} \PY{l+s+s1}{\PYZsq{}}\PY{l+s+s1}{mean}\PY{l+s+s1}{\PYZsq{}}\PY{p}{)}\PY{p}{,} \PY{p}{(}\PY{l+s+s1}{\PYZsq{}}\PY{l+s+s1}{Abweichung}\PY{l+s+s1}{\PYZsq{}}\PY{p}{,} \PY{n}{np}\PY{o}{.}\PY{n}{var}\PY{p}{)}\PY{p}{]}
\end{Verbatim}


    \begin{Verbatim}[commandchars=\\\{\}]
{\color{incolor}In [{\color{incolor}78}]:} \PY{c+c1}{\PYZsh{} 所选择的列应用相同的函数}
         \PY{n}{grouped}\PY{p}{[}\PY{l+s+s1}{\PYZsq{}}\PY{l+s+s1}{tip\PYZus{}pct}\PY{l+s+s1}{\PYZsq{}}\PY{p}{,} \PY{l+s+s1}{\PYZsq{}}\PY{l+s+s1}{total\PYZus{}bill}\PY{l+s+s1}{\PYZsq{}}\PY{p}{]}\PY{o}{.}\PY{n}{agg}\PY{p}{(}\PY{n}{ftuples}\PY{p}{)}
\end{Verbatim}


\begin{Verbatim}[commandchars=\\\{\}]
{\color{outcolor}Out[{\color{outcolor}78}]:}                  tip\_pct              total\_bill            
                     Durchschnitt Abweichung Durchschnitt  Abweichung
         day  smoker                                                 
         Fri  No         0.151650   0.000791    18.420000   25.596333
              Yes        0.174783   0.002631    16.813333   82.562438
         Sat  No         0.158048   0.001581    19.661778   79.908965
              Yes        0.147906   0.003767    21.276667  101.387535
         Sun  No         0.160113   0.001793    20.506667   66.099980
              Yes        0.187250   0.023757    24.120000  109.046044
         Thur No         0.160298   0.001503    17.113111   59.625081
              Yes        0.163863   0.001551    19.190588   69.808518
\end{Verbatim}
            
    Now, suppose we wanted to apply potentially different functions to one
or more of the columns. To do this, pass a dict to \emph{agg}() that
contains a mapping of column names to any of function specifications
listed so far:

    \begin{Verbatim}[commandchars=\\\{\}]
{\color{incolor}In [{\color{incolor}79}]:} \PY{n}{grouped}\PY{o}{.}\PY{n}{agg}\PY{p}{(}\PY{p}{\PYZob{}}\PY{l+s+s1}{\PYZsq{}}\PY{l+s+s1}{tip}\PY{l+s+s1}{\PYZsq{}}\PY{p}{:} \PY{n}{np}\PY{o}{.}\PY{n}{max}\PY{p}{,} \PY{l+s+s1}{\PYZsq{}}\PY{l+s+s1}{size}\PY{l+s+s1}{\PYZsq{}}\PY{p}{:} \PY{l+s+s1}{\PYZsq{}}\PY{l+s+s1}{sum}\PY{l+s+s1}{\PYZsq{}}\PY{p}{\PYZcb{}}\PY{p}{)}
\end{Verbatim}


\begin{Verbatim}[commandchars=\\\{\}]
{\color{outcolor}Out[{\color{outcolor}79}]:}                tip  size
         day  smoker             
         Fri  No       3.50     9
              Yes      4.73    31
         Sat  No       9.00   115
              Yes     10.00   104
         Sun  No       6.00   167
              Yes      6.50    49
         Thur No       6.70   112
              Yes      5.00    40
\end{Verbatim}
            
    \begin{Verbatim}[commandchars=\\\{\}]
{\color{incolor}In [{\color{incolor}80}]:} \PY{c+c1}{\PYZsh{} 不同的列使用不同的函数}
         \PY{n}{grouped}\PY{o}{.}\PY{n}{agg}\PY{p}{(}\PY{p}{\PYZob{}}\PY{l+s+s1}{\PYZsq{}}\PY{l+s+s1}{tip\PYZus{}pct}\PY{l+s+s1}{\PYZsq{}}\PY{p}{:}\PY{p}{[}\PY{l+s+s1}{\PYZsq{}}\PY{l+s+s1}{min}\PY{l+s+s1}{\PYZsq{}}\PY{p}{,} \PY{l+s+s1}{\PYZsq{}}\PY{l+s+s1}{max}\PY{l+s+s1}{\PYZsq{}}\PY{p}{,} \PY{l+s+s1}{\PYZsq{}}\PY{l+s+s1}{mean}\PY{l+s+s1}{\PYZsq{}}\PY{p}{,} \PY{l+s+s1}{\PYZsq{}}\PY{l+s+s1}{std}\PY{l+s+s1}{\PYZsq{}}\PY{p}{]}\PY{p}{,} \PY{l+s+s1}{\PYZsq{}}\PY{l+s+s1}{size}\PY{l+s+s1}{\PYZsq{}}\PY{p}{:} \PY{l+s+s1}{\PYZsq{}}\PY{l+s+s1}{sum}\PY{l+s+s1}{\PYZsq{}}\PY{p}{\PYZcb{}}\PY{p}{)}
\end{Verbatim}


\begin{Verbatim}[commandchars=\\\{\}]
{\color{outcolor}Out[{\color{outcolor}80}]:}               tip\_pct                               size
                           min       max      mean       std  sum
         day  smoker                                             
         Fri  No      0.120385  0.187735  0.151650  0.028123    9
              Yes     0.103555  0.263480  0.174783  0.051293   31
         Sat  No      0.056797  0.291990  0.158048  0.039767  115
              Yes     0.035638  0.325733  0.147906  0.061375  104
         Sun  No      0.059447  0.252672  0.160113  0.042347  167
              Yes     0.065660  0.710345  0.187250  0.154134   49
         Thur No      0.072961  0.266312  0.160298  0.038774  112
              Yes     0.090014  0.241255  0.163863  0.039389   40
\end{Verbatim}
            
    A DataFrame will have hierarchical columns only if multiple functions
are apply to at least one column.

    Returning Aggregated Data Without Row Indexes

    In all of the examples up until now, the aggregated data comes back with
an index potentially hierarchical, composed from the unique group key
combinations. Since this isn't always desirable, we can disable this in
most cases by passing \emph{as\_index=False} to \emph{groupby}():

    \begin{Verbatim}[commandchars=\\\{\}]
{\color{incolor}In [{\color{incolor}81}]:} \PY{n}{tips}\PY{o}{.}\PY{n}{groupby}\PY{p}{(}\PY{p}{[}\PY{l+s+s1}{\PYZsq{}}\PY{l+s+s1}{day}\PY{l+s+s1}{\PYZsq{}}\PY{p}{,} \PY{l+s+s1}{\PYZsq{}}\PY{l+s+s1}{smoker}\PY{l+s+s1}{\PYZsq{}}\PY{p}{]}\PY{p}{,} \PY{n}{as\PYZus{}index}\PY{o}{=}\PY{k+kc}{False}\PY{p}{)}\PY{o}{.}\PY{n}{mean}\PY{p}{(}\PY{p}{)}
\end{Verbatim}


\begin{Verbatim}[commandchars=\\\{\}]
{\color{outcolor}Out[{\color{outcolor}81}]:}     day smoker  total\_bill       tip      size   tip\_pct
         0   Fri     No   18.420000  2.812500  2.250000  0.151650
         1   Fri    Yes   16.813333  2.714000  2.066667  0.174783
         2   Sat     No   19.661778  3.102889  2.555556  0.158048
         3   Sat    Yes   21.276667  2.875476  2.476190  0.147906
         4   Sun     No   20.506667  3.167895  2.929825  0.160113
         5   Sun    Yes   24.120000  3.516842  2.578947  0.187250
         6  Thur     No   17.113111  2.673778  2.488889  0.160298
         7  Thur    Yes   19.190588  3.030000  2.352941  0.163863
\end{Verbatim}
            
    Of course, it's always to obtain the result in this format by calling
\emph{reset\_index}() on the result. Using the \emph{as\_index=False}
method avoids some unneccessary computations.

    \hypertarget{apply-general-split-apply-combine}{%
\subsubsection{10.3 Apply: General
split-apply-combine}\label{apply-general-split-apply-combine}}

    The most general-purpose GroupBy method is \emph{apply}(), which is the
subject of the rest of this section. As illustrated in Fig 10.2,
\emph{apply}() splits the object being manipulated into pieces, invokes
the passed function on each piece, and then attempts to comcanetate the
pieces together.\\
\includegraphics{attachment:image.png}

    \begin{Verbatim}[commandchars=\\\{\}]
{\color{incolor}In [{\color{incolor}82}]:} \PY{n}{tips}\PY{o}{.}\PY{n}{sort\PYZus{}values}\PY{p}{(}\PY{n}{by}\PY{o}{=}\PY{l+s+s1}{\PYZsq{}}\PY{l+s+s1}{tip\PYZus{}pct}\PY{l+s+s1}{\PYZsq{}}\PY{p}{)}\PY{p}{[}\PY{o}{\PYZhy{}}\PY{l+m+mi}{5}\PY{p}{:}\PY{p}{]}
\end{Verbatim}


\begin{Verbatim}[commandchars=\\\{\}]
{\color{outcolor}Out[{\color{outcolor}82}]:}      total\_bill   tip     sex smoker  day    time  size   tip\_pct
         183       23.17  6.50    Male    Yes  Sun  Dinner     4  0.280535
         232       11.61  3.39    Male     No  Sat  Dinner     2  0.291990
         67         3.07  1.00  Female    Yes  Sat  Dinner     1  0.325733
         178        9.60  4.00  Female    Yes  Sun  Dinner     2  0.416667
         172        7.25  5.15    Male    Yes  Sun  Dinner     2  0.710345
\end{Verbatim}
            
    \begin{Verbatim}[commandchars=\\\{\}]
{\color{incolor}In [{\color{incolor}84}]:} \PY{k}{def} \PY{n+nf}{top}\PY{p}{(}\PY{n}{df}\PY{p}{,} \PY{n}{n}\PY{o}{=}\PY{l+m+mi}{5}\PY{p}{,} \PY{n}{column}\PY{o}{=}\PY{l+s+s1}{\PYZsq{}}\PY{l+s+s1}{tip\PYZus{}pct}\PY{l+s+s1}{\PYZsq{}}\PY{p}{)}\PY{p}{:}
             \PY{k}{return} \PY{n}{df}\PY{o}{.}\PY{n}{sort\PYZus{}values}\PY{p}{(}\PY{n}{by}\PY{o}{=}\PY{n}{column}\PY{p}{)}\PY{p}{[}\PY{o}{\PYZhy{}}\PY{n}{n}\PY{p}{:}\PY{p}{]}
\end{Verbatim}


    \begin{Verbatim}[commandchars=\\\{\}]
{\color{incolor}In [{\color{incolor}85}]:} \PY{n}{top}\PY{p}{(}\PY{n}{tips}\PY{p}{,} \PY{n}{n}\PY{o}{=}\PY{l+m+mi}{8}\PY{p}{)}
\end{Verbatim}


\begin{Verbatim}[commandchars=\\\{\}]
{\color{outcolor}Out[{\color{outcolor}85}]:}      total\_bill   tip     sex smoker   day    time  size   tip\_pct
         93        16.32  4.30  Female    Yes   Fri  Dinner     2  0.263480
         149        7.51  2.00    Male     No  Thur   Lunch     2  0.266312
         109       14.31  4.00  Female    Yes   Sat  Dinner     2  0.279525
         183       23.17  6.50    Male    Yes   Sun  Dinner     4  0.280535
         232       11.61  3.39    Male     No   Sat  Dinner     2  0.291990
         67         3.07  1.00  Female    Yes   Sat  Dinner     1  0.325733
         178        9.60  4.00  Female    Yes   Sun  Dinner     2  0.416667
         172        7.25  5.15    Male    Yes   Sun  Dinner     2  0.710345
\end{Verbatim}
            
    \begin{Verbatim}[commandchars=\\\{\}]
{\color{incolor}In [{\color{incolor}86}]:} \PY{n}{tips}\PY{o}{.}\PY{n}{groupby}\PY{p}{(}\PY{l+s+s1}{\PYZsq{}}\PY{l+s+s1}{smoker}\PY{l+s+s1}{\PYZsq{}}\PY{p}{)}\PY{o}{.}\PY{n}{apply}\PY{p}{(}\PY{n}{top}\PY{p}{)}
\end{Verbatim}


\begin{Verbatim}[commandchars=\\\{\}]
{\color{outcolor}Out[{\color{outcolor}86}]:}             total\_bill   tip     sex smoker   day    time  size   tip\_pct
         smoker                                                                   
         No     88        24.71  5.85    Male     No  Thur   Lunch     2  0.236746
                185       20.69  5.00    Male     No   Sun  Dinner     5  0.241663
                51        10.29  2.60  Female     No   Sun  Dinner     2  0.252672
                149        7.51  2.00    Male     No  Thur   Lunch     2  0.266312
                232       11.61  3.39    Male     No   Sat  Dinner     2  0.291990
         Yes    109       14.31  4.00  Female    Yes   Sat  Dinner     2  0.279525
                183       23.17  6.50    Male    Yes   Sun  Dinner     4  0.280535
                67         3.07  1.00  Female    Yes   Sat  Dinner     1  0.325733
                178        9.60  4.00  Female    Yes   Sun  Dinner     2  0.416667
                172        7.25  5.15    Male    Yes   Sun  Dinner     2  0.710345
\end{Verbatim}
            
    What happened here? The \emph{top}() function is called on each row
group from the DataFrame, and then the results are glued together by
\emph{pandas.concat}(), labeling the pieces with the group names. The
results therefore have a hierarchical index whose inner level contains
index values from the original DataFrame.

If we pass a function to \emph{apply}() that takes other arguments or
keywords, we can pass these after the function:

    \begin{Verbatim}[commandchars=\\\{\}]
{\color{incolor}In [{\color{incolor}87}]:} \PY{n}{tips}\PY{o}{.}\PY{n}{groupby}\PY{p}{(}\PY{p}{[}\PY{l+s+s1}{\PYZsq{}}\PY{l+s+s1}{smoker}\PY{l+s+s1}{\PYZsq{}}\PY{p}{,} \PY{l+s+s1}{\PYZsq{}}\PY{l+s+s1}{day}\PY{l+s+s1}{\PYZsq{}}\PY{p}{]}\PY{p}{)}\PY{o}{.}\PY{n}{apply}\PY{p}{(}\PY{n}{top}\PY{p}{,} \PY{n}{n}\PY{o}{=}\PY{l+m+mi}{2}\PY{p}{,} \PY{n}{column}\PY{o}{=}\PY{l+s+s1}{\PYZsq{}}\PY{l+s+s1}{total\PYZus{}bill}\PY{l+s+s1}{\PYZsq{}}\PY{p}{)}
\end{Verbatim}


\begin{Verbatim}[commandchars=\\\{\}]
{\color{outcolor}Out[{\color{outcolor}87}]:}                  total\_bill    tip     sex smoker   day    time  size  \textbackslash{}
         smoker day                                                              
         No     Fri  91        22.49   3.50    Male     No   Fri  Dinner     2   
                     94        22.75   3.25  Female     No   Fri  Dinner     2   
                Sat  59        48.27   6.73    Male     No   Sat  Dinner     4   
                     212       48.33   9.00    Male     No   Sat  Dinner     4   
                Sun  112       38.07   4.00    Male     No   Sun  Dinner     3   
                     156       48.17   5.00    Male     No   Sun  Dinner     6   
                Thur 85        34.83   5.17  Female     No  Thur   Lunch     4   
                     142       41.19   5.00    Male     No  Thur   Lunch     5   
         Yes    Fri  90        28.97   3.00    Male    Yes   Fri  Dinner     2   
                     95        40.17   4.73    Male    Yes   Fri  Dinner     4   
                Sat  102       44.30   2.50  Female    Yes   Sat  Dinner     3   
                     170       50.81  10.00    Male    Yes   Sat  Dinner     3   
                Sun  184       40.55   3.00    Male    Yes   Sun  Dinner     2   
                     182       45.35   3.50    Male    Yes   Sun  Dinner     3   
                Thur 83        32.68   5.00    Male    Yes  Thur   Lunch     2   
                     197       43.11   5.00  Female    Yes  Thur   Lunch     4   
         
                           tip\_pct  
         smoker day                 
         No     Fri  91   0.155625  
                     94   0.142857  
                Sat  59   0.139424  
                     212  0.186220  
                Sun  112  0.105070  
                     156  0.103799  
                Thur 85   0.148435  
                     142  0.121389  
         Yes    Fri  90   0.103555  
                     95   0.117750  
                Sat  102  0.056433  
                     170  0.196812  
                Sun  184  0.073983  
                     182  0.077178  
                Thur 83   0.152999  
                     197  0.115982  
\end{Verbatim}
            
    \begin{quote}
Beyond these basic usage mechanics, getting the most out of
\emph{apply}() may require some creativity. What occurs inside the
function passed is up to you; it only needs to return a pandas object or
a scalar value. The rest of this chapter will mainly consist of examples
showing you how solve various problems using \emph{groupby}().
\end{quote}

    \begin{Verbatim}[commandchars=\\\{\}]
{\color{incolor}In [{\color{incolor}89}]:} \PY{n}{result} \PY{o}{=} \PY{n}{tips}\PY{o}{.}\PY{n}{groupby}\PY{p}{(}\PY{l+s+s1}{\PYZsq{}}\PY{l+s+s1}{smoker}\PY{l+s+s1}{\PYZsq{}}\PY{p}{)}\PY{p}{[}\PY{l+s+s1}{\PYZsq{}}\PY{l+s+s1}{tip\PYZus{}pct}\PY{l+s+s1}{\PYZsq{}}\PY{p}{]}\PY{o}{.}\PY{n}{describe}\PY{p}{(}\PY{p}{)}
\end{Verbatim}


    \begin{Verbatim}[commandchars=\\\{\}]
{\color{incolor}In [{\color{incolor}90}]:} \PY{n}{result}
\end{Verbatim}


\begin{Verbatim}[commandchars=\\\{\}]
{\color{outcolor}Out[{\color{outcolor}90}]:}         count      mean       std       min       25\%       50\%       75\%  \textbackslash{}
         smoker                                                                      
         No      151.0  0.159328  0.039910  0.056797  0.136906  0.155625  0.185014   
         Yes      93.0  0.163196  0.085119  0.035638  0.106771  0.153846  0.195059   
         
                      max  
         smoker            
         No      0.291990  
         Yes     0.710345  
\end{Verbatim}
            
    \begin{Verbatim}[commandchars=\\\{\}]
{\color{incolor}In [{\color{incolor}91}]:} \PY{n}{result}\PY{o}{.}\PY{n}{unstack}\PY{p}{(}\PY{p}{)}
\end{Verbatim}


\begin{Verbatim}[commandchars=\\\{\}]
{\color{outcolor}Out[{\color{outcolor}91}]:}        smoker
         count  No        151.000000
                Yes        93.000000
         mean   No          0.159328
                Yes         0.163196
         std    No          0.039910
                Yes         0.085119
         min    No          0.056797
                Yes         0.035638
         25\%    No          0.136906
                Yes         0.106771
         50\%    No          0.155625
                Yes         0.153846
         75\%    No          0.185014
                Yes         0.195059
         max    No          0.291990
                Yes         0.710345
         dtype: float64
\end{Verbatim}
            
    Inside a GroupBy, when you invoke a method like \emph{describe}(), it is
usually just a shortcut for:

\begin{Shaded}
\begin{Highlighting}[]
\NormalTok{  f }\OperatorTok{=} \KeywordTok{lambda}\NormalTok{ x: x.describe()}
\NormalTok{  grouped.}\BuiltInTok{apply}\NormalTok{(f)}
\end{Highlighting}
\end{Shaded}

    \hypertarget{suppressing-the-group-keys}{%
\subsubsection{Suppressing the Group
Keys}\label{suppressing-the-group-keys}}

    In the preceding examples, we can see that the resulting object has a
herarchical index formed from the group keys along with the indexes of
each piece of the original object. We can disable this by passing
\emph{group\_keys=False} to \emph{groupby}():

    \begin{Verbatim}[commandchars=\\\{\}]
{\color{incolor}In [{\color{incolor}92}]:} \PY{n}{tips}\PY{o}{.}\PY{n}{groupby}\PY{p}{(}\PY{l+s+s1}{\PYZsq{}}\PY{l+s+s1}{smoker}\PY{l+s+s1}{\PYZsq{}}\PY{p}{,} \PY{n}{group\PYZus{}keys}\PY{o}{=}\PY{k+kc}{False}\PY{p}{)}\PY{o}{.}\PY{n}{apply}\PY{p}{(}\PY{n}{top}\PY{p}{)}
\end{Verbatim}


\begin{Verbatim}[commandchars=\\\{\}]
{\color{outcolor}Out[{\color{outcolor}92}]:}      total\_bill   tip     sex smoker   day    time  size   tip\_pct
         88        24.71  5.85    Male     No  Thur   Lunch     2  0.236746
         185       20.69  5.00    Male     No   Sun  Dinner     5  0.241663
         51        10.29  2.60  Female     No   Sun  Dinner     2  0.252672
         149        7.51  2.00    Male     No  Thur   Lunch     2  0.266312
         232       11.61  3.39    Male     No   Sat  Dinner     2  0.291990
         109       14.31  4.00  Female    Yes   Sat  Dinner     2  0.279525
         183       23.17  6.50    Male    Yes   Sun  Dinner     4  0.280535
         67         3.07  1.00  Female    Yes   Sat  Dinner     1  0.325733
         178        9.60  4.00  Female    Yes   Sun  Dinner     2  0.416667
         172        7.25  5.15    Male    Yes   Sun  Dinner     2  0.710345
\end{Verbatim}
            
    \begin{Verbatim}[commandchars=\\\{\}]
{\color{incolor}In [{\color{incolor}94}]:} \PY{n}{tips}\PY{o}{.}\PY{n}{groupby}\PY{p}{(}\PY{l+s+s1}{\PYZsq{}}\PY{l+s+s1}{smoker}\PY{l+s+s1}{\PYZsq{}}\PY{p}{,} \PY{n}{group\PYZus{}keys}\PY{o}{=}\PY{k+kc}{True}\PY{p}{)}\PY{o}{.}\PY{n}{apply}\PY{p}{(}\PY{n}{top}\PY{p}{)}
\end{Verbatim}


\begin{Verbatim}[commandchars=\\\{\}]
{\color{outcolor}Out[{\color{outcolor}94}]:}             total\_bill   tip     sex smoker   day    time  size   tip\_pct
         smoker                                                                   
         No     88        24.71  5.85    Male     No  Thur   Lunch     2  0.236746
                185       20.69  5.00    Male     No   Sun  Dinner     5  0.241663
                51        10.29  2.60  Female     No   Sun  Dinner     2  0.252672
                149        7.51  2.00    Male     No  Thur   Lunch     2  0.266312
                232       11.61  3.39    Male     No   Sat  Dinner     2  0.291990
         Yes    109       14.31  4.00  Female    Yes   Sat  Dinner     2  0.279525
                183       23.17  6.50    Male    Yes   Sun  Dinner     4  0.280535
                67         3.07  1.00  Female    Yes   Sat  Dinner     1  0.325733
                178        9.60  4.00  Female    Yes   Sun  Dinner     2  0.416667
                172        7.25  5.15    Male    Yes   Sun  Dinner     2  0.710345
\end{Verbatim}
            
    \hypertarget{quantile-and-bucket}{%
\subsubsection{Quantile and Bucket}\label{quantile-and-bucket}}

    As we may recall that in Chapter 8, pandas has some tools, in particular
\emph{cut}() nad \emph{qcut}(), for slicing data up into buckets with
the bins of our choosing or by samples quantiles. Combining these
functions with \emph{groupby}() makes it convenient to perform bucket or
quantile analysis on a dataset. Consider a simple random dataset and an
equal-length bucket categorization using \emph{cut}():

    \begin{Verbatim}[commandchars=\\\{\}]
{\color{incolor}In [{\color{incolor}95}]:} \PY{n}{frame} \PY{o}{=} \PY{n}{pd}\PY{o}{.}\PY{n}{DataFrame}\PY{p}{(}\PY{p}{\PYZob{}}\PY{l+s+s1}{\PYZsq{}}\PY{l+s+s1}{data1}\PY{l+s+s1}{\PYZsq{}}\PY{p}{:} \PY{n}{np}\PY{o}{.}\PY{n}{random}\PY{o}{.}\PY{n}{randn}\PY{p}{(}\PY{l+m+mi}{1000}\PY{p}{)}\PY{p}{,}
                              \PY{l+s+s1}{\PYZsq{}}\PY{l+s+s1}{data2}\PY{l+s+s1}{\PYZsq{}}\PY{p}{:} \PY{n}{np}\PY{o}{.}\PY{n}{random}\PY{o}{.}\PY{n}{randn}\PY{p}{(}\PY{l+m+mi}{1000}\PY{p}{)}\PY{p}{\PYZcb{}}\PY{p}{)}
\end{Verbatim}


    \begin{Verbatim}[commandchars=\\\{\}]
{\color{incolor}In [{\color{incolor}97}]:} \PY{n}{quantiles} \PY{o}{=} \PY{n}{pd}\PY{o}{.}\PY{n}{cut}\PY{p}{(}\PY{n}{frame}\PY{o}{.}\PY{n}{data1}\PY{p}{,} \PY{l+m+mi}{4}\PY{p}{)}
\end{Verbatim}


    \begin{Verbatim}[commandchars=\\\{\}]
{\color{incolor}In [{\color{incolor}98}]:} \PY{n}{quantiles}\PY{p}{[}\PY{p}{:}\PY{l+m+mi}{10}\PY{p}{]}
\end{Verbatim}


\begin{Verbatim}[commandchars=\\\{\}]
{\color{outcolor}Out[{\color{outcolor}98}]:} 0     (-0.517, 1.194]
         1    (-2.228, -0.517]
         2    (-2.228, -0.517]
         3     (-0.517, 1.194]
         4    (-2.228, -0.517]
         5     (-0.517, 1.194]
         6     (-0.517, 1.194]
         7    (-2.228, -0.517]
         8     (-0.517, 1.194]
         9      (1.194, 2.905]
         Name: data1, dtype: category
         Categories (4, interval[float64]): [(-3.945, -2.228] < (-2.228, -0.517] < (-0.517, 1.194] < (1.194, 2.905]]
\end{Verbatim}
            
    The Categorical object returned by \emph{cut}() can be passed to
\emph{groupby}(). So we could compute a set of statistics for the
\emph{data2} column like so:

    \begin{Verbatim}[commandchars=\\\{\}]
{\color{incolor}In [{\color{incolor}99}]:} \PY{k}{def} \PY{n+nf}{get\PYZus{}stats}\PY{p}{(}\PY{n}{group}\PY{p}{)}\PY{p}{:}
             \PY{k}{return} \PY{p}{\PYZob{}}\PY{l+s+s1}{\PYZsq{}}\PY{l+s+s1}{min}\PY{l+s+s1}{\PYZsq{}}\PY{p}{:}\PY{n}{group}\PY{o}{.}\PY{n}{min}\PY{p}{(}\PY{p}{)}\PY{p}{,} \PY{l+s+s1}{\PYZsq{}}\PY{l+s+s1}{max}\PY{l+s+s1}{\PYZsq{}}\PY{p}{:} \PY{n}{group}\PY{o}{.}\PY{n}{max}\PY{p}{(}\PY{p}{)}\PY{p}{,}
                    \PY{l+s+s1}{\PYZsq{}}\PY{l+s+s1}{count}\PY{l+s+s1}{\PYZsq{}}\PY{p}{:} \PY{n}{group}\PY{o}{.}\PY{n}{count}\PY{p}{(}\PY{p}{)}\PY{p}{,} \PY{l+s+s1}{\PYZsq{}}\PY{l+s+s1}{mean}\PY{l+s+s1}{\PYZsq{}}\PY{p}{:} \PY{n}{group}\PY{o}{.}\PY{n}{mean}\PY{p}{(}\PY{p}{)}\PY{p}{\PYZcb{}}
\end{Verbatim}


    \begin{Verbatim}[commandchars=\\\{\}]
{\color{incolor}In [{\color{incolor}100}]:} \PY{n}{grouped} \PY{o}{=} \PY{n}{frame}\PY{o}{.}\PY{n}{data2}\PY{o}{.}\PY{n}{groupby}\PY{p}{(}\PY{n}{quantiles}\PY{p}{)}
\end{Verbatim}


    \begin{Verbatim}[commandchars=\\\{\}]
{\color{incolor}In [{\color{incolor}101}]:} \PY{n}{grouped}\PY{o}{.}\PY{n}{apply}\PY{p}{(}\PY{n}{get\PYZus{}stats}\PY{p}{)}\PY{o}{.}\PY{n}{unstack}\PY{p}{(}\PY{p}{)}
\end{Verbatim}


\begin{Verbatim}[commandchars=\\\{\}]
{\color{outcolor}Out[{\color{outcolor}101}]:}                   count       max      mean       min
          data1                                                
          (-3.945, -2.228]   16.0  0.916305 -0.374944 -1.224791
          (-2.228, -0.517]  280.0  2.630693 -0.079320 -2.935569
          (-0.517, 1.194]   579.0  3.293577 -0.005313 -3.206498
          (1.194, 2.905]    125.0  3.293069  0.127818 -2.134369
\end{Verbatim}
            
    These are equal-lenght bucketl to compute equal-size buckets based on
sample quantiles, use \emph{qcut}(). I'll pass \emph{labels=False} to
just get quantile numbers:

    \begin{Verbatim}[commandchars=\\\{\}]
{\color{incolor}In [{\color{incolor}108}]:} \PY{n}{grouping} \PY{o}{=} \PY{n}{pd}\PY{o}{.}\PY{n}{qcut}\PY{p}{(}\PY{n}{frame}\PY{o}{.}\PY{n}{data1}\PY{p}{,} \PY{l+m+mi}{10}\PY{p}{,} \PY{n}{labels}\PY{o}{=}\PY{k+kc}{False}\PY{p}{)}
\end{Verbatim}


    \begin{Verbatim}[commandchars=\\\{\}]
{\color{incolor}In [{\color{incolor}105}]:} \PY{n}{grouped} \PY{o}{=} \PY{n}{frame}\PY{o}{.}\PY{n}{data2}\PY{o}{.}\PY{n}{groupby}\PY{p}{(}\PY{n}{grouping}\PY{p}{)}
\end{Verbatim}


    \begin{Verbatim}[commandchars=\\\{\}]
{\color{incolor}In [{\color{incolor}106}]:} \PY{n}{grouped}\PY{o}{.}\PY{n}{apply}\PY{p}{(}\PY{n}{get\PYZus{}stats}\PY{p}{)}\PY{o}{.}\PY{n}{unstack}\PY{p}{(}\PY{p}{)}
\end{Verbatim}


\begin{Verbatim}[commandchars=\\\{\}]
{\color{outcolor}Out[{\color{outcolor}106}]:}        count       max      mean       min
          data1                                     
          0      100.0  2.226598 -0.233634 -2.813975
          1      100.0  2.630693 -0.005260 -2.935569
          2      100.0  2.051488 -0.040795 -1.949513
          3      100.0  2.405505 -0.121140 -1.918317
          4      100.0  3.045885  0.111722 -3.019736
          5      100.0  3.293577  0.074979 -2.138791
          6      100.0  2.432510  0.001161 -1.851976
          7      100.0  2.478080 -0.030876 -2.407779
          8      100.0  3.293069  0.031796 -3.206498
          9      100.0  2.064010  0.058969 -2.134369
\end{Verbatim}
            
    \hypertarget{example-filling-missing-values-with-group-specific-values}{%
\paragraph{Example: Filling Missing Values with Group-Specific
Values}\label{example-filling-missing-values-with-group-specific-values}}

    When cleaning up missing data, in some cases we will replace data
observations using \emph{dropna}(), but in others we may want to impute
(fill in) the null values using a fixed value or some value derived from
the data. \emph{fillna}() is the right tool to use; for example, here I
fill in NA values with the mean:

    \begin{Verbatim}[commandchars=\\\{\}]
{\color{incolor}In [{\color{incolor}110}]:} \PY{n}{s} \PY{o}{=} \PY{n}{pd}\PY{o}{.}\PY{n}{Series}\PY{p}{(}\PY{n}{np}\PY{o}{.}\PY{n}{random}\PY{o}{.}\PY{n}{randn}\PY{p}{(}\PY{l+m+mi}{6}\PY{p}{)}\PY{p}{)}
\end{Verbatim}


    \begin{Verbatim}[commandchars=\\\{\}]
{\color{incolor}In [{\color{incolor}111}]:} \PY{n}{s}\PY{p}{[}\PY{p}{:}\PY{p}{:}\PY{l+m+mi}{2}\PY{p}{]} \PY{o}{=} \PY{n}{np}\PY{o}{.}\PY{n}{nan}
\end{Verbatim}


    \begin{Verbatim}[commandchars=\\\{\}]
{\color{incolor}In [{\color{incolor}112}]:} \PY{n}{s}
\end{Verbatim}


\begin{Verbatim}[commandchars=\\\{\}]
{\color{outcolor}Out[{\color{outcolor}112}]:} 0         NaN
          1   -0.538433
          2         NaN
          3   -0.453895
          4         NaN
          5    0.398053
          dtype: float64
\end{Verbatim}
            
    \begin{Verbatim}[commandchars=\\\{\}]
{\color{incolor}In [{\color{incolor}113}]:} \PY{n}{s}\PY{o}{.}\PY{n}{fillna}\PY{p}{(}\PY{n}{s}\PY{o}{.}\PY{n}{mean}\PY{p}{(}\PY{p}{)}\PY{p}{)}
\end{Verbatim}


\begin{Verbatim}[commandchars=\\\{\}]
{\color{outcolor}Out[{\color{outcolor}113}]:} 0   -0.198092
          1   -0.538433
          2   -0.198092
          3   -0.453895
          4   -0.198092
          5    0.398053
          dtype: float64
\end{Verbatim}
            
    Suppose we may want to fill value to vary by group. One way to do this
is to group the data and use \emph{apply}() with a function that calls
on each data chunk. Here is some sample data on US states divided into
eastern and western regions:

    \begin{Verbatim}[commandchars=\\\{\}]
{\color{incolor}In [{\color{incolor}119}]:} \PY{n}{states} \PY{o}{=} \PY{p}{[}\PY{l+s+s1}{\PYZsq{}}\PY{l+s+s1}{Ohio}\PY{l+s+s1}{\PYZsq{}}\PY{p}{,} \PY{l+s+s1}{\PYZsq{}}\PY{l+s+s1}{New York}\PY{l+s+s1}{\PYZsq{}}\PY{p}{,} \PY{l+s+s1}{\PYZsq{}}\PY{l+s+s1}{Vermont}\PY{l+s+s1}{\PYZsq{}}\PY{p}{,} \PY{l+s+s1}{\PYZsq{}}\PY{l+s+s1}{Florida}\PY{l+s+s1}{\PYZsq{}}\PY{p}{,} \PY{l+s+s1}{\PYZsq{}}\PY{l+s+s1}{Oregon}\PY{l+s+s1}{\PYZsq{}}\PY{p}{,} \PY{l+s+s1}{\PYZsq{}}\PY{l+s+s1}{Nevada}\PY{l+s+s1}{\PYZsq{}}\PY{p}{,} \PY{l+s+s1}{\PYZsq{}}\PY{l+s+s1}{California}\PY{l+s+s1}{\PYZsq{}}\PY{p}{,} \PY{l+s+s1}{\PYZsq{}}\PY{l+s+s1}{Idaho}\PY{l+s+s1}{\PYZsq{}}\PY{p}{]}
\end{Verbatim}


    \begin{Verbatim}[commandchars=\\\{\}]
{\color{incolor}In [{\color{incolor}120}]:} \PY{n}{group\PYZus{}key} \PY{o}{=} \PY{p}{[}\PY{l+s+s1}{\PYZsq{}}\PY{l+s+s1}{East}\PY{l+s+s1}{\PYZsq{}}\PY{p}{]}\PY{o}{*}\PY{l+m+mi}{4} \PY{o}{+} \PY{p}{[}\PY{l+s+s1}{\PYZsq{}}\PY{l+s+s1}{West}\PY{l+s+s1}{\PYZsq{}}\PY{p}{]}\PY{o}{*}\PY{l+m+mi}{4}
\end{Verbatim}


    \begin{Verbatim}[commandchars=\\\{\}]
{\color{incolor}In [{\color{incolor}121}]:} \PY{n}{data} \PY{o}{=} \PY{n}{pd}\PY{o}{.}\PY{n}{Series}\PY{p}{(}\PY{n}{np}\PY{o}{.}\PY{n}{random}\PY{o}{.}\PY{n}{randn}\PY{p}{(}\PY{l+m+mi}{8}\PY{p}{)}\PY{p}{,} \PY{n}{index}\PY{o}{=}\PY{n}{states}\PY{p}{)}
\end{Verbatim}


    \begin{Verbatim}[commandchars=\\\{\}]
{\color{incolor}In [{\color{incolor}122}]:} \PY{n}{data}
\end{Verbatim}


\begin{Verbatim}[commandchars=\\\{\}]
{\color{outcolor}Out[{\color{outcolor}122}]:} Ohio          1.895558
          New York      1.324299
          Vermont       3.328344
          Florida      -1.442373
          Oregon        1.122452
          Nevada       -0.958831
          California    0.423392
          Idaho         0.250130
          dtype: float64
\end{Verbatim}
            
    \begin{quote}
Note that the syntax {[}`East'{]}*4 produces a list containing four
copies of the element in {[}`East'{]}. Adding list together concatenates
them.
\end{quote}

Let's set some values in the data to be missing:

    \begin{Verbatim}[commandchars=\\\{\}]
{\color{incolor}In [{\color{incolor}123}]:} \PY{n}{data}\PY{p}{[}\PY{p}{[}\PY{l+s+s1}{\PYZsq{}}\PY{l+s+s1}{Vermont}\PY{l+s+s1}{\PYZsq{}}\PY{p}{,} \PY{l+s+s1}{\PYZsq{}}\PY{l+s+s1}{Nevada}\PY{l+s+s1}{\PYZsq{}}\PY{p}{,} \PY{l+s+s1}{\PYZsq{}}\PY{l+s+s1}{Idaho}\PY{l+s+s1}{\PYZsq{}}\PY{p}{]}\PY{p}{]} \PY{o}{=} \PY{n}{np}\PY{o}{.}\PY{n}{nan}
\end{Verbatim}


    \begin{Verbatim}[commandchars=\\\{\}]
{\color{incolor}In [{\color{incolor}124}]:} \PY{n}{data}
\end{Verbatim}


\begin{Verbatim}[commandchars=\\\{\}]
{\color{outcolor}Out[{\color{outcolor}124}]:} Ohio          1.895558
          New York      1.324299
          Vermont            NaN
          Florida      -1.442373
          Oregon        1.122452
          Nevada             NaN
          California    0.423392
          Idaho              NaN
          dtype: float64
\end{Verbatim}
            
    \begin{Verbatim}[commandchars=\\\{\}]
{\color{incolor}In [{\color{incolor}125}]:} \PY{n}{data}\PY{o}{.}\PY{n}{groupby}\PY{p}{(}\PY{n}{group\PYZus{}key}\PY{p}{)}\PY{o}{.}\PY{n}{mean}\PY{p}{(}\PY{p}{)}
\end{Verbatim}


\begin{Verbatim}[commandchars=\\\{\}]
{\color{outcolor}Out[{\color{outcolor}125}]:} East    0.592495
          West    0.772922
          dtype: float64
\end{Verbatim}
            
    We can fill the NA using the group means likes so:

    \begin{Verbatim}[commandchars=\\\{\}]
{\color{incolor}In [{\color{incolor}126}]:} \PY{n}{fill\PYZus{}mean} \PY{o}{=} \PY{k}{lambda} \PY{n}{g}\PY{p}{:} \PY{n}{g}\PY{o}{.}\PY{n}{fillna}\PY{p}{(}\PY{n}{g}\PY{o}{.}\PY{n}{mean}\PY{p}{(}\PY{p}{)}\PY{p}{)}
\end{Verbatim}


    \begin{Verbatim}[commandchars=\\\{\}]
{\color{incolor}In [{\color{incolor}127}]:} \PY{n}{data}\PY{o}{.}\PY{n}{groupby}\PY{p}{(}\PY{n}{group\PYZus{}key}\PY{p}{)}\PY{o}{.}\PY{n}{apply}\PY{p}{(}\PY{n}{fill\PYZus{}mean}\PY{p}{)}
\end{Verbatim}


\begin{Verbatim}[commandchars=\\\{\}]
{\color{outcolor}Out[{\color{outcolor}127}]:} Ohio          1.895558
          New York      1.324299
          Vermont       0.592495
          Florida      -1.442373
          Oregon        1.122452
          Nevada        0.772922
          California    0.423392
          Idaho         0.772922
          dtype: float64
\end{Verbatim}
            
    In another case, we might have predefined fill values in our code that
vary by group. Since the groups have a name attribute set internally, we
can use that:

    \begin{Verbatim}[commandchars=\\\{\}]
{\color{incolor}In [{\color{incolor}128}]:} \PY{n}{fill\PYZus{}values} \PY{o}{=} \PY{p}{\PYZob{}}\PY{l+s+s1}{\PYZsq{}}\PY{l+s+s1}{East}\PY{l+s+s1}{\PYZsq{}}\PY{p}{:} \PY{l+m+mf}{0.5}\PY{p}{,} \PY{l+s+s1}{\PYZsq{}}\PY{l+s+s1}{West}\PY{l+s+s1}{\PYZsq{}}\PY{p}{:} \PY{o}{\PYZhy{}}\PY{l+m+mi}{1}\PY{p}{\PYZcb{}}
\end{Verbatim}


    \begin{Verbatim}[commandchars=\\\{\}]
{\color{incolor}In [{\color{incolor}129}]:} \PY{n}{fill\PYZus{}func} \PY{o}{=} \PY{k}{lambda} \PY{n}{g}\PY{p}{:} \PY{n}{g}\PY{o}{.}\PY{n}{fillna}\PY{p}{(}\PY{n}{fill\PYZus{}values}\PY{p}{[}\PY{n}{g}\PY{o}{.}\PY{n}{name}\PY{p}{]}\PY{p}{)}
\end{Verbatim}


    \begin{Verbatim}[commandchars=\\\{\}]
{\color{incolor}In [{\color{incolor}130}]:} \PY{n}{data}\PY{o}{.}\PY{n}{groupby}\PY{p}{(}\PY{n}{group\PYZus{}key}\PY{p}{)}\PY{o}{.}\PY{n}{apply}\PY{p}{(}\PY{n}{fill\PYZus{}func}\PY{p}{)}
\end{Verbatim}


\begin{Verbatim}[commandchars=\\\{\}]
{\color{outcolor}Out[{\color{outcolor}130}]:} Ohio          1.895558
          New York      1.324299
          Vermont       0.500000
          Florida      -1.442373
          Oregon        1.122452
          Nevada       -1.000000
          California    0.423392
          Idaho        -1.000000
          dtype: float64
\end{Verbatim}
            
    \hypertarget{example-random-sample-and-permtation}{%
\paragraph{Example: Random Sample and
Permtation}\label{example-random-sample-and-permtation}}

    Suppose we wanted to draw a random sample (with or without replacement)
from a large dataset for Monte Carlo simulation purposes or some other
application. There are a number of way to perform the ``draws''; here we
use the \emph{sample}() method for Series.

To demonstrate, here's way to construct a deck of English-style playing
cards:

    \begin{Verbatim}[commandchars=\\\{\}]
{\color{incolor}In [{\color{incolor}131}]:} \PY{n}{suits} \PY{o}{=} \PY{p}{[}\PY{l+s+s1}{\PYZsq{}}\PY{l+s+s1}{H}\PY{l+s+s1}{\PYZsq{}}\PY{p}{,} \PY{l+s+s1}{\PYZsq{}}\PY{l+s+s1}{S}\PY{l+s+s1}{\PYZsq{}}\PY{p}{,} \PY{l+s+s1}{\PYZsq{}}\PY{l+s+s1}{C}\PY{l+s+s1}{\PYZsq{}}\PY{p}{,} \PY{l+s+s1}{\PYZsq{}}\PY{l+s+s1}{D}\PY{l+s+s1}{\PYZsq{}}\PY{p}{]} \PY{c+c1}{\PYZsh{} Heart, Spades, Clubs, Diamonds}
\end{Verbatim}


    \begin{Verbatim}[commandchars=\\\{\}]
{\color{incolor}In [{\color{incolor}132}]:} \PY{n}{card\PYZus{}val} \PY{o}{=} \PY{p}{(}\PY{n+nb}{list}\PY{p}{(}\PY{n+nb}{range}\PY{p}{(}\PY{l+m+mi}{1}\PY{p}{,} \PY{l+m+mi}{11}\PY{p}{)}\PY{p}{)} \PY{o}{+} \PY{p}{[}\PY{l+m+mi}{10}\PY{p}{]}\PY{o}{*}\PY{l+m+mi}{3}\PY{p}{)}\PY{o}{*}\PY{l+m+mi}{4}
\end{Verbatim}


    \begin{Verbatim}[commandchars=\\\{\}]
{\color{incolor}In [{\color{incolor}134}]:} \PY{n}{base\PYZus{}names} \PY{o}{=} \PY{p}{[}\PY{l+s+s1}{\PYZsq{}}\PY{l+s+s1}{A}\PY{l+s+s1}{\PYZsq{}}\PY{p}{]} \PY{o}{+} \PY{n+nb}{list}\PY{p}{(}\PY{n+nb}{range}\PY{p}{(}\PY{l+m+mi}{2}\PY{p}{,} \PY{l+m+mi}{11}\PY{p}{)}\PY{p}{)} \PY{o}{+} \PY{p}{[}\PY{l+s+s1}{\PYZsq{}}\PY{l+s+s1}{J}\PY{l+s+s1}{\PYZsq{}}\PY{p}{,} \PY{l+s+s1}{\PYZsq{}}\PY{l+s+s1}{Q}\PY{l+s+s1}{\PYZsq{}}\PY{p}{,} \PY{l+s+s1}{\PYZsq{}}\PY{l+s+s1}{K}\PY{l+s+s1}{\PYZsq{}}\PY{p}{]}
\end{Verbatim}


    \begin{Verbatim}[commandchars=\\\{\}]
{\color{incolor}In [{\color{incolor}135}]:} \PY{n}{cards} \PY{o}{=} \PY{p}{[}\PY{p}{]}
          \PY{k}{for} \PY{n}{suit} \PY{o+ow}{in} \PY{n}{suits}\PY{p}{:}
              \PY{n}{cards}\PY{o}{.}\PY{n}{extend}\PY{p}{(}\PY{n+nb}{str}\PY{p}{(}\PY{n}{num}\PY{p}{)} \PY{o}{+} \PY{n}{suit} \PY{k}{for} \PY{n}{num} \PY{o+ow}{in} \PY{n}{base\PYZus{}names}\PY{p}{)}
\end{Verbatim}


    \begin{Verbatim}[commandchars=\\\{\}]
{\color{incolor}In [{\color{incolor}136}]:} \PY{n}{deck} \PY{o}{=} \PY{n}{pd}\PY{o}{.}\PY{n}{Series}\PY{p}{(}\PY{n}{card\PYZus{}val}\PY{p}{,} \PY{n}{index}\PY{o}{=}\PY{n}{cards}\PY{p}{)}
\end{Verbatim}


    \begin{Verbatim}[commandchars=\\\{\}]
{\color{incolor}In [{\color{incolor}137}]:} \PY{n}{deck}\PY{p}{[}\PY{p}{:}\PY{l+m+mi}{13}\PY{p}{]}
\end{Verbatim}


\begin{Verbatim}[commandchars=\\\{\}]
{\color{outcolor}Out[{\color{outcolor}137}]:} AH      1
          2H      2
          3H      3
          4H      4
          5H      5
          6H      6
          7H      7
          8H      8
          9H      9
          10H    10
          JH     10
          QH     10
          KH     10
          dtype: int64
\end{Verbatim}
            
    Now, based on what I said before, drawing a hand of five cards from the
deck could be written as:

    \begin{Verbatim}[commandchars=\\\{\}]
{\color{incolor}In [{\color{incolor}144}]:} \PY{k}{def} \PY{n+nf}{draw}\PY{p}{(}\PY{n}{deck}\PY{p}{,} \PY{n}{n}\PY{o}{=}\PY{l+m+mi}{5}\PY{p}{)}\PY{p}{:}
              \PY{k}{return} \PY{n}{deck}\PY{o}{.}\PY{n}{sample}\PY{p}{(}\PY{n}{n}\PY{p}{)}
\end{Verbatim}


    \begin{Verbatim}[commandchars=\\\{\}]
{\color{incolor}In [{\color{incolor}145}]:} \PY{n}{draw}\PY{p}{(}\PY{n}{deck}\PY{p}{)}
\end{Verbatim}


\begin{Verbatim}[commandchars=\\\{\}]
{\color{outcolor}Out[{\color{outcolor}145}]:} 4D    4
          8S    8
          2D    2
          AS    1
          8D    8
          dtype: int64
\end{Verbatim}
            
    Suppose we wanted two random cards from each suit. Because the suit is
the last character of each card name, we could group based on this and
use \emph{apply}():

    \begin{Verbatim}[commandchars=\\\{\}]
{\color{incolor}In [{\color{incolor}146}]:} \PY{n}{get\PYZus{}suit} \PY{o}{=} \PY{k}{lambda} \PY{n}{card}\PY{p}{:} \PY{n}{card}\PY{p}{[}\PY{o}{\PYZhy{}}\PY{l+m+mi}{1}\PY{p}{]}
\end{Verbatim}


    \begin{Verbatim}[commandchars=\\\{\}]
{\color{incolor}In [{\color{incolor}147}]:} \PY{n}{deck}\PY{o}{.}\PY{n}{groupby}\PY{p}{(}\PY{n}{get\PYZus{}suit}\PY{p}{)}\PY{o}{.}\PY{n}{apply}\PY{p}{(}\PY{n}{draw}\PY{p}{,} \PY{n}{n}\PY{o}{=}\PY{l+m+mi}{3}\PY{p}{)}
\end{Verbatim}


\begin{Verbatim}[commandchars=\\\{\}]
{\color{outcolor}Out[{\color{outcolor}147}]:} C  8C      8
             5C      5
             10C    10
          D  QD     10
             7D      7
             3D      3
          H  6H      6
             KH     10
             2H      2
          S  10S    10
             AS      1
             9S      9
          dtype: int64
\end{Verbatim}
            
    Alternatively, we could write:

    \begin{Verbatim}[commandchars=\\\{\}]
{\color{incolor}In [{\color{incolor}149}]:} \PY{n}{deck}\PY{o}{.}\PY{n}{groupby}\PY{p}{(}\PY{n}{get\PYZus{}suit}\PY{p}{,} \PY{n}{group\PYZus{}keys}\PY{o}{=}\PY{k+kc}{False}\PY{p}{)}\PY{o}{.}\PY{n}{apply}\PY{p}{(}\PY{n}{draw}\PY{p}{,} \PY{n}{n}\PY{o}{=}\PY{l+m+mi}{3}\PY{p}{)}
\end{Verbatim}


\begin{Verbatim}[commandchars=\\\{\}]
{\color{outcolor}Out[{\color{outcolor}149}]:} 10C    10
          AC      1
          JC     10
          3D      3
          QD     10
          KD     10
          8H      8
          QH     10
          10H    10
          9S      9
          AS      1
          6S      6
          dtype: int64
\end{Verbatim}
            
    \hypertarget{example-group-weighted-average-and-correlation}{%
\paragraph{Example: Group Weighted Average and
Correlation}\label{example-group-weighted-average-and-correlation}}

    Under the split-apply-combine paradigm of \emph{groupby}(), operations
between columns in a DataFrame or two Series, such as a group weighted
average, are possible. As an example, take this dataset containing keys,
values, and some weights:

    \begin{Verbatim}[commandchars=\\\{\}]
{\color{incolor}In [{\color{incolor}151}]:} \PY{n}{df} \PY{o}{=} \PY{n}{pd}\PY{o}{.}\PY{n}{DataFrame}\PY{p}{(}\PY{p}{\PYZob{}}\PY{l+s+s2}{\PYZdq{}}\PY{l+s+s2}{category}\PY{l+s+s2}{\PYZdq{}}\PY{p}{:} \PY{p}{[}\PY{l+s+s1}{\PYZsq{}}\PY{l+s+s1}{a}\PY{l+s+s1}{\PYZsq{}}\PY{p}{,} \PY{l+s+s1}{\PYZsq{}}\PY{l+s+s1}{a}\PY{l+s+s1}{\PYZsq{}}\PY{p}{,} \PY{l+s+s1}{\PYZsq{}}\PY{l+s+s1}{a}\PY{l+s+s1}{\PYZsq{}}\PY{p}{,} \PY{l+s+s1}{\PYZsq{}}\PY{l+s+s1}{a}\PY{l+s+s1}{\PYZsq{}}\PY{p}{,} \PY{l+s+s1}{\PYZsq{}}\PY{l+s+s1}{b}\PY{l+s+s1}{\PYZsq{}}\PY{p}{,} \PY{l+s+s1}{\PYZsq{}}\PY{l+s+s1}{b}\PY{l+s+s1}{\PYZsq{}}\PY{p}{,} \PY{l+s+s1}{\PYZsq{}}\PY{l+s+s1}{b}\PY{l+s+s1}{\PYZsq{}}\PY{p}{,} \PY{l+s+s1}{\PYZsq{}}\PY{l+s+s1}{b}\PY{l+s+s1}{\PYZsq{}}\PY{p}{]}\PY{p}{,}
                           \PY{l+s+s1}{\PYZsq{}}\PY{l+s+s1}{data}\PY{l+s+s1}{\PYZsq{}}\PY{p}{:} \PY{n}{np}\PY{o}{.}\PY{n}{random}\PY{o}{.}\PY{n}{randn}\PY{p}{(}\PY{l+m+mi}{8}\PY{p}{)}\PY{p}{,} 
                           \PY{l+s+s1}{\PYZsq{}}\PY{l+s+s1}{weights}\PY{l+s+s1}{\PYZsq{}}\PY{p}{:} \PY{n}{np}\PY{o}{.}\PY{n}{random}\PY{o}{.}\PY{n}{rand}\PY{p}{(}\PY{l+m+mi}{8}\PY{p}{)}\PY{p}{\PYZcb{}}\PY{p}{)}
\end{Verbatim}


    \begin{Verbatim}[commandchars=\\\{\}]
{\color{incolor}In [{\color{incolor}152}]:} \PY{n}{df}
\end{Verbatim}


\begin{Verbatim}[commandchars=\\\{\}]
{\color{outcolor}Out[{\color{outcolor}152}]:}   category      data   weights
          0        a  0.988396  0.461152
          1        a -2.377541  0.913845
          2        a -0.338229  0.031887
          3        a  0.123738  0.932333
          4        b  0.806085  0.972196
          5        b -1.751842  0.757740
          6        b -0.128470  0.220630
          7        b  1.315497  0.338298
\end{Verbatim}
            
    The group weighted average by \emph{category} would then be:

    \begin{Verbatim}[commandchars=\\\{\}]
{\color{incolor}In [{\color{incolor}153}]:} \PY{n}{grouped} \PY{o}{=} \PY{n}{df}\PY{o}{.}\PY{n}{groupby}\PY{p}{(}\PY{l+s+s1}{\PYZsq{}}\PY{l+s+s1}{category}\PY{l+s+s1}{\PYZsq{}}\PY{p}{)}
\end{Verbatim}


    \begin{Verbatim}[commandchars=\\\{\}]
{\color{incolor}In [{\color{incolor}154}]:} \PY{n}{get\PYZus{}wavg} \PY{o}{=} \PY{k}{lambda} \PY{n}{g}\PY{p}{:} \PY{n}{np}\PY{o}{.}\PY{n}{average}\PY{p}{(}\PY{n}{g}\PY{p}{[}\PY{l+s+s1}{\PYZsq{}}\PY{l+s+s1}{data}\PY{l+s+s1}{\PYZsq{}}\PY{p}{]}\PY{p}{,} \PY{n}{weights}\PY{o}{=}\PY{n}{g}\PY{p}{[}\PY{l+s+s1}{\PYZsq{}}\PY{l+s+s1}{weights}\PY{l+s+s1}{\PYZsq{}}\PY{p}{]}\PY{p}{)}
\end{Verbatim}


    \begin{Verbatim}[commandchars=\\\{\}]
{\color{incolor}In [{\color{incolor}155}]:} \PY{n}{grouped}\PY{o}{.}\PY{n}{apply}\PY{p}{(}\PY{n}{get\PYZus{}wavg}\PY{p}{)}
\end{Verbatim}


\begin{Verbatim}[commandchars=\\\{\}]
{\color{outcolor}Out[{\color{outcolor}155}]:} category
          a   -0.689258
          b   -0.055522
          dtype: float64
\end{Verbatim}
            
    As another example, consider a financial dataset originally obtained
from Yahoo! Finance containing end-of-day prices for a few stocks and
the S\&P 500 index (the SPX symbol):

    \begin{Verbatim}[commandchars=\\\{\}]
{\color{incolor}In [{\color{incolor}157}]:} \PY{n}{close\PYZus{}px} \PY{o}{=} \PY{n}{pd}\PY{o}{.}\PY{n}{read\PYZus{}csv}\PY{p}{(}\PY{l+s+s1}{\PYZsq{}}\PY{l+s+s1}{./examples/stock\PYZus{}px\PYZus{}2.csv}\PY{l+s+s1}{\PYZsq{}}\PY{p}{,} \PY{n}{parse\PYZus{}dates}\PY{o}{=}\PY{k+kc}{True}\PY{p}{,}
                                \PY{n}{index\PYZus{}col}\PY{o}{=}\PY{l+m+mi}{0}\PY{p}{)}
\end{Verbatim}


    \begin{Verbatim}[commandchars=\\\{\}]
{\color{incolor}In [{\color{incolor}158}]:} \PY{n}{close\PYZus{}px}\PY{o}{.}\PY{n}{info}\PY{p}{(}\PY{p}{)}
\end{Verbatim}


    \begin{Verbatim}[commandchars=\\\{\}]
<class 'pandas.core.frame.DataFrame'>
DatetimeIndex: 2214 entries, 2003-01-02 to 2011-10-14
Data columns (total 4 columns):
AAPL    2214 non-null float64
MSFT    2214 non-null float64
XOM     2214 non-null float64
SPX     2214 non-null float64
dtypes: float64(4)
memory usage: 86.5 KB

    \end{Verbatim}

    \begin{Verbatim}[commandchars=\\\{\}]
{\color{incolor}In [{\color{incolor}159}]:} \PY{n}{close\PYZus{}px}\PY{p}{[}\PY{o}{\PYZhy{}}\PY{l+m+mi}{5}\PY{p}{:}\PY{p}{]}
\end{Verbatim}


\begin{Verbatim}[commandchars=\\\{\}]
{\color{outcolor}Out[{\color{outcolor}159}]:}               AAPL   MSFT    XOM      SPX
          2011-10-10  388.81  26.94  76.28  1194.89
          2011-10-11  400.29  27.00  76.27  1195.54
          2011-10-12  402.19  26.96  77.16  1207.25
          2011-10-13  408.43  27.18  76.37  1203.66
          2011-10-14  422.00  27.27  78.11  1224.58
\end{Verbatim}
            
    One task of interest might be to compute a DataFrame consisting of the
yearly correlations of daily returns (computed from percent changes)
with SPX. As a way to do this, we create a function to compute a
pairwise correlation of each column with SPX:

    \begin{Verbatim}[commandchars=\\\{\}]
{\color{incolor}In [{\color{incolor}160}]:} \PY{n}{spx\PYZus{}corr} \PY{o}{=} \PY{k}{lambda} \PY{n}{x}\PY{p}{:} \PY{n}{x}\PY{o}{.}\PY{n}{corrwith}\PY{p}{(}\PY{n}{x}\PY{p}{[}\PY{l+s+s1}{\PYZsq{}}\PY{l+s+s1}{SPX}\PY{l+s+s1}{\PYZsq{}}\PY{p}{]}\PY{p}{)}
\end{Verbatim}


    Next, we compute percent changes on \emph{close\_px} using
\emph{pct\_change}()):

    \begin{Verbatim}[commandchars=\\\{\}]
{\color{incolor}In [{\color{incolor}161}]:} \PY{n}{rets} \PY{o}{=} \PY{n}{close\PYZus{}px}\PY{o}{.}\PY{n}{pct\PYZus{}change}\PY{p}{(}\PY{p}{)}\PY{o}{.}\PY{n}{dropna}\PY{p}{(}\PY{p}{)}
\end{Verbatim}


    \begin{Verbatim}[commandchars=\\\{\}]
{\color{incolor}In [{\color{incolor}163}]:} \PY{n}{rets}\PY{p}{[}\PY{p}{:}\PY{l+m+mi}{3}\PY{p}{]}
\end{Verbatim}


\begin{Verbatim}[commandchars=\\\{\}]
{\color{outcolor}Out[{\color{outcolor}163}]:}                 AAPL      MSFT       XOM       SPX
          2003-01-03  0.006757  0.001421  0.000684 -0.000484
          2003-01-06  0.000000  0.017975  0.024624  0.022474
          2003-01-07 -0.002685  0.019052 -0.033712 -0.006545
\end{Verbatim}
            
    Lastly, we group these percent changes by year, which can be extracted
from each low label with a one-line function that returns the
\emph{year} attribuet of each \emph{datetime} label:

    \begin{Verbatim}[commandchars=\\\{\}]
{\color{incolor}In [{\color{incolor}162}]:} \PY{n}{get\PYZus{}year} \PY{o}{=} \PY{k}{lambda} \PY{n}{x}\PY{p}{:} \PY{n}{x}\PY{o}{.}\PY{n}{year}
\end{Verbatim}


    \begin{Verbatim}[commandchars=\\\{\}]
{\color{incolor}In [{\color{incolor}164}]:} \PY{n}{by\PYZus{}year} \PY{o}{=} \PY{n}{rets}\PY{o}{.}\PY{n}{groupby}\PY{p}{(}\PY{n}{get\PYZus{}year}\PY{p}{)}
\end{Verbatim}


    \begin{Verbatim}[commandchars=\\\{\}]
{\color{incolor}In [{\color{incolor}168}]:} \PY{n}{by\PYZus{}year}\PY{o}{.}\PY{n}{apply}\PY{p}{(}\PY{n}{spx\PYZus{}corr}\PY{p}{)}
\end{Verbatim}


\begin{Verbatim}[commandchars=\\\{\}]
{\color{outcolor}Out[{\color{outcolor}168}]:}           AAPL      MSFT       XOM  SPX
          2003  0.541124  0.745174  0.661265  1.0
          2004  0.374283  0.588531  0.557742  1.0
          2005  0.467540  0.562374  0.631010  1.0
          2006  0.428267  0.406126  0.518514  1.0
          2007  0.508118  0.658770  0.786264  1.0
          2008  0.681434  0.804626  0.828303  1.0
          2009  0.707103  0.654902  0.797921  1.0
          2010  0.710105  0.730118  0.839057  1.0
          2011  0.691931  0.800996  0.859975  1.0
\end{Verbatim}
            
    We could also compute inter-column correlation. Here we compute the
annual correlation between Apple and Microsoft:

    \begin{Verbatim}[commandchars=\\\{\}]
{\color{incolor}In [{\color{incolor}170}]:} \PY{n}{by\PYZus{}year}\PY{o}{.}\PY{n}{apply}\PY{p}{(}\PY{k}{lambda} \PY{n}{g}\PY{p}{:} \PY{n}{g}\PY{p}{[}\PY{l+s+s1}{\PYZsq{}}\PY{l+s+s1}{AAPL}\PY{l+s+s1}{\PYZsq{}}\PY{p}{]}\PY{o}{.}\PY{n}{corr}\PY{p}{(}\PY{n}{g}\PY{p}{[}\PY{l+s+s1}{\PYZsq{}}\PY{l+s+s1}{MSFT}\PY{l+s+s1}{\PYZsq{}}\PY{p}{]}\PY{p}{)}\PY{p}{)}
\end{Verbatim}


\begin{Verbatim}[commandchars=\\\{\}]
{\color{outcolor}Out[{\color{outcolor}170}]:} 2003    0.480868
          2004    0.259024
          2005    0.300093
          2006    0.161735
          2007    0.417738
          2008    0.611901
          2009    0.432738
          2010    0.571946
          2011    0.581987
          dtype: float64
\end{Verbatim}
            
    \hypertarget{example-group-wise-linear-regression}{%
\paragraph{Example: Group-Wise Linear
Regression}\label{example-group-wise-linear-regression}}

    In the same theme as the previous example, we can use \emph{groupby}()
to perform more complex group-wise statitstical analysis, as long as the
function returns a pandas object or scalar. For example, we can define
the following \emph{regress}() function (using the statsmodels
econometrics library), which executes an ordinary least square
(OLS,最小二乘法) on each chunk of data:

    \begin{Verbatim}[commandchars=\\\{\}]
{\color{incolor}In [{\color{incolor}171}]:} \PY{k+kn}{import} \PY{n+nn}{statsmodels}\PY{n+nn}{.}\PY{n+nn}{api} \PY{k}{as} \PY{n+nn}{sm}
\end{Verbatim}


    \begin{Verbatim}[commandchars=\\\{\}]
C:\textbackslash{}Anaconda3\textbackslash{}lib\textbackslash{}site-packages\textbackslash{}statsmodels\textbackslash{}compat\textbackslash{}pandas.py:56: FutureWarning: The pandas.core.datetools module is deprecated and will be removed in a future version. Please use the pandas.tseries module instead.
  from pandas.core import datetools

    \end{Verbatim}

    \begin{Verbatim}[commandchars=\\\{\}]
{\color{incolor}In [{\color{incolor}198}]:} \PY{k}{def} \PY{n+nf}{regress}\PY{p}{(}\PY{n}{data}\PY{p}{,} \PY{n}{yvar}\PY{p}{,} \PY{n}{xvars}\PY{p}{)}\PY{p}{:}
              \PY{c+c1}{\PYZsh{}print(data)}
              \PY{n}{Y} \PY{o}{=} \PY{n}{data}\PY{p}{[}\PY{n}{yvar}\PY{p}{]}
              \PY{n}{X} \PY{o}{=} \PY{n}{data}\PY{p}{[}\PY{n}{xvars}\PY{p}{]}
              \PY{c+c1}{\PYZsh{}print(X[:5])}
              \PY{n}{X}\PY{p}{[}\PY{l+s+s1}{\PYZsq{}}\PY{l+s+s1}{intercept}\PY{l+s+s1}{\PYZsq{}}\PY{p}{]} \PY{o}{=} \PY{l+m+mf}{1.}
              
              \PY{n}{result} \PY{o}{=} \PY{n}{sm}\PY{o}{.}\PY{n}{OLS}\PY{p}{(}\PY{n}{Y}\PY{p}{,} \PY{n}{X}\PY{p}{)}\PY{o}{.}\PY{n}{fit}\PY{p}{(}\PY{p}{)}
              \PY{k}{return} \PY{n}{result}\PY{o}{.}\PY{n}{params}
\end{Verbatim}


    Now, to run a yearly linear regression of AAPL on SPX returns, execute:

    \begin{Verbatim}[commandchars=\\\{\}]
{\color{incolor}In [{\color{incolor}199}]:} \PY{n}{ans} \PY{o}{=} \PY{n}{by\PYZus{}year}\PY{o}{.}\PY{n}{apply}\PY{p}{(}\PY{n}{regress}\PY{p}{,} \PY{l+s+s1}{\PYZsq{}}\PY{l+s+s1}{AAPL}\PY{l+s+s1}{\PYZsq{}}\PY{p}{,} \PY{p}{[}\PY{l+s+s1}{\PYZsq{}}\PY{l+s+s1}{SPX}\PY{l+s+s1}{\PYZsq{}}\PY{p}{]}\PY{p}{)}
\end{Verbatim}


    \hypertarget{pivot-table-ux6570ux636eux900fux89c6ux8868-and-cross-tabulation-ux4ea4ux53c9ux8868}{%
\subsubsection{10.4 Pivot Table (数据透视表) and Cross-Tabulation
(交叉表)}\label{pivot-table-ux6570ux636eux900fux89c6ux8868-and-cross-tabulation-ux4ea4ux53c9ux8868}}

    Returning to the tipping dataset, suppose we want to compute a table of
group mean (the default \emph{pandas.pivot\_talbe}() aggregation type)
arranged by \emph{day} and \emph{smoker} on the rows:

    \begin{Verbatim}[commandchars=\\\{\}]
{\color{incolor}In [{\color{incolor}204}]:} \PY{n}{tips}\PY{o}{.}\PY{n}{pivot\PYZus{}table}\PY{p}{(}\PY{n}{index}\PY{o}{=}\PY{p}{[}\PY{l+s+s1}{\PYZsq{}}\PY{l+s+s1}{day}\PY{l+s+s1}{\PYZsq{}}\PY{p}{,} \PY{l+s+s1}{\PYZsq{}}\PY{l+s+s1}{smoker}\PY{l+s+s1}{\PYZsq{}}\PY{p}{]}\PY{p}{)}
\end{Verbatim}


\begin{Verbatim}[commandchars=\\\{\}]
{\color{outcolor}Out[{\color{outcolor}204}]:}                  size       tip   tip\_pct  total\_bill
          day  smoker                                          
          Fri  No      2.250000  2.812500  0.151650   18.420000
               Yes     2.066667  2.714000  0.174783   16.813333
          Sat  No      2.555556  3.102889  0.158048   19.661778
               Yes     2.476190  2.875476  0.147906   21.276667
          Sun  No      2.929825  3.167895  0.160113   20.506667
               Yes     2.578947  3.516842  0.187250   24.120000
          Thur No      2.488889  2.673778  0.160298   17.113111
               Yes     2.352941  3.030000  0.163863   19.190588
\end{Verbatim}
            
    This operation tips.pivot\_table(index={[}`day', `smoker'{]}) is
equivalent to:

    \begin{Verbatim}[commandchars=\\\{\}]
{\color{incolor}In [{\color{incolor}205}]:} \PY{n}{tips}\PY{o}{.}\PY{n}{groupby}\PY{p}{(}\PY{p}{[}\PY{l+s+s1}{\PYZsq{}}\PY{l+s+s1}{day}\PY{l+s+s1}{\PYZsq{}}\PY{p}{,} \PY{l+s+s1}{\PYZsq{}}\PY{l+s+s1}{smoker}\PY{l+s+s1}{\PYZsq{}}\PY{p}{]}\PY{p}{)}\PY{o}{.}\PY{n}{apply}\PY{p}{(}\PY{n}{np}\PY{o}{.}\PY{n}{mean}\PY{p}{)}
\end{Verbatim}


\begin{Verbatim}[commandchars=\\\{\}]
{\color{outcolor}Out[{\color{outcolor}205}]:}              total\_bill       tip      size   tip\_pct
          day  smoker                                          
          Fri  No       18.420000  2.812500  2.250000  0.151650
               Yes      16.813333  2.714000  2.066667  0.174783
          Sat  No       19.661778  3.102889  2.555556  0.158048
               Yes      21.276667  2.875476  2.476190  0.147906
          Sun  No       20.506667  3.167895  2.929825  0.160113
               Yes      24.120000  3.516842  2.578947  0.187250
          Thur No       17.113111  2.673778  2.488889  0.160298
               Yes      19.190588  3.030000  2.352941  0.163863
\end{Verbatim}
            
    Now, suppose we want to aggregate only \emph{tip\_pct} and \emph{size},
and additionaly group by time. I'll put \emph{smoker} in the tables
columns and \emph{day} in the rows:

    \begin{Verbatim}[commandchars=\\\{\}]
{\color{incolor}In [{\color{incolor}206}]:} \PY{n}{tips}\PY{o}{.}\PY{n}{pivot\PYZus{}table}\PY{p}{(}\PY{p}{[}\PY{l+s+s1}{\PYZsq{}}\PY{l+s+s1}{tip\PYZus{}pct}\PY{l+s+s1}{\PYZsq{}}\PY{p}{,} \PY{l+s+s1}{\PYZsq{}}\PY{l+s+s1}{size}\PY{l+s+s1}{\PYZsq{}}\PY{p}{]}\PY{p}{,} \PY{n}{index}\PY{o}{=}\PY{p}{[}\PY{l+s+s1}{\PYZsq{}}\PY{l+s+s1}{time}\PY{l+s+s1}{\PYZsq{}}\PY{p}{,} \PY{l+s+s1}{\PYZsq{}}\PY{l+s+s1}{day}\PY{l+s+s1}{\PYZsq{}}\PY{p}{]}\PY{p}{,} \PY{n}{columns}\PY{o}{=}\PY{p}{[}\PY{l+s+s1}{\PYZsq{}}\PY{l+s+s1}{smoker}\PY{l+s+s1}{\PYZsq{}}\PY{p}{]}\PY{p}{)}
\end{Verbatim}


\begin{Verbatim}[commandchars=\\\{\}]
{\color{outcolor}Out[{\color{outcolor}206}]:}                  size             tip\_pct          
          smoker             No       Yes        No       Yes
          time   day                                         
          Dinner Fri   2.000000  2.222222  0.139622  0.165347
                 Sat   2.555556  2.476190  0.158048  0.147906
                 Sun   2.929825  2.578947  0.160113  0.187250
                 Thur  2.000000       NaN  0.159744       NaN
          Lunch  Fri   3.000000  1.833333  0.187735  0.188937
                 Thur  2.500000  2.352941  0.160311  0.163863
\end{Verbatim}
            
    We could augment this table to include partial total by passing
\emph{margins=True}. This has the effect of adding \emph{All} column and
row labels, with corresponding values being the group statistics for all
the data within a single tier:

    \begin{Verbatim}[commandchars=\\\{\}]
{\color{incolor}In [{\color{incolor}207}]:} \PY{n}{tips}\PY{o}{.}\PY{n}{pivot\PYZus{}table}\PY{p}{(}\PY{p}{[}\PY{l+s+s1}{\PYZsq{}}\PY{l+s+s1}{tip\PYZus{}pct}\PY{l+s+s1}{\PYZsq{}}\PY{p}{,} \PY{l+s+s1}{\PYZsq{}}\PY{l+s+s1}{size}\PY{l+s+s1}{\PYZsq{}}\PY{p}{]}\PY{p}{,} \PY{n}{index}\PY{o}{=}\PY{p}{[}\PY{l+s+s1}{\PYZsq{}}\PY{l+s+s1}{time}\PY{l+s+s1}{\PYZsq{}}\PY{p}{,} \PY{l+s+s1}{\PYZsq{}}\PY{l+s+s1}{day}\PY{l+s+s1}{\PYZsq{}}\PY{p}{]}\PY{p}{,}
                          \PY{n}{columns}\PY{o}{=}\PY{p}{[}\PY{l+s+s1}{\PYZsq{}}\PY{l+s+s1}{smoker}\PY{l+s+s1}{\PYZsq{}}\PY{p}{]}\PY{p}{,} \PY{n}{margins}\PY{o}{=}\PY{k+kc}{True}\PY{p}{)}
\end{Verbatim}


\begin{Verbatim}[commandchars=\\\{\}]
{\color{outcolor}Out[{\color{outcolor}207}]:}                  size                       tip\_pct                    
          smoker             No       Yes       All        No       Yes       All
          time   day                                                             
          Dinner Fri   2.000000  2.222222  2.166667  0.139622  0.165347  0.158916
                 Sat   2.555556  2.476190  2.517241  0.158048  0.147906  0.153152
                 Sun   2.929825  2.578947  2.842105  0.160113  0.187250  0.166897
                 Thur  2.000000       NaN  2.000000  0.159744       NaN  0.159744
          Lunch  Fri   3.000000  1.833333  2.000000  0.187735  0.188937  0.188765
                 Thur  2.500000  2.352941  2.459016  0.160311  0.163863  0.161301
          All          2.668874  2.408602  2.569672  0.159328  0.163196  0.160803
\end{Verbatim}
            
    \begin{Verbatim}[commandchars=\\\{\}]
{\color{incolor}In [{\color{incolor}208}]:} \PY{p}{(}\PY{l+m+mf}{2.000000} \PY{o}{+} \PY{l+m+mf}{2.555556} \PY{o}{+} \PY{l+m+mf}{2.929825} \PY{o}{+} \PY{l+m+mf}{2.000000} \PY{o}{+} \PY{l+m+mi}{3} \PY{o}{+} \PY{l+m+mf}{2.5} \PY{o}{+} \PY{l+m+mf}{2.668874}\PY{p}{)} \PY{o}{/} \PY{l+m+mi}{7}
\end{Verbatim}


\begin{Verbatim}[commandchars=\\\{\}]
{\color{outcolor}Out[{\color{outcolor}208}]:} 2.5220364285714285
\end{Verbatim}
            
    Here, the All values are means without taking into account smoker versus
non- smoker (the All columns) or any of the two levels of grouping on
the rows (the All row).\\
\textbf{?????}

    To use a different aggregation function, pass it to \emph{aggfunc}
argument. For example, `count' or `len' will give you a cross-tabulation
(count or frequency) of group sizes:

    \begin{Verbatim}[commandchars=\\\{\}]
{\color{incolor}In [{\color{incolor}210}]:} \PY{n}{tips}\PY{o}{.}\PY{n}{pivot\PYZus{}table}\PY{p}{(}\PY{l+s+s1}{\PYZsq{}}\PY{l+s+s1}{tip\PYZus{}pct}\PY{l+s+s1}{\PYZsq{}}\PY{p}{,} \PY{n}{index}\PY{o}{=}\PY{p}{[}\PY{l+s+s1}{\PYZsq{}}\PY{l+s+s1}{time}\PY{l+s+s1}{\PYZsq{}}\PY{p}{,} \PY{l+s+s1}{\PYZsq{}}\PY{l+s+s1}{smoker}\PY{l+s+s1}{\PYZsq{}}\PY{p}{]}\PY{p}{,} \PY{n}{columns}\PY{o}{=}\PY{l+s+s1}{\PYZsq{}}\PY{l+s+s1}{day}\PY{l+s+s1}{\PYZsq{}}\PY{p}{,}
                                             \PY{n}{aggfunc}\PY{o}{=}\PY{n+nb}{len}\PY{p}{,} \PY{n}{margins}\PY{o}{=}\PY{k+kc}{True}\PY{p}{)}
\end{Verbatim}


\begin{Verbatim}[commandchars=\\\{\}]
{\color{outcolor}Out[{\color{outcolor}210}]:} day             Fri   Sat   Sun  Thur    All
          time   smoker                               
          Dinner No       3.0  45.0  57.0   1.0  106.0
                 Yes      9.0  42.0  19.0   NaN   70.0
          Lunch  No       1.0   NaN   NaN  44.0   45.0
                 Yes      6.0   NaN   NaN  17.0   23.0
          All            19.0  87.0  76.0  62.0  244.0
\end{Verbatim}
            
    or use \emph{np.count\_nonzero} instead:

    \begin{Verbatim}[commandchars=\\\{\}]
{\color{incolor}In [{\color{incolor}212}]:} \PY{n}{tips}\PY{o}{.}\PY{n}{pivot\PYZus{}table}\PY{p}{(}\PY{l+s+s1}{\PYZsq{}}\PY{l+s+s1}{tip\PYZus{}pct}\PY{l+s+s1}{\PYZsq{}}\PY{p}{,} \PY{n}{index}\PY{o}{=}\PY{p}{[}\PY{l+s+s1}{\PYZsq{}}\PY{l+s+s1}{time}\PY{l+s+s1}{\PYZsq{}}\PY{p}{,} \PY{l+s+s1}{\PYZsq{}}\PY{l+s+s1}{smoker}\PY{l+s+s1}{\PYZsq{}}\PY{p}{]}\PY{p}{,} \PY{n}{columns}\PY{o}{=}\PY{l+s+s1}{\PYZsq{}}\PY{l+s+s1}{day}\PY{l+s+s1}{\PYZsq{}}\PY{p}{,}
                                             \PY{n}{aggfunc}\PY{o}{=}\PY{n}{np}\PY{o}{.}\PY{n}{count\PYZus{}nonzero}\PY{p}{,} \PY{n}{margins}\PY{o}{=}\PY{k+kc}{True}\PY{p}{)}
\end{Verbatim}


\begin{Verbatim}[commandchars=\\\{\}]
{\color{outcolor}Out[{\color{outcolor}212}]:} day             Fri   Sat   Sun  Thur    All
          time   smoker                               
          Dinner No       3.0  45.0  57.0   1.0  106.0
                 Yes      9.0  42.0  19.0   NaN   70.0
          Lunch  No       1.0   NaN   NaN  44.0   45.0
                 Yes      6.0   NaN   NaN  17.0   23.0
          All            19.0  87.0  76.0  62.0  244.0
\end{Verbatim}
            
    If some combination are empty (or otherwise NA), we may wish to pass a
fill\_value:

    \begin{Verbatim}[commandchars=\\\{\}]
{\color{incolor}In [{\color{incolor}229}]:} \PY{n}{tips}\PY{o}{.}\PY{n}{pivot\PYZus{}table}\PY{p}{(}\PY{l+s+s1}{\PYZsq{}}\PY{l+s+s1}{tip\PYZus{}pct}\PY{l+s+s1}{\PYZsq{}}\PY{p}{,} \PY{n}{index}\PY{o}{=}\PY{p}{[}\PY{l+s+s1}{\PYZsq{}}\PY{l+s+s1}{time}\PY{l+s+s1}{\PYZsq{}}\PY{p}{,} \PY{l+s+s1}{\PYZsq{}}\PY{l+s+s1}{size}\PY{l+s+s1}{\PYZsq{}}\PY{p}{,} \PY{l+s+s1}{\PYZsq{}}\PY{l+s+s1}{smoker}\PY{l+s+s1}{\PYZsq{}}\PY{p}{]}\PY{p}{,}
                          \PY{n}{columns}\PY{o}{=}\PY{l+s+s1}{\PYZsq{}}\PY{l+s+s1}{day}\PY{l+s+s1}{\PYZsq{}}\PY{p}{,} \PY{n}{aggfunc}\PY{o}{=}\PY{l+s+s1}{\PYZsq{}}\PY{l+s+s1}{mean}\PY{l+s+s1}{\PYZsq{}}\PY{p}{,} \PY{n}{fill\PYZus{}value}\PY{o}{=}\PY{l+m+mi}{0}\PY{p}{)}
\end{Verbatim}


\begin{Verbatim}[commandchars=\\\{\}]
{\color{outcolor}Out[{\color{outcolor}229}]:} day                      Fri       Sat       Sun      Thur
          time   size smoker                                        
          Dinner 1    No      0.000000  0.137931  0.000000  0.000000
                      Yes     0.000000  0.325733  0.000000  0.000000
                 2    No      0.139622  0.162705  0.168859  0.159744
                      Yes     0.171297  0.148668  0.207893  0.000000
                 3    No      0.000000  0.154661  0.152663  0.000000
                      Yes     0.000000  0.144995  0.152660  0.000000
                 4    No      0.000000  0.150096  0.148143  0.000000
                      Yes     0.117750  0.124515  0.193370  0.000000
                 5    No      0.000000  0.000000  0.206928  0.000000
                      Yes     0.000000  0.106572  0.065660  0.000000
                 6    No      0.000000  0.000000  0.103799  0.000000
          Lunch  1    No      0.000000  0.000000  0.000000  0.181728
                      Yes     0.223776  0.000000  0.000000  0.000000
                 2    No      0.000000  0.000000  0.000000  0.166005
                      Yes     0.181969  0.000000  0.000000  0.158843
                 3    No      0.187735  0.000000  0.000000  0.084246
                      Yes     0.000000  0.000000  0.000000  0.204952
                 4    No      0.000000  0.000000  0.000000  0.138919
                      Yes     0.000000  0.000000  0.000000  0.155410
                 5    No      0.000000  0.000000  0.000000  0.121389
                 6    No      0.000000  0.000000  0.000000  0.173706
\end{Verbatim}
            
    See the table below for a summary of \emph{pivot\_table}() methods:\\
\includegraphics{attachment:image.png}

    \hypertarget{cross-tabulations-crosstab}{%
\paragraph{Cross-Tabulations:
Crosstab}\label{cross-tabulations-crosstab}}

    A cross-tabulation is special case of pivot table that computes group
frequences. Here is an example:

    \begin{Verbatim}[commandchars=\\\{\}]
{\color{incolor}In [{\color{incolor}231}]:} \PY{n}{pd}\PY{o}{.}\PY{n}{crosstab}\PY{p}{(}\PY{p}{[}\PY{n}{tips}\PY{o}{.}\PY{n}{time}\PY{p}{,} \PY{n}{tips}\PY{o}{.}\PY{n}{day}\PY{p}{]}\PY{p}{,} \PY{n}{tips}\PY{o}{.}\PY{n}{smoker}\PY{p}{,} \PY{n}{margins}\PY{o}{=}\PY{k+kc}{True}\PY{p}{)}
\end{Verbatim}


\begin{Verbatim}[commandchars=\\\{\}]
{\color{outcolor}Out[{\color{outcolor}231}]:} smoker        No  Yes  All
          time   day                
          Dinner Fri     3    9   12
                 Sat    45   42   87
                 Sun    57   19   76
                 Thur    1    0    1
          Lunch  Fri     1    6    7
                 Thur   44   17   61
          All          151   93  244
\end{Verbatim}
            
    The first two arguments to \emph{pd.crosstab}() can either be an array
or Series or a list of arrays.

    本章完结!!!


    % Add a bibliography block to the postdoc
    
    
    
    \end{document}
