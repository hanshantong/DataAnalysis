
% Default to the notebook output style

    


% Inherit from the specified cell style.




    
\documentclass[11pt]{article}

    \usepackage{fontspec, xunicode, xltxtra}
    \setmainfont{Microsoft YaHei}
    
    \usepackage[T1]{fontenc}
    % Nicer default font (+ math font) than Computer Modern for most use cases
    \usepackage{mathpazo}

    % Basic figure setup, for now with no caption control since it's done
    % automatically by Pandoc (which extracts ![](path) syntax from Markdown).
    \usepackage{graphicx}
    % We will generate all images so they have a width \maxwidth. This means
    % that they will get their normal width if they fit onto the page, but
    % are scaled down if they would overflow the margins.
    \makeatletter
    \def\maxwidth{\ifdim\Gin@nat@width>\linewidth\linewidth
    \else\Gin@nat@width\fi}
    \makeatother
    \let\Oldincludegraphics\includegraphics
    % Set max figure width to be 80% of text width, for now hardcoded.
    \renewcommand{\includegraphics}[1]{\Oldincludegraphics[width=.8\maxwidth]{#1}}
    % Ensure that by default, figures have no caption (until we provide a
    % proper Figure object with a Caption API and a way to capture that
    % in the conversion process - todo).
    \usepackage{caption}
    \DeclareCaptionLabelFormat{nolabel}{}
    \captionsetup{labelformat=nolabel}

    \usepackage{adjustbox} % Used to constrain images to a maximum size 
    \usepackage{xcolor} % Allow colors to be defined
    \usepackage{enumerate} % Needed for markdown enumerations to work
    \usepackage{geometry} % Used to adjust the document margins
    \usepackage{amsmath} % Equations
    \usepackage{amssymb} % Equations
    \usepackage{textcomp} % defines textquotesingle
    % Hack from http://tex.stackexchange.com/a/47451/13684:
    \AtBeginDocument{%
        \def\PYZsq{\textquotesingle}% Upright quotes in Pygmentized code
    }
    \usepackage{upquote} % Upright quotes for verbatim code
    \usepackage{eurosym} % defines \euro
    \usepackage[mathletters]{ucs} % Extended unicode (utf-8) support
    \usepackage[utf8x]{inputenc} % Allow utf-8 characters in the tex document
    \usepackage{fancyvrb} % verbatim replacement that allows latex
    \usepackage{grffile} % extends the file name processing of package graphics 
                         % to support a larger range 
    % The hyperref package gives us a pdf with properly built
    % internal navigation ('pdf bookmarks' for the table of contents,
    % internal cross-reference links, web links for URLs, etc.)
    \usepackage{hyperref}
    \usepackage{longtable} % longtable support required by pandoc >1.10
    \usepackage{booktabs}  % table support for pandoc > 1.12.2
    \usepackage[inline]{enumitem} % IRkernel/repr support (it uses the enumerate* environment)
    \usepackage[normalem]{ulem} % ulem is needed to support strikethroughs (\sout)
                                % normalem makes italics be italics, not underlines
    \usepackage{mathrsfs}
    

    
    
    % Colors for the hyperref package
    \definecolor{urlcolor}{rgb}{0,.145,.698}
    \definecolor{linkcolor}{rgb}{.71,0.21,0.01}
    \definecolor{citecolor}{rgb}{.12,.54,.11}

    % ANSI colors
    \definecolor{ansi-black}{HTML}{3E424D}
    \definecolor{ansi-black-intense}{HTML}{282C36}
    \definecolor{ansi-red}{HTML}{E75C58}
    \definecolor{ansi-red-intense}{HTML}{B22B31}
    \definecolor{ansi-green}{HTML}{00A250}
    \definecolor{ansi-green-intense}{HTML}{007427}
    \definecolor{ansi-yellow}{HTML}{DDB62B}
    \definecolor{ansi-yellow-intense}{HTML}{B27D12}
    \definecolor{ansi-blue}{HTML}{208FFB}
    \definecolor{ansi-blue-intense}{HTML}{0065CA}
    \definecolor{ansi-magenta}{HTML}{D160C4}
    \definecolor{ansi-magenta-intense}{HTML}{A03196}
    \definecolor{ansi-cyan}{HTML}{60C6C8}
    \definecolor{ansi-cyan-intense}{HTML}{258F8F}
    \definecolor{ansi-white}{HTML}{C5C1B4}
    \definecolor{ansi-white-intense}{HTML}{A1A6B2}
    \definecolor{ansi-default-inverse-fg}{HTML}{FFFFFF}
    \definecolor{ansi-default-inverse-bg}{HTML}{000000}

    % commands and environments needed by pandoc snippets
    % extracted from the output of `pandoc -s`
    \providecommand{\tightlist}{%
      \setlength{\itemsep}{0pt}\setlength{\parskip}{0pt}}
    \DefineVerbatimEnvironment{Highlighting}{Verbatim}{commandchars=\\\{\}}
    % Add ',fontsize=\small' for more characters per line
    \newenvironment{Shaded}{}{}
    \newcommand{\KeywordTok}[1]{\textcolor[rgb]{0.00,0.44,0.13}{\textbf{{#1}}}}
    \newcommand{\DataTypeTok}[1]{\textcolor[rgb]{0.56,0.13,0.00}{{#1}}}
    \newcommand{\DecValTok}[1]{\textcolor[rgb]{0.25,0.63,0.44}{{#1}}}
    \newcommand{\BaseNTok}[1]{\textcolor[rgb]{0.25,0.63,0.44}{{#1}}}
    \newcommand{\FloatTok}[1]{\textcolor[rgb]{0.25,0.63,0.44}{{#1}}}
    \newcommand{\CharTok}[1]{\textcolor[rgb]{0.25,0.44,0.63}{{#1}}}
    \newcommand{\StringTok}[1]{\textcolor[rgb]{0.25,0.44,0.63}{{#1}}}
    \newcommand{\CommentTok}[1]{\textcolor[rgb]{0.38,0.63,0.69}{\textit{{#1}}}}
    \newcommand{\OtherTok}[1]{\textcolor[rgb]{0.00,0.44,0.13}{{#1}}}
    \newcommand{\AlertTok}[1]{\textcolor[rgb]{1.00,0.00,0.00}{\textbf{{#1}}}}
    \newcommand{\FunctionTok}[1]{\textcolor[rgb]{0.02,0.16,0.49}{{#1}}}
    \newcommand{\RegionMarkerTok}[1]{{#1}}
    \newcommand{\ErrorTok}[1]{\textcolor[rgb]{1.00,0.00,0.00}{\textbf{{#1}}}}
    \newcommand{\NormalTok}[1]{{#1}}
    
    % Additional commands for more recent versions of Pandoc
    \newcommand{\ConstantTok}[1]{\textcolor[rgb]{0.53,0.00,0.00}{{#1}}}
    \newcommand{\SpecialCharTok}[1]{\textcolor[rgb]{0.25,0.44,0.63}{{#1}}}
    \newcommand{\VerbatimStringTok}[1]{\textcolor[rgb]{0.25,0.44,0.63}{{#1}}}
    \newcommand{\SpecialStringTok}[1]{\textcolor[rgb]{0.73,0.40,0.53}{{#1}}}
    \newcommand{\ImportTok}[1]{{#1}}
    \newcommand{\DocumentationTok}[1]{\textcolor[rgb]{0.73,0.13,0.13}{\textit{{#1}}}}
    \newcommand{\AnnotationTok}[1]{\textcolor[rgb]{0.38,0.63,0.69}{\textbf{\textit{{#1}}}}}
    \newcommand{\CommentVarTok}[1]{\textcolor[rgb]{0.38,0.63,0.69}{\textbf{\textit{{#1}}}}}
    \newcommand{\VariableTok}[1]{\textcolor[rgb]{0.10,0.09,0.49}{{#1}}}
    \newcommand{\ControlFlowTok}[1]{\textcolor[rgb]{0.00,0.44,0.13}{\textbf{{#1}}}}
    \newcommand{\OperatorTok}[1]{\textcolor[rgb]{0.40,0.40,0.40}{{#1}}}
    \newcommand{\BuiltInTok}[1]{{#1}}
    \newcommand{\ExtensionTok}[1]{{#1}}
    \newcommand{\PreprocessorTok}[1]{\textcolor[rgb]{0.74,0.48,0.00}{{#1}}}
    \newcommand{\AttributeTok}[1]{\textcolor[rgb]{0.49,0.56,0.16}{{#1}}}
    \newcommand{\InformationTok}[1]{\textcolor[rgb]{0.38,0.63,0.69}{\textbf{\textit{{#1}}}}}
    \newcommand{\WarningTok}[1]{\textcolor[rgb]{0.38,0.63,0.69}{\textbf{\textit{{#1}}}}}
    
    
    % Define a nice break command that doesn't care if a line doesn't already
    % exist.
    \def\br{\hspace*{\fill} \\* }
    % Math Jax compatibility definitions
    \def\gt{>}
    \def\lt{<}
    \let\Oldtex\TeX
    \let\Oldlatex\LaTeX
    \renewcommand{\TeX}{\textrm{\Oldtex}}
    \renewcommand{\LaTeX}{\textrm{\Oldlatex}}
    % Document parameters
    % Document title
    \title{DataAnalysisLearning-Data Cleaning and Preparation}
    
    
    
    
    

    % Pygments definitions
    
\makeatletter
\def\PY@reset{\let\PY@it=\relax \let\PY@bf=\relax%
    \let\PY@ul=\relax \let\PY@tc=\relax%
    \let\PY@bc=\relax \let\PY@ff=\relax}
\def\PY@tok#1{\csname PY@tok@#1\endcsname}
\def\PY@toks#1+{\ifx\relax#1\empty\else%
    \PY@tok{#1}\expandafter\PY@toks\fi}
\def\PY@do#1{\PY@bc{\PY@tc{\PY@ul{%
    \PY@it{\PY@bf{\PY@ff{#1}}}}}}}
\def\PY#1#2{\PY@reset\PY@toks#1+\relax+\PY@do{#2}}

\expandafter\def\csname PY@tok@w\endcsname{\def\PY@tc##1{\textcolor[rgb]{0.73,0.73,0.73}{##1}}}
\expandafter\def\csname PY@tok@c\endcsname{\let\PY@it=\textit\def\PY@tc##1{\textcolor[rgb]{0.25,0.50,0.50}{##1}}}
\expandafter\def\csname PY@tok@cp\endcsname{\def\PY@tc##1{\textcolor[rgb]{0.74,0.48,0.00}{##1}}}
\expandafter\def\csname PY@tok@k\endcsname{\let\PY@bf=\textbf\def\PY@tc##1{\textcolor[rgb]{0.00,0.50,0.00}{##1}}}
\expandafter\def\csname PY@tok@kp\endcsname{\def\PY@tc##1{\textcolor[rgb]{0.00,0.50,0.00}{##1}}}
\expandafter\def\csname PY@tok@kt\endcsname{\def\PY@tc##1{\textcolor[rgb]{0.69,0.00,0.25}{##1}}}
\expandafter\def\csname PY@tok@o\endcsname{\def\PY@tc##1{\textcolor[rgb]{0.40,0.40,0.40}{##1}}}
\expandafter\def\csname PY@tok@ow\endcsname{\let\PY@bf=\textbf\def\PY@tc##1{\textcolor[rgb]{0.67,0.13,1.00}{##1}}}
\expandafter\def\csname PY@tok@nb\endcsname{\def\PY@tc##1{\textcolor[rgb]{0.00,0.50,0.00}{##1}}}
\expandafter\def\csname PY@tok@nf\endcsname{\def\PY@tc##1{\textcolor[rgb]{0.00,0.00,1.00}{##1}}}
\expandafter\def\csname PY@tok@nc\endcsname{\let\PY@bf=\textbf\def\PY@tc##1{\textcolor[rgb]{0.00,0.00,1.00}{##1}}}
\expandafter\def\csname PY@tok@nn\endcsname{\let\PY@bf=\textbf\def\PY@tc##1{\textcolor[rgb]{0.00,0.00,1.00}{##1}}}
\expandafter\def\csname PY@tok@ne\endcsname{\let\PY@bf=\textbf\def\PY@tc##1{\textcolor[rgb]{0.82,0.25,0.23}{##1}}}
\expandafter\def\csname PY@tok@nv\endcsname{\def\PY@tc##1{\textcolor[rgb]{0.10,0.09,0.49}{##1}}}
\expandafter\def\csname PY@tok@no\endcsname{\def\PY@tc##1{\textcolor[rgb]{0.53,0.00,0.00}{##1}}}
\expandafter\def\csname PY@tok@nl\endcsname{\def\PY@tc##1{\textcolor[rgb]{0.63,0.63,0.00}{##1}}}
\expandafter\def\csname PY@tok@ni\endcsname{\let\PY@bf=\textbf\def\PY@tc##1{\textcolor[rgb]{0.60,0.60,0.60}{##1}}}
\expandafter\def\csname PY@tok@na\endcsname{\def\PY@tc##1{\textcolor[rgb]{0.49,0.56,0.16}{##1}}}
\expandafter\def\csname PY@tok@nt\endcsname{\let\PY@bf=\textbf\def\PY@tc##1{\textcolor[rgb]{0.00,0.50,0.00}{##1}}}
\expandafter\def\csname PY@tok@nd\endcsname{\def\PY@tc##1{\textcolor[rgb]{0.67,0.13,1.00}{##1}}}
\expandafter\def\csname PY@tok@s\endcsname{\def\PY@tc##1{\textcolor[rgb]{0.73,0.13,0.13}{##1}}}
\expandafter\def\csname PY@tok@sd\endcsname{\let\PY@it=\textit\def\PY@tc##1{\textcolor[rgb]{0.73,0.13,0.13}{##1}}}
\expandafter\def\csname PY@tok@si\endcsname{\let\PY@bf=\textbf\def\PY@tc##1{\textcolor[rgb]{0.73,0.40,0.53}{##1}}}
\expandafter\def\csname PY@tok@se\endcsname{\let\PY@bf=\textbf\def\PY@tc##1{\textcolor[rgb]{0.73,0.40,0.13}{##1}}}
\expandafter\def\csname PY@tok@sr\endcsname{\def\PY@tc##1{\textcolor[rgb]{0.73,0.40,0.53}{##1}}}
\expandafter\def\csname PY@tok@ss\endcsname{\def\PY@tc##1{\textcolor[rgb]{0.10,0.09,0.49}{##1}}}
\expandafter\def\csname PY@tok@sx\endcsname{\def\PY@tc##1{\textcolor[rgb]{0.00,0.50,0.00}{##1}}}
\expandafter\def\csname PY@tok@m\endcsname{\def\PY@tc##1{\textcolor[rgb]{0.40,0.40,0.40}{##1}}}
\expandafter\def\csname PY@tok@gh\endcsname{\let\PY@bf=\textbf\def\PY@tc##1{\textcolor[rgb]{0.00,0.00,0.50}{##1}}}
\expandafter\def\csname PY@tok@gu\endcsname{\let\PY@bf=\textbf\def\PY@tc##1{\textcolor[rgb]{0.50,0.00,0.50}{##1}}}
\expandafter\def\csname PY@tok@gd\endcsname{\def\PY@tc##1{\textcolor[rgb]{0.63,0.00,0.00}{##1}}}
\expandafter\def\csname PY@tok@gi\endcsname{\def\PY@tc##1{\textcolor[rgb]{0.00,0.63,0.00}{##1}}}
\expandafter\def\csname PY@tok@gr\endcsname{\def\PY@tc##1{\textcolor[rgb]{1.00,0.00,0.00}{##1}}}
\expandafter\def\csname PY@tok@ge\endcsname{\let\PY@it=\textit}
\expandafter\def\csname PY@tok@gs\endcsname{\let\PY@bf=\textbf}
\expandafter\def\csname PY@tok@gp\endcsname{\let\PY@bf=\textbf\def\PY@tc##1{\textcolor[rgb]{0.00,0.00,0.50}{##1}}}
\expandafter\def\csname PY@tok@go\endcsname{\def\PY@tc##1{\textcolor[rgb]{0.53,0.53,0.53}{##1}}}
\expandafter\def\csname PY@tok@gt\endcsname{\def\PY@tc##1{\textcolor[rgb]{0.00,0.27,0.87}{##1}}}
\expandafter\def\csname PY@tok@err\endcsname{\def\PY@bc##1{\setlength{\fboxsep}{0pt}\fcolorbox[rgb]{1.00,0.00,0.00}{1,1,1}{\strut ##1}}}
\expandafter\def\csname PY@tok@kc\endcsname{\let\PY@bf=\textbf\def\PY@tc##1{\textcolor[rgb]{0.00,0.50,0.00}{##1}}}
\expandafter\def\csname PY@tok@kd\endcsname{\let\PY@bf=\textbf\def\PY@tc##1{\textcolor[rgb]{0.00,0.50,0.00}{##1}}}
\expandafter\def\csname PY@tok@kn\endcsname{\let\PY@bf=\textbf\def\PY@tc##1{\textcolor[rgb]{0.00,0.50,0.00}{##1}}}
\expandafter\def\csname PY@tok@kr\endcsname{\let\PY@bf=\textbf\def\PY@tc##1{\textcolor[rgb]{0.00,0.50,0.00}{##1}}}
\expandafter\def\csname PY@tok@bp\endcsname{\def\PY@tc##1{\textcolor[rgb]{0.00,0.50,0.00}{##1}}}
\expandafter\def\csname PY@tok@fm\endcsname{\def\PY@tc##1{\textcolor[rgb]{0.00,0.00,1.00}{##1}}}
\expandafter\def\csname PY@tok@vc\endcsname{\def\PY@tc##1{\textcolor[rgb]{0.10,0.09,0.49}{##1}}}
\expandafter\def\csname PY@tok@vg\endcsname{\def\PY@tc##1{\textcolor[rgb]{0.10,0.09,0.49}{##1}}}
\expandafter\def\csname PY@tok@vi\endcsname{\def\PY@tc##1{\textcolor[rgb]{0.10,0.09,0.49}{##1}}}
\expandafter\def\csname PY@tok@vm\endcsname{\def\PY@tc##1{\textcolor[rgb]{0.10,0.09,0.49}{##1}}}
\expandafter\def\csname PY@tok@sa\endcsname{\def\PY@tc##1{\textcolor[rgb]{0.73,0.13,0.13}{##1}}}
\expandafter\def\csname PY@tok@sb\endcsname{\def\PY@tc##1{\textcolor[rgb]{0.73,0.13,0.13}{##1}}}
\expandafter\def\csname PY@tok@sc\endcsname{\def\PY@tc##1{\textcolor[rgb]{0.73,0.13,0.13}{##1}}}
\expandafter\def\csname PY@tok@dl\endcsname{\def\PY@tc##1{\textcolor[rgb]{0.73,0.13,0.13}{##1}}}
\expandafter\def\csname PY@tok@s2\endcsname{\def\PY@tc##1{\textcolor[rgb]{0.73,0.13,0.13}{##1}}}
\expandafter\def\csname PY@tok@sh\endcsname{\def\PY@tc##1{\textcolor[rgb]{0.73,0.13,0.13}{##1}}}
\expandafter\def\csname PY@tok@s1\endcsname{\def\PY@tc##1{\textcolor[rgb]{0.73,0.13,0.13}{##1}}}
\expandafter\def\csname PY@tok@mb\endcsname{\def\PY@tc##1{\textcolor[rgb]{0.40,0.40,0.40}{##1}}}
\expandafter\def\csname PY@tok@mf\endcsname{\def\PY@tc##1{\textcolor[rgb]{0.40,0.40,0.40}{##1}}}
\expandafter\def\csname PY@tok@mh\endcsname{\def\PY@tc##1{\textcolor[rgb]{0.40,0.40,0.40}{##1}}}
\expandafter\def\csname PY@tok@mi\endcsname{\def\PY@tc##1{\textcolor[rgb]{0.40,0.40,0.40}{##1}}}
\expandafter\def\csname PY@tok@il\endcsname{\def\PY@tc##1{\textcolor[rgb]{0.40,0.40,0.40}{##1}}}
\expandafter\def\csname PY@tok@mo\endcsname{\def\PY@tc##1{\textcolor[rgb]{0.40,0.40,0.40}{##1}}}
\expandafter\def\csname PY@tok@ch\endcsname{\let\PY@it=\textit\def\PY@tc##1{\textcolor[rgb]{0.25,0.50,0.50}{##1}}}
\expandafter\def\csname PY@tok@cm\endcsname{\let\PY@it=\textit\def\PY@tc##1{\textcolor[rgb]{0.25,0.50,0.50}{##1}}}
\expandafter\def\csname PY@tok@cpf\endcsname{\let\PY@it=\textit\def\PY@tc##1{\textcolor[rgb]{0.25,0.50,0.50}{##1}}}
\expandafter\def\csname PY@tok@c1\endcsname{\let\PY@it=\textit\def\PY@tc##1{\textcolor[rgb]{0.25,0.50,0.50}{##1}}}
\expandafter\def\csname PY@tok@cs\endcsname{\let\PY@it=\textit\def\PY@tc##1{\textcolor[rgb]{0.25,0.50,0.50}{##1}}}

\def\PYZbs{\char`\\}
\def\PYZus{\char`\_}
\def\PYZob{\char`\{}
\def\PYZcb{\char`\}}
\def\PYZca{\char`\^}
\def\PYZam{\char`\&}
\def\PYZlt{\char`\<}
\def\PYZgt{\char`\>}
\def\PYZsh{\char`\#}
\def\PYZpc{\char`\%}
\def\PYZdl{\char`\$}
\def\PYZhy{\char`\-}
\def\PYZsq{\char`\'}
\def\PYZdq{\char`\"}
\def\PYZti{\char`\~}
% for compatibility with earlier versions
\def\PYZat{@}
\def\PYZlb{[}
\def\PYZrb{]}
\makeatother


    % Exact colors from NB
    \definecolor{incolor}{rgb}{0.0, 0.0, 0.5}
    \definecolor{outcolor}{rgb}{0.545, 0.0, 0.0}



    
    % Prevent overflowing lines due to hard-to-break entities
    \sloppy 
    % Setup hyperref package
    \hypersetup{
      breaklinks=true,  % so long urls are correctly broken across lines
      colorlinks=true,
      urlcolor=urlcolor,
      linkcolor=linkcolor,
      citecolor=citecolor,
      }
    % Slightly bigger margins than the latex defaults
    
    \geometry{verbose,tmargin=1in,bmargin=1in,lmargin=1in,rmargin=1in}
    
    

    \begin{document}
    
    
    \maketitle
    
    

    
    \begin{Verbatim}[commandchars=\\\{\}]
{\color{incolor}In [{\color{incolor}1}]:} \PY{c+c1}{\PYZsh{} \PYZhy{}*\PYZhy{}* codig: utf\PYZhy{}8 \PYZhy{}*\PYZhy{}}
        \PY{c+c1}{\PYZsh{} @author: tongzi}
        \PY{c+c1}{\PYZsh{} @description: Data Cleanning and Preparation}
        \PY{c+c1}{\PYZsh{} @created date: 2019/07/04}
        \PY{c+c1}{\PYZsh{} @license}
\end{Verbatim}

    \begin{Verbatim}[commandchars=\\\{\}]
{\color{incolor}In [{\color{incolor}2}]:} \PY{k+kn}{import} \PY{n+nn}{numpy} \PY{k}{as} \PY{n+nn}{np}
        \PY{k+kn}{import} \PY{n+nn}{pandas} \PY{k}{as} \PY{n+nn}{pd}
        \PY{k+kn}{import} \PY{n+nn}{matplotlib}\PY{n+nn}{.}\PY{n+nn}{pyplot} \PY{k}{as} \PY{n+nn}{plt}
        \PY{k+kn}{import} \PY{n+nn}{seaborn} \PY{k}{as} \PY{n+nn}{sns}
        \PY{o}{\PYZpc{}}\PY{k}{matplotlib} inline
\end{Verbatim}

    \begin{Verbatim}[commandchars=\\\{\}]
C:\textbackslash{}Anaconda3\textbackslash{}lib\textbackslash{}importlib\textbackslash{}\_bootstrap.py:219: RuntimeWarning: numpy.ufunc size changed, may indicate binary incompatibility. Expected 192 from C header, got 216 from PyObject
  return f(*args, **kwds)
C:\textbackslash{}Anaconda3\textbackslash{}lib\textbackslash{}importlib\textbackslash{}\_bootstrap.py:219: RuntimeWarning: numpy.ufunc size changed, may indicate binary incompatibility. Expected 192 from C header, got 216 from PyObject
  return f(*args, **kwds)
C:\textbackslash{}Anaconda3\textbackslash{}lib\textbackslash{}importlib\textbackslash{}\_bootstrap.py:219: RuntimeWarning: numpy.ufunc size changed, may indicate binary incompatibility. Expected 192 from C header, got 216 from PyObject
  return f(*args, **kwds)
C:\textbackslash{}Anaconda3\textbackslash{}lib\textbackslash{}importlib\textbackslash{}\_bootstrap.py:219: RuntimeWarning: numpy.ufunc size changed, may indicate binary incompatibility. Expected 216, got 192
  return f(*args, **kwds)
C:\textbackslash{}Anaconda3\textbackslash{}lib\textbackslash{}importlib\textbackslash{}\_bootstrap.py:219: RuntimeWarning: numpy.ufunc size changed, may indicate binary incompatibility. Expected 192 from C header, got 216 from PyObject
  return f(*args, **kwds)

    \end{Verbatim}

    \hypertarget{in-this-chapter-we-will-discuss-tools-for-missing-data-duplicate-data-string-manipulation-and-some-other-analytical-data-transformations.}{%
\paragraph{In this chapter, we will discuss tools for missing data,
duplicate data, string manipulation, and some other analytical data
transformations.}\label{in-this-chapter-we-will-discuss-tools-for-missing-data-duplicate-data-string-manipulation-and-some-other-analytical-data-transformations.}}

    \hypertarget{handling-missing-data}{%
\subsubsection{Handling Missing Data}\label{handling-missing-data}}

    Missing data occurs commonly in many data analysis applications. One of
the goal of pandas is to make working with missing data as painless as
possible. For example, all of the descriptive statistic on pandas
objects exclude missing data by default.\\
For numeric data, pandas uses the float-point value \textbf{NaN} (Not a
Number) to represent missing data. We call this a sentinel value that
can be easily detected:

    \begin{Verbatim}[commandchars=\\\{\}]
{\color{incolor}In [{\color{incolor}3}]:} \PY{n}{string\PYZus{}data} \PY{o}{=} \PY{n}{pd}\PY{o}{.}\PY{n}{Series}\PY{p}{(}\PY{p}{[}\PY{l+s+s1}{\PYZsq{}}\PY{l+s+s1}{aardvark}\PY{l+s+s1}{\PYZsq{}}\PY{p}{,} \PY{l+s+s1}{\PYZsq{}}\PY{l+s+s1}{artichoke}\PY{l+s+s1}{\PYZsq{}}\PY{p}{,} \PY{n}{np}\PY{o}{.}\PY{n}{nan}\PY{p}{,} \PY{l+s+s1}{\PYZsq{}}\PY{l+s+s1}{avocado}\PY{l+s+s1}{\PYZsq{}}\PY{p}{]}\PY{p}{)}
\end{Verbatim}

    \begin{Verbatim}[commandchars=\\\{\}]
{\color{incolor}In [{\color{incolor}4}]:} \PY{n}{string\PYZus{}data}
\end{Verbatim}

\begin{Verbatim}[commandchars=\\\{\}]
{\color{outcolor}Out[{\color{outcolor}4}]:} 0     aardvark
        1    artichoke
        2          NaN
        3      avocado
        dtype: object
\end{Verbatim}
            
    \begin{Verbatim}[commandchars=\\\{\}]
{\color{incolor}In [{\color{incolor}5}]:} \PY{n}{string\PYZus{}data}\PY{o}{.}\PY{n}{isnull}\PY{p}{(}\PY{p}{)}
\end{Verbatim}

\begin{Verbatim}[commandchars=\\\{\}]
{\color{outcolor}Out[{\color{outcolor}5}]:} 0    False
        1    False
        2     True
        3    False
        dtype: bool
\end{Verbatim}
            
    Below is a table that lists some functions related to missing data
handling:\\
\includegraphics{attachment:image.png}

    \hypertarget{filtering-out-missing-data}{%
\subsubsection{Filtering Out Missing
Data}\label{filtering-out-missing-data}}

    \begin{Verbatim}[commandchars=\\\{\}]
{\color{incolor}In [{\color{incolor}6}]:} \PY{k+kn}{from} \PY{n+nn}{numpy} \PY{k}{import} \PY{n}{nan} \PY{k}{as} \PY{n}{NA}
\end{Verbatim}

    \begin{Verbatim}[commandchars=\\\{\}]
{\color{incolor}In [{\color{incolor}7}]:} \PY{n}{data} \PY{o}{=} \PY{n}{pd}\PY{o}{.}\PY{n}{Series}\PY{p}{(}\PY{p}{[}\PY{l+m+mi}{1}\PY{p}{,} \PY{n}{NA}\PY{p}{,} \PY{l+m+mf}{3.5}\PY{p}{,} \PY{l+m+mi}{7}\PY{p}{]}\PY{p}{)}
\end{Verbatim}

    \begin{Verbatim}[commandchars=\\\{\}]
{\color{incolor}In [{\color{incolor}8}]:} \PY{n}{data}\PY{o}{.}\PY{n}{dropna}\PY{p}{(}\PY{p}{)}
\end{Verbatim}

\begin{Verbatim}[commandchars=\\\{\}]
{\color{outcolor}Out[{\color{outcolor}8}]:} 0    1.0
        2    3.5
        3    7.0
        dtype: float64
\end{Verbatim}
            
    As we can see above, the \emph{dropna}() method returns the Series with
only the non-null data and index values. This is equivalent to:

    \begin{Verbatim}[commandchars=\\\{\}]
{\color{incolor}In [{\color{incolor}9}]:} \PY{n}{data}\PY{p}{[}\PY{n}{data}\PY{o}{.}\PY{n}{notnull}\PY{p}{(}\PY{p}{)}\PY{p}{]}
\end{Verbatim}

\begin{Verbatim}[commandchars=\\\{\}]
{\color{outcolor}Out[{\color{outcolor}9}]:} 0    1.0
        2    3.5
        3    7.0
        dtype: float64
\end{Verbatim}
            
    With DataFrame, things are a bit more different, we may want to drop
rows or columns that are all NA or only those containing any NAs.
\emph{dropna}() method by default drops any row containing a missing
value:

    \begin{Verbatim}[commandchars=\\\{\}]
{\color{incolor}In [{\color{incolor}10}]:} \PY{n}{data} \PY{o}{=} \PY{n}{pd}\PY{o}{.}\PY{n}{DataFrame}\PY{p}{(}\PY{p}{[}\PY{p}{[}\PY{l+m+mf}{1.}\PY{p}{,} \PY{l+m+mf}{6.5}\PY{p}{,} \PY{l+m+mf}{3.}\PY{p}{]}\PY{p}{,} \PY{p}{[}\PY{l+m+mf}{1.}\PY{p}{,} \PY{n}{NA}\PY{p}{,} \PY{n}{NA}\PY{p}{]}\PY{p}{,}
         \PY{o}{.}\PY{o}{.}\PY{o}{.}\PY{o}{.}\PY{p}{:} \PY{p}{[}\PY{n}{NA}\PY{p}{,} \PY{n}{NA}\PY{p}{,} \PY{n}{NA}\PY{p}{]}\PY{p}{,} \PY{p}{[}\PY{n}{NA}\PY{p}{,} \PY{l+m+mf}{6.5}\PY{p}{,} \PY{l+m+mf}{3.}\PY{p}{]}\PY{p}{]}\PY{p}{)}
\end{Verbatim}

    \begin{Verbatim}[commandchars=\\\{\}]
{\color{incolor}In [{\color{incolor}11}]:} \PY{n}{data}
\end{Verbatim}

\begin{Verbatim}[commandchars=\\\{\}]
{\color{outcolor}Out[{\color{outcolor}11}]:}      0    1    2
         0  1.0  6.5  3.0
         1  1.0  NaN  NaN
         2  NaN  NaN  NaN
         3  NaN  6.5  3.0
\end{Verbatim}
            
    \begin{Verbatim}[commandchars=\\\{\}]
{\color{incolor}In [{\color{incolor}12}]:} \PY{n}{data}\PY{o}{.}\PY{n}{dropna}\PY{p}{(}\PY{p}{)}
\end{Verbatim}

\begin{Verbatim}[commandchars=\\\{\}]
{\color{outcolor}Out[{\color{outcolor}12}]:}      0    1    2
         0  1.0  6.5  3.0
\end{Verbatim}
            
    \begin{quote}
从上述的例子可以看出,对于DataFrame,\emph{dropna}()默认会将至少包含一个缺失值的行删除掉。
\end{quote}

    \begin{Verbatim}[commandchars=\\\{\}]
{\color{incolor}In [{\color{incolor}15}]:} \PY{c+c1}{\PYZsh{} 因为每一列都至少包含一个缺失值,所以全部删除}
         \PY{n}{data}\PY{o}{.}\PY{n}{dropna}\PY{p}{(}\PY{n}{axis}\PY{o}{=}\PY{l+s+s1}{\PYZsq{}}\PY{l+s+s1}{columns}\PY{l+s+s1}{\PYZsq{}}\PY{p}{)}
\end{Verbatim}

\begin{Verbatim}[commandchars=\\\{\}]
{\color{outcolor}Out[{\color{outcolor}15}]:} Empty DataFrame
         Columns: []
         Index: [0, 1, 2, 3]
\end{Verbatim}
            
    \begin{Verbatim}[commandchars=\\\{\}]
{\color{incolor}In [{\color{incolor}16}]:} \PY{n}{data}
\end{Verbatim}

\begin{Verbatim}[commandchars=\\\{\}]
{\color{outcolor}Out[{\color{outcolor}16}]:}      0    1    2
         0  1.0  6.5  3.0
         1  1.0  NaN  NaN
         2  NaN  NaN  NaN
         3  NaN  6.5  3.0
\end{Verbatim}
            
    Passing \emph{how}=`all', will only drop rows that are all NA:

    \begin{Verbatim}[commandchars=\\\{\}]
{\color{incolor}In [{\color{incolor}17}]:} \PY{n}{data}\PY{o}{.}\PY{n}{dropna}\PY{p}{(}\PY{n}{how}\PY{o}{=}\PY{l+s+s1}{\PYZsq{}}\PY{l+s+s1}{all}\PY{l+s+s1}{\PYZsq{}}\PY{p}{)}
\end{Verbatim}

\begin{Verbatim}[commandchars=\\\{\}]
{\color{outcolor}Out[{\color{outcolor}17}]:}      0    1    2
         0  1.0  6.5  3.0
         1  1.0  NaN  NaN
         3  NaN  6.5  3.0
\end{Verbatim}
            
    \begin{Verbatim}[commandchars=\\\{\}]
{\color{incolor}In [{\color{incolor}18}]:} \PY{c+c1}{\PYZsh{} 增加一列}
         \PY{n}{data}\PY{p}{[}\PY{l+m+mi}{4}\PY{p}{]} \PY{o}{=} \PY{n}{NA}
\end{Verbatim}

    \begin{Verbatim}[commandchars=\\\{\}]
{\color{incolor}In [{\color{incolor}19}]:} \PY{n}{data}
\end{Verbatim}

\begin{Verbatim}[commandchars=\\\{\}]
{\color{outcolor}Out[{\color{outcolor}19}]:}      0    1    2   4
         0  1.0  6.5  3.0 NaN
         1  1.0  NaN  NaN NaN
         2  NaN  NaN  NaN NaN
         3  NaN  6.5  3.0 NaN
\end{Verbatim}
            
    \begin{Verbatim}[commandchars=\\\{\}]
{\color{incolor}In [{\color{incolor}20}]:} \PY{n}{data}\PY{o}{.}\PY{n}{dropna}\PY{p}{(}\PY{n}{axis}\PY{o}{=}\PY{l+s+s1}{\PYZsq{}}\PY{l+s+s1}{columns}\PY{l+s+s1}{\PYZsq{}}\PY{p}{,} \PY{n}{how}\PY{o}{=}\PY{l+s+s1}{\PYZsq{}}\PY{l+s+s1}{all}\PY{l+s+s1}{\PYZsq{}}\PY{p}{)}
\end{Verbatim}

\begin{Verbatim}[commandchars=\\\{\}]
{\color{outcolor}Out[{\color{outcolor}20}]:}      0    1    2
         0  1.0  6.5  3.0
         1  1.0  NaN  NaN
         2  NaN  NaN  NaN
         3  NaN  6.5  3.0
\end{Verbatim}
            
    另外一个过滤DataFrame中行数据的主题是关于时间序列的,假设我们只想保留一定数量的观测值,可以通过指定\emph{thresh}参数:

    \begin{Verbatim}[commandchars=\\\{\}]
{\color{incolor}In [{\color{incolor}21}]:} \PY{n}{df} \PY{o}{=} \PY{n}{pd}\PY{o}{.}\PY{n}{DataFrame}\PY{p}{(}\PY{n}{np}\PY{o}{.}\PY{n}{random}\PY{o}{.}\PY{n}{randn}\PY{p}{(}\PY{l+m+mi}{7}\PY{p}{,} \PY{l+m+mi}{3}\PY{p}{)}\PY{p}{)}
\end{Verbatim}

    \begin{Verbatim}[commandchars=\\\{\}]
{\color{incolor}In [{\color{incolor}22}]:} \PY{n}{df}
\end{Verbatim}

\begin{Verbatim}[commandchars=\\\{\}]
{\color{outcolor}Out[{\color{outcolor}22}]:}           0         1         2
         0 -0.377788 -0.099363  0.773704
         1 -0.807716 -0.697565  0.309078
         2 -1.002575 -0.002337  0.970096
         3  1.768859 -0.646667  0.167954
         4  0.687819  3.742656 -0.482712
         5 -0.035491  0.267234  1.161451
         6 -0.654256 -0.612303  2.564999
\end{Verbatim}
            
    \begin{Verbatim}[commandchars=\\\{\}]
{\color{incolor}In [{\color{incolor}23}]:} \PY{c+c1}{\PYZsh{} 将前4行,第1列置为NaN}
         \PY{n}{df}\PY{o}{.}\PY{n}{iloc}\PY{p}{[}\PY{p}{:}\PY{l+m+mi}{4}\PY{p}{,} \PY{l+m+mi}{1}\PY{p}{]} \PY{o}{=} \PY{n}{NA}
\end{Verbatim}

    \begin{Verbatim}[commandchars=\\\{\}]
{\color{incolor}In [{\color{incolor}26}]:} \PY{c+c1}{\PYZsh{} 将前2行,第2列置为NaN}
         \PY{n}{df}\PY{o}{.}\PY{n}{iloc}\PY{p}{[}\PY{p}{:}\PY{l+m+mi}{2}\PY{p}{,} \PY{l+m+mi}{2}\PY{p}{]} \PY{o}{=} \PY{n}{NA}
\end{Verbatim}

    \begin{Verbatim}[commandchars=\\\{\}]
{\color{incolor}In [{\color{incolor}27}]:} \PY{n}{df}
\end{Verbatim}

\begin{Verbatim}[commandchars=\\\{\}]
{\color{outcolor}Out[{\color{outcolor}27}]:}           0         1         2
         0 -0.377788       NaN       NaN
         1 -0.807716       NaN       NaN
         2 -1.002575       NaN  0.970096
         3  1.768859       NaN  0.167954
         4  0.687819  3.742656 -0.482712
         5 -0.035491  0.267234  1.161451
         6 -0.654256 -0.612303  2.564999
\end{Verbatim}
            
    \begin{Verbatim}[commandchars=\\\{\}]
{\color{incolor}In [{\color{incolor}28}]:} \PY{c+c1}{\PYZsh{} 这样调用会删除所有包含缺失值(NaN)的行}
         \PY{n}{df}\PY{o}{.}\PY{n}{dropna}\PY{p}{(}\PY{p}{)}
\end{Verbatim}

\begin{Verbatim}[commandchars=\\\{\}]
{\color{outcolor}Out[{\color{outcolor}28}]:}           0         1         2
         4  0.687819  3.742656 -0.482712
         5 -0.035491  0.267234  1.161451
         6 -0.654256 -0.612303  2.564999
\end{Verbatim}
            
    \begin{Verbatim}[commandchars=\\\{\}]
{\color{incolor}In [{\color{incolor}29}]:} \PY{c+c1}{\PYZsh{} 通过指定thresh参数,可以指定只删除}
         \PY{c+c1}{\PYZsh{} 一定数量的包含缺失值的行}
         \PY{n}{df}\PY{o}{.}\PY{n}{dropna}\PY{p}{(}\PY{n}{thresh}\PY{o}{=}\PY{l+m+mi}{2}\PY{p}{)}
\end{Verbatim}

\begin{Verbatim}[commandchars=\\\{\}]
{\color{outcolor}Out[{\color{outcolor}29}]:}           0         1         2
         2 -1.002575       NaN  0.970096
         3  1.768859       NaN  0.167954
         4  0.687819  3.742656 -0.482712
         5 -0.035491  0.267234  1.161451
         6 -0.654256 -0.612303  2.564999
\end{Verbatim}
            
    \hypertarget{filling-in-missing-data}{%
\subsubsection{Filling In Missing Data}\label{filling-in-missing-data}}

    Rather than filtering out missing data (potentially discarding other
data along with it), we may want to fill in the `hole' in any number of
ways. For most purposes, the \emph{fillna}() method is the workhorse
function to use:

    \begin{Verbatim}[commandchars=\\\{\}]
{\color{incolor}In [{\color{incolor}30}]:} \PY{c+c1}{\PYZsh{} 用0填充所有列的缺失值}
         \PY{n}{df}\PY{o}{.}\PY{n}{fillna}\PY{p}{(}\PY{l+m+mi}{0}\PY{p}{)}
\end{Verbatim}

\begin{Verbatim}[commandchars=\\\{\}]
{\color{outcolor}Out[{\color{outcolor}30}]:}           0         1         2
         0 -0.377788  0.000000  0.000000
         1 -0.807716  0.000000  0.000000
         2 -1.002575  0.000000  0.970096
         3  1.768859  0.000000  0.167954
         4  0.687819  3.742656 -0.482712
         5 -0.035491  0.267234  1.161451
         6 -0.654256 -0.612303  2.564999
\end{Verbatim}
            
    \begin{Verbatim}[commandchars=\\\{\}]
{\color{incolor}In [{\color{incolor}32}]:} \PY{c+c1}{\PYZsh{} 为每一个列指定填充值}
         \PY{c+c1}{\PYZsh{} 第1列用10填充,第2列用200填充}
         \PY{n}{df}\PY{o}{.}\PY{n}{fillna}\PY{p}{(}\PY{p}{\PYZob{}}\PY{l+m+mi}{1}\PY{p}{:}\PY{l+m+mi}{10}\PY{p}{,} \PY{l+m+mi}{2}\PY{p}{:}\PY{l+m+mi}{200}\PY{p}{\PYZcb{}}\PY{p}{)}
\end{Verbatim}

\begin{Verbatim}[commandchars=\\\{\}]
{\color{outcolor}Out[{\color{outcolor}32}]:}           0          1           2
         0 -0.377788  10.000000  200.000000
         1 -0.807716  10.000000  200.000000
         2 -1.002575  10.000000    0.970096
         3  1.768859  10.000000    0.167954
         4  0.687819   3.742656   -0.482712
         5 -0.035491   0.267234    1.161451
         6 -0.654256  -0.612303    2.564999
\end{Verbatim}
            
    \emph{fillna}() returns a new object, but we can modify the existing
object in-place by passing the argument \emph{inplace}:

    \begin{Verbatim}[commandchars=\\\{\}]
{\color{incolor}In [{\color{incolor}33}]:} \PY{n}{df}\PY{o}{.}\PY{n}{fillna}\PY{p}{(}\PY{l+m+mi}{0}\PY{p}{,} \PY{n}{inplace}\PY{o}{=}\PY{k+kc}{True}\PY{p}{)}
\end{Verbatim}

    \begin{Verbatim}[commandchars=\\\{\}]
{\color{incolor}In [{\color{incolor}34}]:} \PY{n}{df}
\end{Verbatim}

\begin{Verbatim}[commandchars=\\\{\}]
{\color{outcolor}Out[{\color{outcolor}34}]:}           0         1         2
         0 -0.377788  0.000000  0.000000
         1 -0.807716  0.000000  0.000000
         2 -1.002575  0.000000  0.970096
         3  1.768859  0.000000  0.167954
         4  0.687819  3.742656 -0.482712
         5 -0.035491  0.267234  1.161451
         6 -0.654256 -0.612303  2.564999
\end{Verbatim}
            
    The same interpolations available for reindexing can be used for
\emph{fillna}():
\textgreater{}在重索引中使用的插值方式也可以用于\emph{fillna}()方法中。

    \begin{Verbatim}[commandchars=\\\{\}]
{\color{incolor}In [{\color{incolor}35}]:} \PY{n}{df} \PY{o}{=} \PY{n}{pd}\PY{o}{.}\PY{n}{DataFrame}\PY{p}{(}\PY{n}{np}\PY{o}{.}\PY{n}{random}\PY{o}{.}\PY{n}{randn}\PY{p}{(}\PY{l+m+mi}{6}\PY{p}{,} \PY{l+m+mi}{3}\PY{p}{)}\PY{p}{)}
\end{Verbatim}

    \begin{Verbatim}[commandchars=\\\{\}]
{\color{incolor}In [{\color{incolor}36}]:} \PY{c+c1}{\PYZsh{} 隐式索引(implicite indexing)}
         \PY{n}{df}\PY{o}{.}\PY{n}{iloc}\PY{p}{[}\PY{l+m+mi}{2}\PY{p}{:}\PY{p}{,} \PY{l+m+mi}{1}\PY{p}{]} \PY{o}{=} \PY{n}{NA} \PY{c+c1}{\PYZsh{} 第2行之后,第1列置为NaN}
\end{Verbatim}

    \begin{Verbatim}[commandchars=\\\{\}]
{\color{incolor}In [{\color{incolor}39}]:} \PY{c+c1}{\PYZsh{} 隐式索引(implicite indexing)}
         \PY{n}{df}\PY{o}{.}\PY{n}{iloc}\PY{p}{[}\PY{l+m+mi}{4}\PY{p}{:}\PY{p}{,} \PY{l+m+mi}{2}\PY{p}{]} \PY{o}{=} \PY{n}{NA} \PY{c+c1}{\PYZsh{} 第4行之后,第2列置为NaN}
\end{Verbatim}

    \begin{Verbatim}[commandchars=\\\{\}]
{\color{incolor}In [{\color{incolor}40}]:} \PY{n}{df}
\end{Verbatim}

\begin{Verbatim}[commandchars=\\\{\}]
{\color{outcolor}Out[{\color{outcolor}40}]:}           0         1         2
         0 -0.018569  0.250788  0.468610
         1 -1.147574  1.921718  1.831329
         2 -1.351726       NaN -0.247973
         3  0.824898       NaN  0.608928
         4 -0.540984       NaN       NaN
         5 -0.748586       NaN       NaN
\end{Verbatim}
            
    \begin{Verbatim}[commandchars=\\\{\}]
{\color{incolor}In [{\color{incolor}41}]:} \PY{c+c1}{\PYZsh{} \PYZsq{}ffill\PYZsq{}表示forward fill,用缺失值前面的数填充它自己}
         \PY{n}{df}\PY{o}{.}\PY{n}{fillna}\PY{p}{(}\PY{n}{method}\PY{o}{=}\PY{l+s+s1}{\PYZsq{}}\PY{l+s+s1}{ffill}\PY{l+s+s1}{\PYZsq{}}\PY{p}{)} 
\end{Verbatim}

\begin{Verbatim}[commandchars=\\\{\}]
{\color{outcolor}Out[{\color{outcolor}41}]:}           0         1         2
         0 -0.018569  0.250788  0.468610
         1 -1.147574  1.921718  1.831329
         2 -1.351726  1.921718 -0.247973
         3  0.824898  1.921718  0.608928
         4 -0.540984  1.921718  0.608928
         5 -0.748586  1.921718  0.608928
\end{Verbatim}
            
    \begin{Verbatim}[commandchars=\\\{\}]
{\color{incolor}In [{\color{incolor}42}]:} \PY{c+c1}{\PYZsh{} \PYZsq{}ffill\PYZsq{}表示forward fill,用缺失值前面的数填充它自己,并}
         \PY{c+c1}{\PYZsh{} 且向后填充的连续个缺失值限制为2个}
         \PY{n}{df}\PY{o}{.}\PY{n}{fillna}\PY{p}{(}\PY{n}{method}\PY{o}{=}\PY{l+s+s1}{\PYZsq{}}\PY{l+s+s1}{ffill}\PY{l+s+s1}{\PYZsq{}}\PY{p}{,} \PY{n}{limit}\PY{o}{=}\PY{l+m+mi}{2}\PY{p}{)}
\end{Verbatim}

\begin{Verbatim}[commandchars=\\\{\}]
{\color{outcolor}Out[{\color{outcolor}42}]:}           0         1         2
         0 -0.018569  0.250788  0.468610
         1 -1.147574  1.921718  1.831329
         2 -1.351726  1.921718 -0.247973
         3  0.824898  1.921718  0.608928
         4 -0.540984       NaN  0.608928
         5 -0.748586       NaN  0.608928
\end{Verbatim}
            
    With the \emph{fillna}(), we can do lots of things with a little
creativity. For example, we might pass the mean or median value of a
Series:

    \begin{Verbatim}[commandchars=\\\{\}]
{\color{incolor}In [{\color{incolor}43}]:} \PY{n}{data} \PY{o}{=} \PY{n}{pd}\PY{o}{.}\PY{n}{Series}\PY{p}{(}\PY{p}{[}\PY{l+m+mf}{1.}\PY{p}{,} \PY{n}{NA}\PY{p}{,} \PY{l+m+mf}{3.5}\PY{p}{,} \PY{n}{NA}\PY{p}{,} \PY{l+m+mi}{7}\PY{p}{]}\PY{p}{)}
\end{Verbatim}

    \begin{Verbatim}[commandchars=\\\{\}]
{\color{incolor}In [{\color{incolor}44}]:} \PY{n}{data}
\end{Verbatim}

\begin{Verbatim}[commandchars=\\\{\}]
{\color{outcolor}Out[{\color{outcolor}44}]:} 0    1.0
         1    NaN
         2    3.5
         3    NaN
         4    7.0
         dtype: float64
\end{Verbatim}
            
    \begin{Verbatim}[commandchars=\\\{\}]
{\color{incolor}In [{\color{incolor}45}]:} \PY{c+c1}{\PYZsh{}用均值(1.0 + 3.5 + 7.0) / 3填充}
         \PY{n}{data}\PY{o}{.}\PY{n}{fillna}\PY{p}{(}\PY{n}{data}\PY{o}{.}\PY{n}{mean}\PY{p}{(}\PY{p}{)}\PY{p}{)}
\end{Verbatim}

\begin{Verbatim}[commandchars=\\\{\}]
{\color{outcolor}Out[{\color{outcolor}45}]:} 0    1.000000
         1    3.833333
         2    3.500000
         3    3.833333
         4    7.000000
         dtype: float64
\end{Verbatim}
            
    \begin{Verbatim}[commandchars=\\\{\}]
{\color{incolor}In [{\color{incolor}48}]:} \PY{p}{(}\PY{l+m+mf}{1.0} \PY{o}{+} \PY{l+m+mf}{3.5} \PY{o}{+} \PY{l+m+mf}{7.0}\PY{p}{)} \PY{o}{/} \PY{l+m+mi}{3}
\end{Verbatim}

\begin{Verbatim}[commandchars=\\\{\}]
{\color{outcolor}Out[{\color{outcolor}48}]:} 3.8333333333333335
\end{Verbatim}
            
    \emph{fillna}()方法的参数说明:\\
\includegraphics{attachment:image.png}

    \hypertarget{data-transformation}{%
\subsubsection{Data Transformation}\label{data-transformation}}

    So far in this chapter, we've been concerned with rearranging data.
Filtering, cleaning and other transformations are another class of
important operations.

    \hypertarget{remobing-duplicates}{%
\paragraph{Remobing Duplicates}\label{remobing-duplicates}}

    Duplicate row may be found in a DataFrame for any number of reasons. For
example:

    \begin{Verbatim}[commandchars=\\\{\}]
{\color{incolor}In [{\color{incolor}49}]:} \PY{n}{data} \PY{o}{=} \PY{n}{pd}\PY{o}{.}\PY{n}{DataFrame}\PY{p}{(}\PY{p}{\PYZob{}}\PY{l+s+s1}{\PYZsq{}}\PY{l+s+s1}{k1}\PY{l+s+s1}{\PYZsq{}}\PY{p}{:} \PY{p}{[}\PY{l+s+s1}{\PYZsq{}}\PY{l+s+s1}{one}\PY{l+s+s1}{\PYZsq{}}\PY{p}{,} \PY{l+s+s1}{\PYZsq{}}\PY{l+s+s1}{two}\PY{l+s+s1}{\PYZsq{}}\PY{p}{]} \PY{o}{*} \PY{l+m+mi}{3} \PY{o}{+} \PY{p}{[}\PY{l+s+s1}{\PYZsq{}}\PY{l+s+s1}{two}\PY{l+s+s1}{\PYZsq{}}\PY{p}{]}\PY{p}{,}
         \PY{o}{.}\PY{o}{.}\PY{o}{.}\PY{o}{.}\PY{p}{:} \PY{l+s+s1}{\PYZsq{}}\PY{l+s+s1}{k2}\PY{l+s+s1}{\PYZsq{}}\PY{p}{:} \PY{p}{[}\PY{l+m+mi}{1}\PY{p}{,} \PY{l+m+mi}{1}\PY{p}{,} \PY{l+m+mi}{2}\PY{p}{,} \PY{l+m+mi}{3}\PY{p}{,} \PY{l+m+mi}{3}\PY{p}{,} \PY{l+m+mi}{4}\PY{p}{,} \PY{l+m+mi}{4}\PY{p}{]}\PY{p}{\PYZcb{}}\PY{p}{)}
\end{Verbatim}

    \begin{Verbatim}[commandchars=\\\{\}]
{\color{incolor}In [{\color{incolor}50}]:} \PY{n}{data}
\end{Verbatim}

\begin{Verbatim}[commandchars=\\\{\}]
{\color{outcolor}Out[{\color{outcolor}50}]:}     k1  k2
         0  one   1
         1  two   1
         2  one   2
         3  two   3
         4  one   3
         5  two   4
         6  two   4
\end{Verbatim}
            
    The DataFrame method \emph{duplicated}() returns a Series indicating
whether each row is a dulicate (has been observed in a previous row) or
not:

    \begin{Verbatim}[commandchars=\\\{\}]
{\color{incolor}In [{\color{incolor}51}]:} \PY{n}{data}\PY{o}{.}\PY{n}{duplicated}\PY{p}{(}\PY{p}{)}
\end{Verbatim}

\begin{Verbatim}[commandchars=\\\{\}]
{\color{outcolor}Out[{\color{outcolor}51}]:} 0    False
         1    False
         2    False
         3    False
         4    False
         5    False
         6     True
         dtype: bool
\end{Verbatim}
            
    Relately, the DataFrame \emph{drop\_duplicates}() returns a DataFrame
where the duplicated array is False:

    \begin{Verbatim}[commandchars=\\\{\}]
{\color{incolor}In [{\color{incolor}52}]:} \PY{n}{data}\PY{o}{.}\PY{n}{drop\PYZus{}duplicates}\PY{p}{(}\PY{p}{)}
\end{Verbatim}

\begin{Verbatim}[commandchars=\\\{\}]
{\color{outcolor}Out[{\color{outcolor}52}]:}     k1  k2
         0  one   1
         1  two   1
         2  one   2
         3  two   3
         4  one   3
         5  two   4
\end{Verbatim}
            
    上述的方法进行相关操作时默认考虑了所有的列;但是我们可以只指定部分的列以探测重复值,举个例子:

    \begin{Verbatim}[commandchars=\\\{\}]
{\color{incolor}In [{\color{incolor}53}]:} \PY{n}{data}\PY{p}{[}\PY{l+s+s1}{\PYZsq{}}\PY{l+s+s1}{v1}\PY{l+s+s1}{\PYZsq{}}\PY{p}{]} \PY{o}{=} \PY{n+nb}{range}\PY{p}{(}\PY{l+m+mi}{7}\PY{p}{)}
\end{Verbatim}

    \begin{Verbatim}[commandchars=\\\{\}]
{\color{incolor}In [{\color{incolor}54}]:} \PY{n}{data}
\end{Verbatim}

\begin{Verbatim}[commandchars=\\\{\}]
{\color{outcolor}Out[{\color{outcolor}54}]:}     k1  k2  v1
         0  one   1   0
         1  two   1   1
         2  one   2   2
         3  two   3   3
         4  one   3   4
         5  two   4   5
         6  two   4   6
\end{Verbatim}
            
    \begin{Verbatim}[commandchars=\\\{\}]
{\color{incolor}In [{\color{incolor}56}]:} \PY{c+c1}{\PYZsh{} 只对k1列除去重复值}
         \PY{n}{data}\PY{o}{.}\PY{n}{drop\PYZus{}duplicates}\PY{p}{(}\PY{p}{[}\PY{l+s+s1}{\PYZsq{}}\PY{l+s+s1}{k1}\PY{l+s+s1}{\PYZsq{}}\PY{p}{]}\PY{p}{)}
\end{Verbatim}

\begin{Verbatim}[commandchars=\\\{\}]
{\color{outcolor}Out[{\color{outcolor}56}]:}     k1  k2  v1
         0  one   1   0
         1  two   1   1
\end{Verbatim}
            
    \emph{duplicated}()和\emph{drop\_duplicates}()默认保留第一个找到的重复值,通过传入参数\emph{keep}=`last',将会保留最后一个:

    \begin{Verbatim}[commandchars=\\\{\}]
{\color{incolor}In [{\color{incolor}57}]:} \PY{n}{data}\PY{o}{.}\PY{n}{drop\PYZus{}duplicates}\PY{p}{(}\PY{p}{[}\PY{l+s+s1}{\PYZsq{}}\PY{l+s+s1}{k1}\PY{l+s+s1}{\PYZsq{}}\PY{p}{,} \PY{l+s+s1}{\PYZsq{}}\PY{l+s+s1}{k2}\PY{l+s+s1}{\PYZsq{}}\PY{p}{]}\PY{p}{,} \PY{n}{keep}\PY{o}{=}\PY{l+s+s1}{\PYZsq{}}\PY{l+s+s1}{last}\PY{l+s+s1}{\PYZsq{}}\PY{p}{)}
\end{Verbatim}

\begin{Verbatim}[commandchars=\\\{\}]
{\color{outcolor}Out[{\color{outcolor}57}]:}     k1  k2  v1
         0  one   1   0
         1  two   1   1
         2  one   2   2
         3  two   3   3
         4  one   3   4
         6  two   4   6
\end{Verbatim}
            
    \hypertarget{transformationg-data-using-a-function-or-mapping}{%
\paragraph{Transformationg Data Using a Function or
Mapping}\label{transformationg-data-using-a-function-or-mapping}}

    对于很对数据集,我们可能想对数组,Series或者DataFrame中的行进行一些转换。举个例子:

    \begin{Verbatim}[commandchars=\\\{\}]
{\color{incolor}In [{\color{incolor}58}]:} \PY{n}{data} \PY{o}{=} \PY{n}{pd}\PY{o}{.}\PY{n}{DataFrame}\PY{p}{(}\PY{p}{\PYZob{}}\PY{l+s+s1}{\PYZsq{}}\PY{l+s+s1}{food}\PY{l+s+s1}{\PYZsq{}}\PY{p}{:} \PY{p}{[}\PY{l+s+s1}{\PYZsq{}}\PY{l+s+s1}{bacon}\PY{l+s+s1}{\PYZsq{}}\PY{p}{,} \PY{l+s+s1}{\PYZsq{}}\PY{l+s+s1}{pulled pork}\PY{l+s+s1}{\PYZsq{}}\PY{p}{,} \PY{l+s+s1}{\PYZsq{}}\PY{l+s+s1}{bacon}\PY{l+s+s1}{\PYZsq{}}\PY{p}{,}
         \PY{o}{.}\PY{o}{.}\PY{o}{.}\PY{o}{.}\PY{p}{:} \PY{l+s+s1}{\PYZsq{}}\PY{l+s+s1}{Pastrami}\PY{l+s+s1}{\PYZsq{}}\PY{p}{,} \PY{l+s+s1}{\PYZsq{}}\PY{l+s+s1}{corned beef}\PY{l+s+s1}{\PYZsq{}}\PY{p}{,} \PY{l+s+s1}{\PYZsq{}}\PY{l+s+s1}{Bacon}\PY{l+s+s1}{\PYZsq{}}\PY{p}{,}
         \PY{o}{.}\PY{o}{.}\PY{o}{.}\PY{o}{.}\PY{p}{:} \PY{l+s+s1}{\PYZsq{}}\PY{l+s+s1}{pastrami}\PY{l+s+s1}{\PYZsq{}}\PY{p}{,} \PY{l+s+s1}{\PYZsq{}}\PY{l+s+s1}{honey ham}\PY{l+s+s1}{\PYZsq{}}\PY{p}{,} \PY{l+s+s1}{\PYZsq{}}\PY{l+s+s1}{nova lox}\PY{l+s+s1}{\PYZsq{}}\PY{p}{]}\PY{p}{,}
         \PY{o}{.}\PY{o}{.}\PY{o}{.}\PY{o}{.}\PY{p}{:} \PY{l+s+s1}{\PYZsq{}}\PY{l+s+s1}{ounces}\PY{l+s+s1}{\PYZsq{}}\PY{p}{:} \PY{p}{[}\PY{l+m+mi}{4}\PY{p}{,} \PY{l+m+mi}{3}\PY{p}{,} \PY{l+m+mi}{12}\PY{p}{,} \PY{l+m+mi}{6}\PY{p}{,} \PY{l+m+mf}{7.5}\PY{p}{,} \PY{l+m+mi}{8}\PY{p}{,} \PY{l+m+mi}{3}\PY{p}{,} \PY{l+m+mi}{5}\PY{p}{,} \PY{l+m+mi}{6}\PY{p}{]}\PY{p}{\PYZcb{}}\PY{p}{)}
\end{Verbatim}

    \begin{Verbatim}[commandchars=\\\{\}]
{\color{incolor}In [{\color{incolor}59}]:} \PY{n}{data}
\end{Verbatim}

\begin{Verbatim}[commandchars=\\\{\}]
{\color{outcolor}Out[{\color{outcolor}59}]:}           food  ounces
         0        bacon     4.0
         1  pulled pork     3.0
         2        bacon    12.0
         3     Pastrami     6.0
         4  corned beef     7.5
         5        Bacon     8.0
         6     pastrami     3.0
         7    honey ham     5.0
         8     nova lox     6.0
\end{Verbatim}
            
    假设我们想增加一列用于指定哪些肉来自于哪种类型的动物:
\textgreater{}Suppose we want to add a column indicating the type of the
animal that each food came from.

    \begin{Verbatim}[commandchars=\\\{\}]
{\color{incolor}In [{\color{incolor}60}]:} \PY{n}{meat\PYZus{}to\PYZus{}animal} \PY{o}{=} \PY{p}{\PYZob{}}
         \PY{l+s+s1}{\PYZsq{}}\PY{l+s+s1}{bacon}\PY{l+s+s1}{\PYZsq{}}\PY{p}{:} \PY{l+s+s1}{\PYZsq{}}\PY{l+s+s1}{pig}\PY{l+s+s1}{\PYZsq{}}\PY{p}{,}
         \PY{l+s+s1}{\PYZsq{}}\PY{l+s+s1}{pulled pork}\PY{l+s+s1}{\PYZsq{}}\PY{p}{:} \PY{l+s+s1}{\PYZsq{}}\PY{l+s+s1}{pig}\PY{l+s+s1}{\PYZsq{}}\PY{p}{,}
         \PY{l+s+s1}{\PYZsq{}}\PY{l+s+s1}{pastrami}\PY{l+s+s1}{\PYZsq{}}\PY{p}{:} \PY{l+s+s1}{\PYZsq{}}\PY{l+s+s1}{cow}\PY{l+s+s1}{\PYZsq{}}\PY{p}{,}
         \PY{l+s+s1}{\PYZsq{}}\PY{l+s+s1}{corned beef}\PY{l+s+s1}{\PYZsq{}}\PY{p}{:} \PY{l+s+s1}{\PYZsq{}}\PY{l+s+s1}{cow}\PY{l+s+s1}{\PYZsq{}}\PY{p}{,}
         \PY{l+s+s1}{\PYZsq{}}\PY{l+s+s1}{honey ham}\PY{l+s+s1}{\PYZsq{}}\PY{p}{:} \PY{l+s+s1}{\PYZsq{}}\PY{l+s+s1}{pig}\PY{l+s+s1}{\PYZsq{}}\PY{p}{,}
         \PY{l+s+s1}{\PYZsq{}}\PY{l+s+s1}{nova lox}\PY{l+s+s1}{\PYZsq{}}\PY{p}{:} \PY{l+s+s1}{\PYZsq{}}\PY{l+s+s1}{salmon}\PY{l+s+s1}{\PYZsq{}}
         \PY{p}{\PYZcb{}}
\end{Verbatim}

    The \emph{map}() method on a Series accepts a function or dict-like
containing a mapping, but here we have a problem in that some of the
meats are capitalized and others are not. Thus, we need to convert each
value to lowercase using the \emph{Series.str.lower}() method:

    \begin{Verbatim}[commandchars=\\\{\}]
{\color{incolor}In [{\color{incolor}61}]:} \PY{n}{lowercased} \PY{o}{=} \PY{n}{data}\PY{p}{[}\PY{l+s+s1}{\PYZsq{}}\PY{l+s+s1}{food}\PY{l+s+s1}{\PYZsq{}}\PY{p}{]}\PY{o}{.}\PY{n}{str}\PY{o}{.}\PY{n}{lower}\PY{p}{(}\PY{p}{)}
\end{Verbatim}

    \begin{Verbatim}[commandchars=\\\{\}]
{\color{incolor}In [{\color{incolor}62}]:} \PY{n}{lowercased}
\end{Verbatim}

\begin{Verbatim}[commandchars=\\\{\}]
{\color{outcolor}Out[{\color{outcolor}62}]:} 0          bacon
         1    pulled pork
         2          bacon
         3       pastrami
         4    corned beef
         5          bacon
         6       pastrami
         7      honey ham
         8       nova lox
         Name: food, dtype: object
\end{Verbatim}
            
    \begin{Verbatim}[commandchars=\\\{\}]
{\color{incolor}In [{\color{incolor}63}]:} \PY{c+c1}{\PYZsh{} 增加一列animal}
         \PY{n}{data}\PY{p}{[}\PY{l+s+s1}{\PYZsq{}}\PY{l+s+s1}{animal}\PY{l+s+s1}{\PYZsq{}}\PY{p}{]} \PY{o}{=} \PY{n}{lowercased}\PY{o}{.}\PY{n}{map}\PY{p}{(}\PY{n}{meat\PYZus{}to\PYZus{}animal}\PY{p}{)}
\end{Verbatim}

    \begin{Verbatim}[commandchars=\\\{\}]
{\color{incolor}In [{\color{incolor}64}]:} \PY{n}{data}
\end{Verbatim}

\begin{Verbatim}[commandchars=\\\{\}]
{\color{outcolor}Out[{\color{outcolor}64}]:}           food  ounces  animal
         0        bacon     4.0     pig
         1  pulled pork     3.0     pig
         2        bacon    12.0     pig
         3     Pastrami     6.0     cow
         4  corned beef     7.5     cow
         5        Bacon     8.0     pig
         6     pastrami     3.0     cow
         7    honey ham     5.0     pig
         8     nova lox     6.0  salmon
\end{Verbatim}
            
    We could also pass a function that does the same work:

    \begin{Verbatim}[commandchars=\\\{\}]
{\color{incolor}In [{\color{incolor}67}]:} \PY{n}{data}\PY{p}{[}\PY{l+s+s1}{\PYZsq{}}\PY{l+s+s1}{food}\PY{l+s+s1}{\PYZsq{}}\PY{p}{]}\PY{o}{.}\PY{n}{map}\PY{p}{(}\PY{k}{lambda} \PY{n}{x}\PY{p}{:} \PY{n}{meat\PYZus{}to\PYZus{}animal}\PY{p}{[}\PY{n}{x}\PY{o}{.}\PY{n}{lower}\PY{p}{(}\PY{p}{)}\PY{p}{]}\PY{p}{)}
\end{Verbatim}

\begin{Verbatim}[commandchars=\\\{\}]
{\color{outcolor}Out[{\color{outcolor}67}]:} 0       pig
         1       pig
         2       pig
         3       cow
         4       cow
         5       pig
         6       cow
         7       pig
         8    salmon
         Name: food, dtype: object
\end{Verbatim}
            
    Using \emph{map}() method is a convenient way to perform element-wise
transformation and other related data cleaning-related operations.

    \hypertarget{replacing-values}{%
\paragraph{Replacing Values}\label{replacing-values}}

    使用\emph{fillna}()方法填充缺失值是多数值替换下的一个特殊情况。正如上述我们看到的一样,\emph{map}()方法可以用来修改一个对象中的部分值但是\emph{replace}()方法提供一种简单的且更加灵活的方法来进行替换。Let's
consider this Series:

    \begin{Verbatim}[commandchars=\\\{\}]
{\color{incolor}In [{\color{incolor}69}]:} \PY{n}{data} \PY{o}{=} \PY{n}{pd}\PY{o}{.}\PY{n}{Series}\PY{p}{(}\PY{p}{[}\PY{l+m+mf}{1.}\PY{p}{,} \PY{o}{\PYZhy{}}\PY{l+m+mf}{999.}\PY{p}{,} \PY{l+m+mf}{2.}\PY{p}{,} \PY{o}{\PYZhy{}}\PY{l+m+mf}{999.}\PY{p}{,} \PY{o}{\PYZhy{}}\PY{l+m+mf}{1000.}\PY{p}{,} \PY{l+m+mf}{3.}\PY{p}{]}\PY{p}{)}
\end{Verbatim}

    \begin{Verbatim}[commandchars=\\\{\}]
{\color{incolor}In [{\color{incolor}70}]:} \PY{n}{data}
\end{Verbatim}

\begin{Verbatim}[commandchars=\\\{\}]
{\color{outcolor}Out[{\color{outcolor}70}]:} 0       1.0
         1    -999.0
         2       2.0
         3    -999.0
         4   -1000.0
         5       3.0
         dtype: float64
\end{Verbatim}
            
    假设我们认为-999.0是一个异常值,但是我们需要将这个值修改为pandas所认识的异常值(缺失值),我们可以使用\emph{replace}()方法,它将会返回一个新的对象除非指定参数:
inplace=True.

    \begin{Verbatim}[commandchars=\\\{\}]
{\color{incolor}In [{\color{incolor}71}]:} \PY{n}{data}\PY{o}{.}\PY{n}{replace}\PY{p}{(}\PY{o}{\PYZhy{}}\PY{l+m+mi}{999}\PY{p}{,} \PY{n}{np}\PY{o}{.}\PY{n}{nan}\PY{p}{)}
\end{Verbatim}

\begin{Verbatim}[commandchars=\\\{\}]
{\color{outcolor}Out[{\color{outcolor}71}]:} 0       1.0
         1       NaN
         2       2.0
         3       NaN
         4   -1000.0
         5       3.0
         dtype: float64
\end{Verbatim}
            
    If we want to replace multile values at once, we instead pass a list and
then the substitute value (替代值):

    \begin{Verbatim}[commandchars=\\\{\}]
{\color{incolor}In [{\color{incolor}72}]:} \PY{n}{data}\PY{o}{.}\PY{n}{replace}\PY{p}{(}\PY{p}{[}\PY{o}{\PYZhy{}}\PY{l+m+mi}{999}\PY{p}{,} \PY{o}{\PYZhy{}}\PY{l+m+mi}{1000}\PY{p}{]}\PY{p}{,} \PY{n}{np}\PY{o}{.}\PY{n}{nan}\PY{p}{)}
\end{Verbatim}

\begin{Verbatim}[commandchars=\\\{\}]
{\color{outcolor}Out[{\color{outcolor}72}]:} 0    1.0
         1    NaN
         2    2.0
         3    NaN
         4    NaN
         5    3.0
         dtype: float64
\end{Verbatim}
            
    To use a different replacement for each value, pass a list of
substitute:

    \begin{Verbatim}[commandchars=\\\{\}]
{\color{incolor}In [{\color{incolor}74}]:} \PY{c+c1}{\PYZsh{} \PYZhy{}999用np.nan替换}
         \PY{c+c1}{\PYZsh{} \PYZhy{}1000用0替换}
         \PY{n}{data}\PY{o}{.}\PY{n}{replace}\PY{p}{(}\PY{p}{[}\PY{o}{\PYZhy{}}\PY{l+m+mi}{999}\PY{p}{,} \PY{o}{\PYZhy{}}\PY{l+m+mi}{1000}\PY{p}{]}\PY{p}{,} \PY{p}{[}\PY{n}{np}\PY{o}{.}\PY{n}{nan}\PY{p}{,} \PY{l+m+mi}{0}\PY{p}{]}\PY{p}{)}
\end{Verbatim}

\begin{Verbatim}[commandchars=\\\{\}]
{\color{outcolor}Out[{\color{outcolor}74}]:} 0    1.0
         1    NaN
         2    2.0
         3    NaN
         4    0.0
         5    3.0
         dtype: float64
\end{Verbatim}
            
    The argument passed can also be a dict:

    \begin{Verbatim}[commandchars=\\\{\}]
{\color{incolor}In [{\color{incolor}75}]:} \PY{n}{data}\PY{o}{.}\PY{n}{replace}\PY{p}{(}\PY{p}{\PYZob{}}\PY{o}{\PYZhy{}}\PY{l+m+mi}{999}\PY{p}{:} \PY{n}{np}\PY{o}{.}\PY{n}{nan}\PY{p}{,} \PY{o}{\PYZhy{}}\PY{l+m+mi}{1000}\PY{p}{:}\PY{l+m+mi}{0}\PY{p}{\PYZcb{}}\PY{p}{)}
\end{Verbatim}

\begin{Verbatim}[commandchars=\\\{\}]
{\color{outcolor}Out[{\color{outcolor}75}]:} 0    1.0
         1    NaN
         2    2.0
         3    NaN
         4    0.0
         5    3.0
         dtype: float64
\end{Verbatim}
            
    \begin{quote}
The \emph{data.replace}() is distinct from \emph{data.str.replace}(),
which performs string substitute element-wise.
\end{quote}

    \hypertarget{renaming-axis-indexes}{%
\paragraph{Renaming Axis Indexes}\label{renaming-axis-indexes}}

    类似Series和DataFrame中的值一样,它们的标签值也是可以修改的,举个例子:

    \begin{Verbatim}[commandchars=\\\{\}]
{\color{incolor}In [{\color{incolor}76}]:}  \PY{n}{data} \PY{o}{=} \PY{n}{pd}\PY{o}{.}\PY{n}{DataFrame}\PY{p}{(}\PY{n}{np}\PY{o}{.}\PY{n}{arange}\PY{p}{(}\PY{l+m+mi}{12}\PY{p}{)}\PY{o}{.}\PY{n}{reshape}\PY{p}{(}\PY{p}{(}\PY{l+m+mi}{3}\PY{p}{,} \PY{l+m+mi}{4}\PY{p}{)}\PY{p}{)}\PY{p}{,}
         \PY{o}{.}\PY{o}{.}\PY{o}{.}\PY{o}{.}\PY{p}{:} \PY{n}{index}\PY{o}{=}\PY{p}{[}\PY{l+s+s1}{\PYZsq{}}\PY{l+s+s1}{Ohio}\PY{l+s+s1}{\PYZsq{}}\PY{p}{,} \PY{l+s+s1}{\PYZsq{}}\PY{l+s+s1}{Colorado}\PY{l+s+s1}{\PYZsq{}}\PY{p}{,} \PY{l+s+s1}{\PYZsq{}}\PY{l+s+s1}{New York}\PY{l+s+s1}{\PYZsq{}}\PY{p}{]}\PY{p}{,}
         \PY{o}{.}\PY{o}{.}\PY{o}{.}\PY{o}{.}\PY{p}{:} \PY{n}{columns}\PY{o}{=}\PY{p}{[}\PY{l+s+s1}{\PYZsq{}}\PY{l+s+s1}{one}\PY{l+s+s1}{\PYZsq{}}\PY{p}{,} \PY{l+s+s1}{\PYZsq{}}\PY{l+s+s1}{two}\PY{l+s+s1}{\PYZsq{}}\PY{p}{,} \PY{l+s+s1}{\PYZsq{}}\PY{l+s+s1}{three}\PY{l+s+s1}{\PYZsq{}}\PY{p}{,} \PY{l+s+s1}{\PYZsq{}}\PY{l+s+s1}{four}\PY{l+s+s1}{\PYZsq{}}\PY{p}{]}\PY{p}{)}
\end{Verbatim}

    \begin{Verbatim}[commandchars=\\\{\}]
{\color{incolor}In [{\color{incolor}77}]:} \PY{n}{data}
\end{Verbatim}

\begin{Verbatim}[commandchars=\\\{\}]
{\color{outcolor}Out[{\color{outcolor}77}]:}           one  two  three  four
         Ohio        0    1      2     3
         Colorado    4    5      6     7
         New York    8    9     10    11
\end{Verbatim}
            
    想Series一样,轴的标签也有一个\emph{map}()方法:

    \begin{Verbatim}[commandchars=\\\{\}]
{\color{incolor}In [{\color{incolor}79}]:} \PY{c+c1}{\PYZsh{} 将所有行标签值全部大写}
         \PY{n}{data}\PY{o}{.}\PY{n}{index}\PY{o}{.}\PY{n}{map}\PY{p}{(}\PY{k}{lambda} \PY{n}{x}\PY{p}{:} \PY{n}{x}\PY{o}{.}\PY{n}{upper}\PY{p}{(}\PY{p}{)}\PY{p}{)}
\end{Verbatim}

\begin{Verbatim}[commandchars=\\\{\}]
{\color{outcolor}Out[{\color{outcolor}79}]:} Index(['OHIO', 'COLORADO', 'NEW YORK'], dtype='object')
\end{Verbatim}
            
    替换原来的行标签:

    \begin{Verbatim}[commandchars=\\\{\}]
{\color{incolor}In [{\color{incolor}81}]:} \PY{n}{data}\PY{o}{.}\PY{n}{index} \PY{o}{=} \PY{n}{data}\PY{o}{.}\PY{n}{index}\PY{o}{.}\PY{n}{map}\PY{p}{(}\PY{k}{lambda} \PY{n}{x}\PY{p}{:} \PY{n}{x}\PY{o}{.}\PY{n}{upper}\PY{p}{(}\PY{p}{)}\PY{p}{)}
\end{Verbatim}

    \begin{Verbatim}[commandchars=\\\{\}]
{\color{incolor}In [{\color{incolor}82}]:} \PY{n}{data}
\end{Verbatim}

\begin{Verbatim}[commandchars=\\\{\}]
{\color{outcolor}Out[{\color{outcolor}82}]:}           one  two  three  four
         OHIO        0    1      2     3
         COLORADO    4    5      6     7
         NEW YORK    8    9     10    11
\end{Verbatim}
            
    If we want to create a transformed version of dataset without modifying
the original, a useful method is \emph{rename}():

    \begin{Verbatim}[commandchars=\\\{\}]
{\color{incolor}In [{\color{incolor}85}]:} \PY{c+c1}{\PYZsh{} 行标签首字母大写}
         \PY{c+c1}{\PYZsh{} 列标签全部大写}
         \PY{n}{data}\PY{o}{.}\PY{n}{rename}\PY{p}{(}\PY{n}{index}\PY{o}{=}\PY{n+nb}{str}\PY{o}{.}\PY{n}{title}\PY{p}{,} \PY{n}{columns}\PY{o}{=}\PY{n+nb}{str}\PY{o}{.}\PY{n}{upper}\PY{p}{)}
\end{Verbatim}

\begin{Verbatim}[commandchars=\\\{\}]
{\color{outcolor}Out[{\color{outcolor}85}]:}           ONE  TWO  THREE  FOUR
         Ohio        0    1      2     3
         Colorado    4    5      6     7
         New York    8    9     10    11
\end{Verbatim}
            
    \emph{rename}()可以用来重命名标签(行标签和列标签):

    \begin{Verbatim}[commandchars=\\\{\}]
{\color{incolor}In [{\color{incolor}88}]:} \PY{n}{data}
\end{Verbatim}

\begin{Verbatim}[commandchars=\\\{\}]
{\color{outcolor}Out[{\color{outcolor}88}]:}           one  two  three  four
         OHIO        0    1      2     3
         COLORADO    4    5      6     7
         NEW YORK    8    9     10    11
\end{Verbatim}
            
    \begin{Verbatim}[commandchars=\\\{\}]
{\color{incolor}In [{\color{incolor}89}]:} \PY{n}{data}\PY{o}{.}\PY{n}{rename}\PY{p}{(}\PY{n}{index}\PY{o}{=}\PY{p}{\PYZob{}}\PY{l+s+s1}{\PYZsq{}}\PY{l+s+s1}{NEW YORK}\PY{l+s+s1}{\PYZsq{}}\PY{p}{:} \PY{l+s+s1}{\PYZsq{}}\PY{l+s+s1}{Nanning}\PY{l+s+s1}{\PYZsq{}}\PY{p}{\PYZcb{}}\PY{p}{,}
                    \PY{n}{columns}\PY{o}{=}\PY{p}{\PYZob{}}\PY{l+s+s1}{\PYZsq{}}\PY{l+s+s1}{four}\PY{l+s+s1}{\PYZsq{}}\PY{p}{:} \PY{l+m+mi}{4}\PY{p}{\PYZcb{}}\PY{p}{)}
\end{Verbatim}

\begin{Verbatim}[commandchars=\\\{\}]
{\color{outcolor}Out[{\color{outcolor}89}]:}           one  two  three   4
         OHIO        0    1      2   3
         COLORADO    4    5      6   7
         Nanning     8    9     10  11
\end{Verbatim}
            
    \emph{rename}() method saves us from the chore (家庭杂事,累人的事) of
copying the DataFrame manually and assigning to its index and columns
attributes. By passing the argument \emph{inplace}, we can modify a
dataset in-place:

    \begin{Verbatim}[commandchars=\\\{\}]
{\color{incolor}In [{\color{incolor}90}]:} \PY{n}{data}\PY{o}{.}\PY{n}{rename}\PY{p}{(}\PY{n}{index}\PY{o}{=}\PY{p}{\PYZob{}}\PY{l+s+s1}{\PYZsq{}}\PY{l+s+s1}{OHIO}\PY{l+s+s1}{\PYZsq{}}\PY{p}{:}\PY{l+s+s1}{\PYZsq{}}\PY{l+s+s1}{Liuzhou}\PY{l+s+s1}{\PYZsq{}}\PY{p}{\PYZcb{}}\PY{p}{,} \PY{n}{inplace}\PY{o}{=}\PY{k+kc}{True}\PY{p}{)}
\end{Verbatim}

    \begin{Verbatim}[commandchars=\\\{\}]
{\color{incolor}In [{\color{incolor}91}]:} \PY{n}{data}
\end{Verbatim}

\begin{Verbatim}[commandchars=\\\{\}]
{\color{outcolor}Out[{\color{outcolor}91}]:}           one  two  three  four
         Liuzhou     0    1      2     3
         COLORADO    4    5      6     7
         NEW YORK    8    9     10    11
\end{Verbatim}
            
    \hypertarget{discretization-and-binning}{%
\paragraph{Discretization and
Binning}\label{discretization-and-binning}}

离散化和装箱

    Continuous data is discretized or otherwise separated into ``bins'' for
analysis. Suppose we have data about a group of people in a study, and
we want to group them into discrete age buckets (桶):

    \begin{Verbatim}[commandchars=\\\{\}]
{\color{incolor}In [{\color{incolor}92}]:} \PY{n}{ages} \PY{o}{=} \PY{p}{[}\PY{l+m+mi}{20}\PY{p}{,} \PY{l+m+mi}{22}\PY{p}{,} \PY{l+m+mi}{25}\PY{p}{,} \PY{l+m+mi}{27}\PY{p}{,} \PY{l+m+mi}{21}\PY{p}{,} \PY{l+m+mi}{23}\PY{p}{,} \PY{l+m+mi}{37}\PY{p}{,} \PY{l+m+mi}{31}\PY{p}{,} \PY{l+m+mi}{61}\PY{p}{,} \PY{l+m+mi}{45}\PY{p}{,} \PY{l+m+mi}{41}\PY{p}{,} \PY{l+m+mi}{32}\PY{p}{]}
\end{Verbatim}

    Let's divide these into bins of 18 to 25, 26 to 35, 36 to 60, and
finally 61 and older. To do so, we have to use \emph{cut}(), a method in
pandas:

    \begin{Verbatim}[commandchars=\\\{\}]
{\color{incolor}In [{\color{incolor}93}]:} \PY{n}{bins} \PY{o}{=} \PY{p}{[}\PY{l+m+mi}{18}\PY{p}{,} \PY{l+m+mi}{25} \PY{p}{,}\PY{l+m+mi}{35}\PY{p}{,} \PY{l+m+mi}{60}\PY{p}{,} \PY{l+m+mi}{100}\PY{p}{]}
\end{Verbatim}

    \begin{Verbatim}[commandchars=\\\{\}]
{\color{incolor}In [{\color{incolor}94}]:} \PY{n}{cats} \PY{o}{=} \PY{n}{pd}\PY{o}{.}\PY{n}{cut}\PY{p}{(}\PY{n}{ages}\PY{p}{,} \PY{n}{bins}\PY{p}{)}
\end{Verbatim}

    \begin{Verbatim}[commandchars=\\\{\}]
{\color{incolor}In [{\color{incolor}95}]:} \PY{n}{cats}
\end{Verbatim}

\begin{Verbatim}[commandchars=\\\{\}]
{\color{outcolor}Out[{\color{outcolor}95}]:} [(18, 25], (18, 25], (18, 25], (25, 35], (18, 25], {\ldots}, (25, 35], (60, 100], (35, 60], (35, 60], (25, 35]]
         Length: 12
         Categories (4, interval[int64]): [(18, 25] < (25, 35] < (35, 60] < (60, 100]]
\end{Verbatim}
            
    The object pandas returns is a special Categorical object. The object's
\emph{codes} attribute contains a categories array specifying the
distinct category names along with a labeling for the \emph{ages} data:

    \begin{Verbatim}[commandchars=\\\{\}]
{\color{incolor}In [{\color{incolor}97}]:} \PY{c+c1}{\PYZsh{} 因为有四个区间,所以这里分成了4类:0,1,2,3,4}
         \PY{n}{cats}\PY{o}{.}\PY{n}{codes}
\end{Verbatim}

\begin{Verbatim}[commandchars=\\\{\}]
{\color{outcolor}Out[{\color{outcolor}97}]:} array([0, 0, 0, 1, 0, 0, 2, 1, 3, 2, 2, 1], dtype=int8)
\end{Verbatim}
            
    \begin{Verbatim}[commandchars=\\\{\}]
{\color{incolor}In [{\color{incolor}99}]:} \PY{c+c1}{\PYZsh{} 每一个类别的划分}
         \PY{n}{cats}\PY{o}{.}\PY{n}{categories}
\end{Verbatim}

\begin{Verbatim}[commandchars=\\\{\}]
{\color{outcolor}Out[{\color{outcolor}99}]:} IntervalIndex([(18, 25], (25, 35], (35, 60], (60, 100]],
                       closed='right',
                       dtype='interval[int64]')
\end{Verbatim}
            
    使用\emph{pandas.value\_counts}()方法计算每一个类别的人数:

    \begin{Verbatim}[commandchars=\\\{\}]
{\color{incolor}In [{\color{incolor}100}]:} \PY{n}{pd}\PY{o}{.}\PY{n}{value\PYZus{}counts}\PY{p}{(}\PY{n}{cats}\PY{p}{)}
\end{Verbatim}

\begin{Verbatim}[commandchars=\\\{\}]
{\color{outcolor}Out[{\color{outcolor}100}]:} (18, 25]     5
          (35, 60]     3
          (25, 35]     3
          (60, 100]    1
          dtype: int64
\end{Verbatim}
            
    Consistent with mathematical notation for intervals (,区间,间隔), a
parenthesis means that the side is open, while the square bracket mean
it is closed (inclusive). We can change which side is closed by passing
\emph{right}=False:

    \begin{Verbatim}[commandchars=\\\{\}]
{\color{incolor}In [{\color{incolor}101}]:} \PY{n}{pd}\PY{o}{.}\PY{n}{cut}\PY{p}{(}\PY{n}{ages}\PY{p}{,} \PY{p}{[}\PY{l+m+mi}{18}\PY{p}{,} \PY{l+m+mi}{26}\PY{p}{,} \PY{l+m+mi}{36}\PY{p}{,} \PY{l+m+mi}{61}\PY{p}{,} \PY{l+m+mi}{100}\PY{p}{]}\PY{p}{,} \PY{n}{right}\PY{o}{=}\PY{k+kc}{False}\PY{p}{)}
\end{Verbatim}

\begin{Verbatim}[commandchars=\\\{\}]
{\color{outcolor}Out[{\color{outcolor}101}]:} [[18, 26), [18, 26), [18, 26), [26, 36), [18, 26), {\ldots}, [26, 36), [61, 100), [36, 61), [36, 61), [26, 36)]
          Length: 12
          Categories (4, interval[int64]): [[18, 26) < [26, 36) < [36, 61) < [61, 100)]
\end{Verbatim}
            
    We can also pass our own bin names by passing a list or an array to the
\emph{labels} option:

    \begin{Verbatim}[commandchars=\\\{\}]
{\color{incolor}In [{\color{incolor}102}]:} \PY{n}{group\PYZus{}names} \PY{o}{=} \PY{p}{[}\PY{l+s+s2}{\PYZdq{}}\PY{l+s+s2}{Youth}\PY{l+s+s2}{\PYZdq{}}\PY{p}{,} \PY{l+s+s2}{\PYZdq{}}\PY{l+s+s2}{YoungAdult}\PY{l+s+s2}{\PYZdq{}}\PY{p}{,} \PY{l+s+s2}{\PYZdq{}}\PY{l+s+s2}{MiddleAged}\PY{l+s+s2}{\PYZdq{}}\PY{p}{,} \PY{l+s+s2}{\PYZdq{}}\PY{l+s+s2}{Senior}\PY{l+s+s2}{\PYZdq{}}\PY{p}{]}
\end{Verbatim}

    \begin{Verbatim}[commandchars=\\\{\}]
{\color{incolor}In [{\color{incolor}103}]:} \PY{n}{pd}\PY{o}{.}\PY{n}{cut}\PY{p}{(}\PY{n}{ages}\PY{p}{,} \PY{n}{bins}\PY{p}{,} \PY{n}{labels}\PY{o}{=}\PY{n}{group\PYZus{}names}\PY{p}{)}
\end{Verbatim}

\begin{Verbatim}[commandchars=\\\{\}]
{\color{outcolor}Out[{\color{outcolor}103}]:} [Youth, Youth, Youth, YoungAdult, Youth, {\ldots}, YoungAdult, Senior, MiddleAged, MiddleAged, YoungAdult]
          Length: 12
          Categories (4, object): [Youth < YoungAdult < MiddleAged < Senior]
\end{Verbatim}
            
    If we pass an integer number of bins to the \emph{cut}() method instead
of explicit edges, it will compute the equal-length bins based on the
miminum and maximum values in the data. Consider the case of some
uniformly distributed data chopped into fourths:

    \begin{Verbatim}[commandchars=\\\{\}]
{\color{incolor}In [{\color{incolor}104}]:} \PY{n}{data} \PY{o}{=} \PY{n}{np}\PY{o}{.}\PY{n}{random}\PY{o}{.}\PY{n}{rand}\PY{p}{(}\PY{l+m+mi}{20}\PY{p}{)}
\end{Verbatim}

    \begin{Verbatim}[commandchars=\\\{\}]
{\color{incolor}In [{\color{incolor}106}]:} \PY{c+c1}{\PYZsh{} 根据data中的最大值和最小值分成4个区间(也叫bins,桶)}
          \PY{c+c1}{\PYZsh{} 保留2位小数}
          \PY{n}{pd}\PY{o}{.}\PY{n}{cut}\PY{p}{(}\PY{n}{data}\PY{p}{,} \PY{l+m+mi}{4}\PY{p}{,} \PY{n}{precision}\PY{o}{=}\PY{l+m+mi}{2}\PY{p}{)}
\end{Verbatim}

\begin{Verbatim}[commandchars=\\\{\}]
{\color{outcolor}Out[{\color{outcolor}106}]:} [(0.0023, 0.25], (0.49, 0.74], (0.0023, 0.25], (0.0023, 0.25], (0.0023, 0.25], {\ldots}, (0.49, 0.74], (0.25, 0.49], (0.49, 0.74], (0.49, 0.74], (0.74, 0.99]]
          Length: 20
          Categories (4, interval[float64]): [(0.0023, 0.25] < (0.25, 0.49] < (0.49, 0.74] < (0.74, 0.99]]
\end{Verbatim}
            
    The \emph{precision}=2 option limits the decimal precision to two
digits.

    A closely related function, \emph{qcut}(), bins the data on sample
quantiles. Depending on the distribution of the data, using \emph{cut}()
method will not usually result in each bin having same number of data
point. Since \emph{qcut}() uses sample quantiles instead, by definition
you will obtain roughly equal-size bins:

    \begin{Verbatim}[commandchars=\\\{\}]
{\color{incolor}In [{\color{incolor}107}]:} \PY{n}{data} \PY{o}{=} \PY{n}{np}\PY{o}{.}\PY{n}{random}\PY{o}{.}\PY{n}{randn}\PY{p}{(}\PY{l+m+mi}{1000}\PY{p}{)} \PY{c+c1}{\PYZsh{} normal distribution}
\end{Verbatim}

    \begin{Verbatim}[commandchars=\\\{\}]
{\color{incolor}In [{\color{incolor}116}]:} \PY{n}{cats} \PY{o}{=} \PY{n}{pd}\PY{o}{.}\PY{n}{qcut}\PY{p}{(}\PY{n}{data}\PY{p}{,} \PY{l+m+mi}{4}\PY{p}{)} \PY{c+c1}{\PYZsh{} cut into quartile}
\end{Verbatim}

    \begin{Verbatim}[commandchars=\\\{\}]
{\color{incolor}In [{\color{incolor}117}]:} \PY{n}{cats}
\end{Verbatim}

\begin{Verbatim}[commandchars=\\\{\}]
{\color{outcolor}Out[{\color{outcolor}117}]:} [(-0.00142, 0.631], (-3.009, -0.611], (-0.611, -0.00142], (-0.611, -0.00142], (-0.00142, 0.631], {\ldots}, (-3.009, -0.611], (-3.009, -0.611], (-3.009, -0.611], (-0.00142, 0.631], (-0.00142, 0.631]]
          Length: 1000
          Categories (4, interval[float64]): [(-3.009, -0.611] < (-0.611, -0.00142] < (-0.00142, 0.631] < (0.631, 3.77]]
\end{Verbatim}
            
    \begin{Verbatim}[commandchars=\\\{\}]
{\color{incolor}In [{\color{incolor}118}]:} \PY{n}{pd}\PY{o}{.}\PY{n}{value\PYZus{}counts}\PY{p}{(}\PY{n}{cats}\PY{p}{)}
\end{Verbatim}

\begin{Verbatim}[commandchars=\\\{\}]
{\color{outcolor}Out[{\color{outcolor}118}]:} (0.631, 3.77]         250
          (-0.00142, 0.631]     250
          (-0.611, -0.00142]    250
          (-3.009, -0.611]      250
          dtype: int64
\end{Verbatim}
            
    Similar to \emph{cut}(), we can pass our own quantiles (numbers between
0 and 1, inclusive):

    \begin{Verbatim}[commandchars=\\\{\}]
{\color{incolor}In [{\color{incolor}119}]:} \PY{n}{pd}\PY{o}{.}\PY{n}{qcut}\PY{p}{(}\PY{n}{data}\PY{p}{,} \PY{p}{[}\PY{l+m+mi}{0}\PY{p}{,} \PY{l+m+mf}{0.1}\PY{p}{,} \PY{l+m+mf}{0.5}\PY{p}{,} \PY{l+m+mf}{0.6}\PY{p}{,} \PY{l+m+mf}{0.9}\PY{p}{,} \PY{l+m+mf}{1.}\PY{p}{]}\PY{p}{)}
\end{Verbatim}

\begin{Verbatim}[commandchars=\\\{\}]
{\color{outcolor}Out[{\color{outcolor}119}]:} [(-0.00142, 0.215], (-3.009, -1.253], (-1.253, -0.00142], (-1.253, -0.00142], (-0.00142, 0.215], {\ldots}, (-1.253, -0.00142], (-3.009, -1.253], (-1.253, -0.00142], (0.215, 1.25], (-0.00142, 0.215]]
          Length: 1000
          Categories (5, interval[float64]): [(-3.009, -1.253] < (-1.253, -0.00142] < (-0.00142, 0.215] < (0.215, 1.25] < (1.25, 3.77]]
\end{Verbatim}
            
    \hypertarget{detecting-and-filtering-outliers-ux5f02ux5e38ux503c}{%
\paragraph{Detecting and Filtering Outliers
(异常值)}\label{detecting-and-filtering-outliers-ux5f02ux5e38ux503c}}

    过滤或者转换异常值是应用数组操作的一键大事,举个例子:

    \begin{Verbatim}[commandchars=\\\{\}]
{\color{incolor}In [{\color{incolor}120}]:} \PY{n}{data} \PY{o}{=} \PY{n}{pd}\PY{o}{.}\PY{n}{DataFrame}\PY{p}{(}\PY{n}{np}\PY{o}{.}\PY{n}{random}\PY{o}{.}\PY{n}{randn}\PY{p}{(}\PY{l+m+mi}{1000}\PY{p}{,} \PY{l+m+mi}{4}\PY{p}{)}\PY{p}{)}
\end{Verbatim}

    \begin{Verbatim}[commandchars=\\\{\}]
{\color{incolor}In [{\color{incolor}121}]:} \PY{n}{data}\PY{o}{.}\PY{n}{describe}\PY{p}{(}\PY{p}{)}
\end{Verbatim}

\begin{Verbatim}[commandchars=\\\{\}]
{\color{outcolor}Out[{\color{outcolor}121}]:}                  0            1            2            3
          count  1000.000000  1000.000000  1000.000000  1000.000000
          mean      0.062161     0.010908    -0.012074     0.010671
          std       0.994697     1.007757     0.964354     0.999827
          min      -3.299686    -2.952604    -3.579078    -3.577651
          25\%      -0.620926    -0.645775    -0.700070    -0.668896
          50\%       0.014344    -0.017357     0.012803     0.038414
          75\%       0.764888     0.677508     0.626660     0.709005
          max       3.192125     3.342354     3.205287     2.542400
\end{Verbatim}
            
    Suppose we want to find values in one of the columns exceding 3 in
absolute value:

    \begin{Verbatim}[commandchars=\\\{\}]
{\color{incolor}In [{\color{incolor}122}]:} \PY{n}{col} \PY{o}{=} \PY{n}{data}\PY{p}{[}\PY{l+m+mi}{2}\PY{p}{]}
\end{Verbatim}

    \begin{Verbatim}[commandchars=\\\{\}]
{\color{incolor}In [{\color{incolor}123}]:} \PY{n}{col}\PY{p}{[}\PY{n}{np}\PY{o}{.}\PY{n}{abs}\PY{p}{(}\PY{n}{col}\PY{p}{)} \PY{o}{\PYZgt{}} \PY{l+m+mi}{3}\PY{p}{]}
\end{Verbatim}

\begin{Verbatim}[commandchars=\\\{\}]
{\color{outcolor}Out[{\color{outcolor}123}]:} 189    3.109010
          389   -3.579078
          824    3.205287
          Name: 2, dtype: float64
\end{Verbatim}
            
    从所有的行中选择至少有一个数值的绝对值大于3的数据出来,可以在一个布尔值DataFrame使用上\emph{any}()方法进行选择:

    \begin{Verbatim}[commandchars=\\\{\}]
{\color{incolor}In [{\color{incolor}124}]:} \PY{n}{data}\PY{p}{[}\PY{p}{(}\PY{n}{np}\PY{o}{.}\PY{n}{abs}\PY{p}{(}\PY{n}{data}\PY{p}{)} \PY{o}{\PYZgt{}} \PY{l+m+mi}{3}\PY{p}{)}\PY{o}{.}\PY{n}{any}\PY{p}{(}\PY{l+m+mi}{1}\PY{p}{)}\PY{p}{]}
\end{Verbatim}

\begin{Verbatim}[commandchars=\\\{\}]
{\color{outcolor}Out[{\color{outcolor}124}]:}             0         1         2         3
          189 -0.513485  0.464559  3.109010  0.958465
          389 -0.377858  0.341131 -3.579078 -0.906682
          450  0.516619  0.099933 -0.835361 -3.163249
          599 -2.139750  1.593949  1.175514 -3.435382
          638 -1.383712  3.318922 -0.770849 -2.257575
          668 -0.805004  0.233830  0.284846 -3.577651
          735 -3.299686  0.157850  1.325712 -0.321169
          824  0.266576  2.287888  3.205287  1.195037
          844 -0.756320  3.342354 -2.250097 -1.664915
          852  3.192125 -0.897974  0.696980  0.707895
          953  1.806269  0.304860 -0.447591 -3.053805
\end{Verbatim}
            
    将{[}-3, 3{]}之外的数据都乘以3:

    \begin{Verbatim}[commandchars=\\\{\}]
{\color{incolor}In [{\color{incolor}125}]:} \PY{n}{data}\PY{p}{[}\PY{n}{np}\PY{o}{.}\PY{n}{abs}\PY{p}{(}\PY{n}{data}\PY{p}{)} \PY{o}{\PYZgt{}} \PY{l+m+mi}{3}\PY{p}{]} \PY{o}{=} \PY{n}{np}\PY{o}{.}\PY{n}{sign}\PY{p}{(}\PY{n}{data}\PY{p}{)} \PY{o}{*} \PY{l+m+mi}{3}
\end{Verbatim}

    \begin{Verbatim}[commandchars=\\\{\}]
{\color{incolor}In [{\color{incolor}126}]:} \PY{n}{data}\PY{o}{.}\PY{n}{describe}\PY{p}{(}\PY{p}{)}
\end{Verbatim}

\begin{Verbatim}[commandchars=\\\{\}]
{\color{outcolor}Out[{\color{outcolor}126}]:}                  0            1            2            3
          count  1000.000000  1000.000000  1000.000000  1000.000000
          mean      0.062269     0.010246    -0.011809     0.011901
          std       0.993141     1.005683     0.961369     0.995834
          min      -3.000000    -2.952604    -3.000000    -3.000000
          25\%      -0.620926    -0.645775    -0.700070    -0.668896
          50\%       0.014344    -0.017357     0.012803     0.038414
          75\%       0.764888     0.677508     0.626660     0.709005
          max       3.000000     3.000000     3.000000     2.542400
\end{Verbatim}
            
    \begin{Verbatim}[commandchars=\\\{\}]
{\color{incolor}In [{\color{incolor}130}]:} \PY{n}{test} \PY{o}{=} \PY{n}{np}\PY{o}{.}\PY{n}{random}\PY{o}{.}\PY{n}{randint}\PY{p}{(}\PY{o}{\PYZhy{}}\PY{l+m+mi}{10}\PY{p}{,} \PY{l+m+mi}{10}\PY{p}{,} \PY{n}{size}\PY{o}{=}\PY{p}{(}\PY{l+m+mi}{3}\PY{p}{,}\PY{l+m+mi}{3}\PY{p}{)}\PY{p}{)}
\end{Verbatim}

    \begin{Verbatim}[commandchars=\\\{\}]
{\color{incolor}In [{\color{incolor}136}]:} \PY{n}{test} \PY{o}{=} \PY{n}{pd}\PY{o}{.}\PY{n}{DataFrame}\PY{p}{(}\PY{n}{test}\PY{p}{)}
\end{Verbatim}

    \begin{Verbatim}[commandchars=\\\{\}]
{\color{incolor}In [{\color{incolor}137}]:} \PY{n}{test}
\end{Verbatim}

\begin{Verbatim}[commandchars=\\\{\}]
{\color{outcolor}Out[{\color{outcolor}137}]:}    0  1  2
          0 -8 -2 -9
          1 -4  7 -3
          2  0  8 -8
\end{Verbatim}
            
    \begin{Verbatim}[commandchars=\\\{\}]
{\color{incolor}In [{\color{incolor}139}]:} \PY{n}{test}\PY{p}{[}\PY{n}{np}\PY{o}{.}\PY{n}{abs}\PY{p}{(}\PY{n}{test}\PY{p}{)} \PY{o}{\PYZgt{}} \PY{l+m+mi}{3}\PY{p}{]} \PY{o}{=} \PY{n}{np}\PY{o}{.}\PY{n}{sign}\PY{p}{(}\PY{n}{test}\PY{p}{)} \PY{o}{*} \PY{l+m+mi}{3}
\end{Verbatim}

    \begin{Verbatim}[commandchars=\\\{\}]
{\color{incolor}In [{\color{incolor}140}]:} \PY{n}{test}
\end{Verbatim}

\begin{Verbatim}[commandchars=\\\{\}]
{\color{outcolor}Out[{\color{outcolor}140}]:}    0  1  2
          0 -3 -2 -3
          1 -3  3 -3
          2  0  3 -3
\end{Verbatim}
            
    \begin{Verbatim}[commandchars=\\\{\}]
{\color{incolor}In [{\color{incolor}141}]:} \PY{n}{np}\PY{o}{.}\PY{n}{sign}\PY{p}{(}\PY{n}{test}\PY{p}{)} \PY{o}{*} \PY{l+m+mi}{3}
\end{Verbatim}

\begin{Verbatim}[commandchars=\\\{\}]
{\color{outcolor}Out[{\color{outcolor}141}]:}    0  1  2
          0 -3 -3 -3
          1 -3  3 -3
          2  0  3 -3
\end{Verbatim}
            
    \emph{np.sign}(data)根据实参数data各个元素的符号翻译-1和1的数组,数组的大小(shape)和data一样。

    \hypertarget{permutation-and-random-sampling}{%
\paragraph{Permutation and Random
Sampling}\label{permutation-and-random-sampling}}

    Permuting (randomly reordering) a Series or the rows of a DataFrame is
easy to do using the \emph{numpy.random.permutation}(). Calling
\emph{permutation}() with the length of the axis you want to permuate
produces an array of integers indicating the new ordering:

    \begin{Verbatim}[commandchars=\\\{\}]
{\color{incolor}In [{\color{incolor}142}]:} \PY{n}{df} \PY{o}{=} \PY{n}{pd}\PY{o}{.}\PY{n}{DataFrame}\PY{p}{(}\PY{n}{np}\PY{o}{.}\PY{n}{arange}\PY{p}{(}\PY{l+m+mi}{5} \PY{o}{*} \PY{l+m+mi}{4}\PY{p}{)}\PY{o}{.}\PY{n}{reshape}\PY{p}{(}\PY{p}{(}\PY{l+m+mi}{5}\PY{p}{,} \PY{l+m+mi}{4}\PY{p}{)}\PY{p}{)}\PY{p}{)}
\end{Verbatim}

    \begin{Verbatim}[commandchars=\\\{\}]
{\color{incolor}In [{\color{incolor}143}]:} \PY{n}{df}
\end{Verbatim}

\begin{Verbatim}[commandchars=\\\{\}]
{\color{outcolor}Out[{\color{outcolor}143}]:}     0   1   2   3
          0   0   1   2   3
          1   4   5   6   7
          2   8   9  10  11
          3  12  13  14  15
          4  16  17  18  19
\end{Verbatim}
            
    \begin{Verbatim}[commandchars=\\\{\}]
{\color{incolor}In [{\color{incolor}146}]:} \PY{n}{sampler} \PY{o}{=} \PY{n}{np}\PY{o}{.}\PY{n}{random}\PY{o}{.}\PY{n}{permutation}\PY{p}{(}\PY{l+m+mi}{5}\PY{p}{)}
\end{Verbatim}

    \begin{Verbatim}[commandchars=\\\{\}]
{\color{incolor}In [{\color{incolor}147}]:} \PY{n}{sampler}
\end{Verbatim}

\begin{Verbatim}[commandchars=\\\{\}]
{\color{outcolor}Out[{\color{outcolor}147}]:} array([3, 2, 0, 1, 4])
\end{Verbatim}
            
    The array can be used in iloc-based indexing or the equivalent function
\emph{take}() function:

    \begin{Verbatim}[commandchars=\\\{\}]
{\color{incolor}In [{\color{incolor}148}]:} \PY{n}{df}\PY{o}{.}\PY{n}{iloc}\PY{p}{[}\PY{n}{sampler}\PY{p}{]}
\end{Verbatim}

\begin{Verbatim}[commandchars=\\\{\}]
{\color{outcolor}Out[{\color{outcolor}148}]:}     0   1   2   3
          3  12  13  14  15
          2   8   9  10  11
          0   0   1   2   3
          1   4   5   6   7
          4  16  17  18  19
\end{Verbatim}
            
    \begin{Verbatim}[commandchars=\\\{\}]
{\color{incolor}In [{\color{incolor}149}]:} \PY{n}{df}\PY{o}{.}\PY{n}{take}\PY{p}{(}\PY{n}{sampler}\PY{p}{)}
\end{Verbatim}

\begin{Verbatim}[commandchars=\\\{\}]
{\color{outcolor}Out[{\color{outcolor}149}]:}     0   1   2   3
          3  12  13  14  15
          2   8   9  10  11
          0   0   1   2   3
          1   4   5   6   7
          4  16  17  18  19
\end{Verbatim}
            
    To select a random subset without replacement, we can use the
\emph{sample}() method on Series and DataFrame:

    \begin{Verbatim}[commandchars=\\\{\}]
{\color{incolor}In [{\color{incolor}152}]:} \PY{c+c1}{\PYZsh{} 按照行进行采样}
          \PY{n}{df}\PY{o}{.}\PY{n}{sample}\PY{p}{(}\PY{n}{n}\PY{o}{=}\PY{l+m+mi}{3}\PY{p}{)}
\end{Verbatim}

\begin{Verbatim}[commandchars=\\\{\}]
{\color{outcolor}Out[{\color{outcolor}152}]:}     0   1   2   3
          4  16  17  18  19
          2   8   9  10  11
          3  12  13  14  15
\end{Verbatim}
            
    \begin{Verbatim}[commandchars=\\\{\}]
{\color{incolor}In [{\color{incolor}156}]:} \PY{n}{df}\PY{o}{.}\PY{n}{sample}\PY{p}{(}\PY{n}{n}\PY{o}{=}\PY{l+m+mi}{2}\PY{p}{,} \PY{n}{axis}\PY{o}{=}\PY{l+m+mi}{1}\PY{p}{)}
\end{Verbatim}

\begin{Verbatim}[commandchars=\\\{\}]
{\color{outcolor}Out[{\color{outcolor}156}]:}     2   0
          0   2   0
          1   6   4
          2  10   8
          3  14  12
          4  18  16
\end{Verbatim}
            
    To generate a sample with replacement (to allow repeat values), pass the
argument \emph{replace}=True to sample:

    \begin{Verbatim}[commandchars=\\\{\}]
{\color{incolor}In [{\color{incolor}157}]:} \PY{n}{choices} \PY{o}{=} \PY{n}{pd}\PY{o}{.}\PY{n}{Series}\PY{p}{(}\PY{p}{[}\PY{l+m+mi}{5}\PY{p}{,} \PY{l+m+mi}{7}\PY{p}{,} \PY{o}{\PYZhy{}}\PY{l+m+mi}{1}\PY{p}{,} \PY{l+m+mi}{6}\PY{p}{,} \PY{l+m+mi}{4}\PY{p}{]}\PY{p}{)}
\end{Verbatim}

    \begin{Verbatim}[commandchars=\\\{\}]
{\color{incolor}In [{\color{incolor}159}]:} \PY{n}{draws} \PY{o}{=} \PY{n}{choices}\PY{o}{.}\PY{n}{sample}\PY{p}{(}\PY{n}{n}\PY{o}{=}\PY{l+m+mi}{10}\PY{p}{,} \PY{n}{replace}\PY{o}{=}\PY{k+kc}{True}\PY{p}{)}
\end{Verbatim}

    \begin{Verbatim}[commandchars=\\\{\}]
{\color{incolor}In [{\color{incolor}160}]:} \PY{n}{draws}
\end{Verbatim}

\begin{Verbatim}[commandchars=\\\{\}]
{\color{outcolor}Out[{\color{outcolor}160}]:} 0    5
          3    6
          2   -1
          4    4
          1    7
          3    6
          2   -1
          4    4
          0    5
          3    6
          dtype: int64
\end{Verbatim}
            
    \hypertarget{computing-indicatordummy-variables}{%
\paragraph{Computing Indicator/Dummy
Variables}\label{computing-indicatordummy-variables}}

    在统计模型或机器学习学习中有一类重要的转换是将一类变量转成指标矩阵。加入一个DataFrame中某一个列包含\emph{k}个不重复的值(该列元素可以大于\emph{k}个),那么我们生成一个包含\emph{k}列的只有0和1元素值的矩阵。可以通过\emph{pandas.get\_dummies}()方法得到这样的矩阵。举个例子:

    \begin{Verbatim}[commandchars=\\\{\}]
{\color{incolor}In [{\color{incolor}161}]:} \PY{n}{df} \PY{o}{=} \PY{n}{pd}\PY{o}{.}\PY{n}{DataFrame}\PY{p}{(}\PY{p}{\PYZob{}}\PY{l+s+s1}{\PYZsq{}}\PY{l+s+s1}{key}\PY{l+s+s1}{\PYZsq{}}\PY{p}{:} \PY{p}{[}\PY{l+s+s1}{\PYZsq{}}\PY{l+s+s1}{b}\PY{l+s+s1}{\PYZsq{}}\PY{p}{,} \PY{l+s+s1}{\PYZsq{}}\PY{l+s+s1}{b}\PY{l+s+s1}{\PYZsq{}}\PY{p}{,} \PY{l+s+s1}{\PYZsq{}}\PY{l+s+s1}{a}\PY{l+s+s1}{\PYZsq{}}\PY{p}{,} \PY{l+s+s1}{\PYZsq{}}\PY{l+s+s1}{c}\PY{l+s+s1}{\PYZsq{}}\PY{p}{,} \PY{l+s+s1}{\PYZsq{}}\PY{l+s+s1}{a}\PY{l+s+s1}{\PYZsq{}}\PY{p}{,} \PY{l+s+s1}{\PYZsq{}}\PY{l+s+s1}{b}\PY{l+s+s1}{\PYZsq{}}\PY{p}{]}\PY{p}{,}
          \PY{o}{.}\PY{o}{.}\PY{o}{.}\PY{o}{.}\PY{o}{.}\PY{p}{:} \PY{l+s+s1}{\PYZsq{}}\PY{l+s+s1}{data1}\PY{l+s+s1}{\PYZsq{}}\PY{p}{:} \PY{n+nb}{range}\PY{p}{(}\PY{l+m+mi}{6}\PY{p}{)}\PY{p}{\PYZcb{}}\PY{p}{)}
\end{Verbatim}

    \begin{Verbatim}[commandchars=\\\{\}]
{\color{incolor}In [{\color{incolor}162}]:} \PY{n}{df}
\end{Verbatim}

\begin{Verbatim}[commandchars=\\\{\}]
{\color{outcolor}Out[{\color{outcolor}162}]:}   key  data1
          0   b      0
          1   b      1
          2   a      2
          3   c      3
          4   a      4
          5   b      5
\end{Verbatim}
            
    \begin{Verbatim}[commandchars=\\\{\}]
{\color{incolor}In [{\color{incolor}163}]:} \PY{n}{pd}\PY{o}{.}\PY{n}{get\PYZus{}dummies}\PY{p}{(}\PY{n}{df}\PY{p}{[}\PY{l+s+s1}{\PYZsq{}}\PY{l+s+s1}{key}\PY{l+s+s1}{\PYZsq{}}\PY{p}{]}\PY{p}{)}
\end{Verbatim}

\begin{Verbatim}[commandchars=\\\{\}]
{\color{outcolor}Out[{\color{outcolor}163}]:}    a  b  c
          0  0  1  0
          1  0  1  0
          2  1  0  0
          3  0  0  1
          4  1  0  0
          5  0  1  0
\end{Verbatim}
            
    从上面的结果可以看出,\emph{df}的\emph{key}列有a, b,
c三个不同的元素值,调用\emph{pandas.get\_dummies}()函数后形成了一个具有三列的矩阵,矩阵中的0代表该位置对应的列名在原矩阵不存在,1表示该位置对应的列名在原矩阵中的对应位置存在。比如上面矩阵的第一行第一列的位置是0,对应的列名是a,而a不在原矩阵(df)的第一行第一列。

    给指标矩阵的列名添加前缀,可通过\emph{pandas.get\_dummies}()的参数\emph{prefix}指定:

    \begin{Verbatim}[commandchars=\\\{\}]
{\color{incolor}In [{\color{incolor}164}]:} \PY{n}{dummies} \PY{o}{=} \PY{n}{pd}\PY{o}{.}\PY{n}{get\PYZus{}dummies}\PY{p}{(}\PY{n}{df}\PY{p}{[}\PY{l+s+s1}{\PYZsq{}}\PY{l+s+s1}{key}\PY{l+s+s1}{\PYZsq{}}\PY{p}{]}\PY{p}{,} \PY{n}{prefix}\PY{o}{=}\PY{l+s+s1}{\PYZsq{}}\PY{l+s+s1}{key\PYZus{}}\PY{l+s+s1}{\PYZsq{}}\PY{p}{)}
\end{Verbatim}

    \begin{Verbatim}[commandchars=\\\{\}]
{\color{incolor}In [{\color{incolor}165}]:} \PY{n}{dummies}
\end{Verbatim}

\begin{Verbatim}[commandchars=\\\{\}]
{\color{outcolor}Out[{\color{outcolor}165}]:}    key\_\_a  key\_\_b  key\_\_c
          0       0       1       0
          1       0       1       0
          2       1       0       0
          3       0       0       1
          4       1       0       0
          5       0       1       0
\end{Verbatim}
            
    df{[}`data1'{]}

    \begin{Verbatim}[commandchars=\\\{\}]
{\color{incolor}In [{\color{incolor}174}]:} \PY{n}{df}\PY{p}{[}\PY{l+s+s1}{\PYZsq{}}\PY{l+s+s1}{data1}\PY{l+s+s1}{\PYZsq{}}\PY{p}{]} \PY{c+c1}{\PYZsh{} 这种索引方式返回的是一个Series对象}
\end{Verbatim}

\begin{Verbatim}[commandchars=\\\{\}]
{\color{outcolor}Out[{\color{outcolor}174}]:} 0    0
          1    1
          2    2
          3    3
          4    4
          5    5
          Name: data1, dtype: int64
\end{Verbatim}
            
    \begin{Verbatim}[commandchars=\\\{\}]
{\color{incolor}In [{\color{incolor}175}]:} \PY{n}{df}\PY{p}{[}\PY{p}{[}\PY{l+s+s1}{\PYZsq{}}\PY{l+s+s1}{data1}\PY{l+s+s1}{\PYZsq{}}\PY{p}{]}\PY{p}{]} \PY{c+c1}{\PYZsh{} 这种方式返回的是DataFrame对象}
\end{Verbatim}

\begin{Verbatim}[commandchars=\\\{\}]
{\color{outcolor}Out[{\color{outcolor}175}]:}    data1
          0      0
          1      1
          2      2
          3      3
          4      4
          5      5
\end{Verbatim}
            
    \begin{Verbatim}[commandchars=\\\{\}]
{\color{incolor}In [{\color{incolor}176}]:} \PY{n}{df}\PY{p}{[}\PY{p}{[}\PY{l+s+s1}{\PYZsq{}}\PY{l+s+s1}{data1}\PY{l+s+s1}{\PYZsq{}}\PY{p}{]}\PY{p}{]}\PY{o}{.}\PY{n}{join}\PY{p}{(}\PY{n}{dummies}\PY{p}{)} \PY{c+c1}{\PYZsh{} DataFrame互相连接}
\end{Verbatim}

\begin{Verbatim}[commandchars=\\\{\}]
{\color{outcolor}Out[{\color{outcolor}176}]:}    data1  key\_\_a  key\_\_b  key\_\_c
          0      0       0       1       0
          1      1       0       1       0
          2      2       1       0       0
          3      3       0       0       1
          4      4       1       0       0
          5      5       0       1       0
\end{Verbatim}
            
    \begin{Verbatim}[commandchars=\\\{\}]
{\color{incolor}In [{\color{incolor}179}]:} \PY{c+c1}{\PYZsh{} 将列名进行修改}
          \PY{n}{data}\PY{o}{.}\PY{n}{rename}\PY{p}{(}\PY{n}{columns}\PY{o}{=}\PY{p}{\PYZob{}}\PY{l+m+mi}{0}\PY{p}{:}\PY{l+s+s1}{\PYZsq{}}\PY{l+s+s1}{one}\PY{l+s+s1}{\PYZsq{}}\PY{p}{,} \PY{l+m+mi}{1}\PY{p}{:}\PY{l+s+s1}{\PYZsq{}}\PY{l+s+s1}{two}\PY{l+s+s1}{\PYZsq{}}\PY{p}{,} \PY{l+m+mi}{2}\PY{p}{:}\PY{l+s+s1}{\PYZsq{}}\PY{l+s+s1}{three}\PY{l+s+s1}{\PYZsq{}}\PY{p}{,} \PY{l+m+mi}{3}\PY{p}{:}\PY{l+s+s1}{\PYZsq{}}\PY{l+s+s1}{four}\PY{l+s+s1}{\PYZsq{}}\PY{p}{\PYZcb{}}\PY{p}{,} \PY{n}{inplace}\PY{o}{=}\PY{k+kc}{True}\PY{p}{)}
\end{Verbatim}

    \begin{Verbatim}[commandchars=\\\{\}]
{\color{incolor}In [{\color{incolor}185}]:} \PY{c+c1}{\PYZsh{} 取出\PYZsq{}one\PYZsq{}和\PYZsq{}two\PYZsq{}列的数据构成一个DataFrame}
          \PY{n}{dd} \PY{o}{=} \PY{n}{data}\PY{p}{[}\PY{p}{[}\PY{l+s+s1}{\PYZsq{}}\PY{l+s+s1}{one}\PY{l+s+s1}{\PYZsq{}}\PY{p}{,} \PY{l+s+s1}{\PYZsq{}}\PY{l+s+s1}{four}\PY{l+s+s1}{\PYZsq{}}\PY{p}{]}\PY{p}{]} 
\end{Verbatim}

    \begin{Verbatim}[commandchars=\\\{\}]
{\color{incolor}In [{\color{incolor}186}]:} \PY{c+c1}{\PYZsh{} 取出一\PYZsq{}one\PYZsq{}列,构成一个Series}
          \PY{n}{data}\PY{p}{[}\PY{l+s+s1}{\PYZsq{}}\PY{l+s+s1}{one}\PY{l+s+s1}{\PYZsq{}}\PY{p}{]}
\end{Verbatim}

\begin{Verbatim}[commandchars=\\\{\}]
{\color{outcolor}Out[{\color{outcolor}186}]:} 0     -1.486255
          1     -0.176648
          2     -1.150700
          3     -0.769186
          4     -0.882545
          5      0.043969
          6     -0.941941
          7      0.919140
          8     -0.591679
          9      0.603175
          10    -0.005484
          11     0.294564
          12    -1.446958
          13    -0.513096
          14     0.037718
          15    -0.107953
          16     0.613648
          17     0.156419
          18    -0.725193
          19    -0.385183
          20     0.641327
          21    -1.110199
          22    -0.237171
          23     0.072884
          24     0.460074
          25    -0.541733
          26     0.057325
          27     1.426806
          28    -1.200798
          29    -0.142353
                   {\ldots}   
          970    0.445650
          971   -0.055439
          972    1.682989
          973    0.123819
          974    2.127047
          975    0.710947
          976    0.279677
          977   -0.419060
          978    2.621613
          979    0.270127
          980    0.776597
          981    0.041169
          982   -0.337197
          983    0.828886
          984   -0.569314
          985    0.886491
          986    1.532338
          987    1.209169
          988    0.263185
          989    0.709655
          990    0.224568
          991   -0.319106
          992    1.229200
          993    1.168087
          994    2.138031
          995    0.211693
          996    1.655879
          997    0.591293
          998   -1.533716
          999    0.300527
          Name: one, Length: 1000, dtype: float64
\end{Verbatim}
            
    If a row in a DataFrame belongs to multiple categories, things are a bit
more complicated. 比如一部电影可能既是喜剧类又是浪漫类电影。

    \begin{Verbatim}[commandchars=\\\{\}]
{\color{incolor}In [{\color{incolor}189}]:} \PY{c+c1}{\PYZsh{} 获取\PYZsq{}one\PYZsq{}和\PYZsq{}four\PYZsq{}在data.columns中的索引}
          \PY{n}{data}\PY{o}{.}\PY{n}{columns}\PY{o}{.}\PY{n}{get\PYZus{}indexer}\PY{p}{(}\PY{p}{[}\PY{l+s+s1}{\PYZsq{}}\PY{l+s+s1}{one}\PY{l+s+s1}{\PYZsq{}}\PY{p}{,} \PY{l+s+s1}{\PYZsq{}}\PY{l+s+s1}{four}\PY{l+s+s1}{\PYZsq{}}\PY{p}{]}\PY{p}{)}
\end{Verbatim}

\begin{Verbatim}[commandchars=\\\{\}]
{\color{outcolor}Out[{\color{outcolor}189}]:} array([0, 3], dtype=int64)
\end{Verbatim}
            
    A useful recipe for staticstical applications is to combine
\emph{get\_dummies}() with a discretization function like \emph{cut}():

    \begin{Verbatim}[commandchars=\\\{\}]
{\color{incolor}In [{\color{incolor}190}]:} \PY{n}{rng} \PY{o}{=} \PY{n}{np}\PY{o}{.}\PY{n}{random}\PY{o}{.}\PY{n}{RandomState}\PY{p}{(}\PY{l+m+mi}{12345}\PY{p}{)} \PY{c+c1}{\PYZsh{} 设置局部种子 隔离其它}
\end{Verbatim}

    \begin{Verbatim}[commandchars=\\\{\}]
{\color{incolor}In [{\color{incolor}191}]:} \PY{n}{values} \PY{o}{=} \PY{n}{rng}\PY{o}{.}\PY{n}{rand}\PY{p}{(}\PY{l+m+mi}{10}\PY{p}{)}
\end{Verbatim}

    \begin{Verbatim}[commandchars=\\\{\}]
{\color{incolor}In [{\color{incolor}192}]:} \PY{n}{values}
\end{Verbatim}

\begin{Verbatim}[commandchars=\\\{\}]
{\color{outcolor}Out[{\color{outcolor}192}]:} array([0.92961609, 0.31637555, 0.18391881, 0.20456028, 0.56772503,
                 0.5955447 , 0.96451452, 0.6531771 , 0.74890664, 0.65356987])
\end{Verbatim}
            
    \begin{Verbatim}[commandchars=\\\{\}]
{\color{incolor}In [{\color{incolor}193}]:} \PY{c+c1}{\PYZsh{} 区间划分,每个区间类似一个桶,把在这个区间内的数据都}
          \PY{c+c1}{\PYZsh{} 放到这个桶里}
          \PY{n}{bins} \PY{o}{=} \PY{p}{[}\PY{l+m+mi}{0}\PY{p}{,} \PY{l+m+mf}{0.2}\PY{p}{,} \PY{l+m+mf}{0.4}\PY{p}{,} \PY{l+m+mf}{0.6}\PY{p}{,} \PY{l+m+mf}{0.8}\PY{p}{,} \PY{l+m+mf}{1.0}\PY{p}{]}
\end{Verbatim}

    \begin{Verbatim}[commandchars=\\\{\}]
{\color{incolor}In [{\color{incolor}194}]:} \PY{n}{cats} \PY{o}{=} \PY{n}{pd}\PY{o}{.}\PY{n}{cut}\PY{p}{(}\PY{n}{values}\PY{p}{,} \PY{n}{bins}\PY{p}{)}
\end{Verbatim}

    \begin{Verbatim}[commandchars=\\\{\}]
{\color{incolor}In [{\color{incolor}195}]:} \PY{n}{cats}
\end{Verbatim}

\begin{Verbatim}[commandchars=\\\{\}]
{\color{outcolor}Out[{\color{outcolor}195}]:} [(0.8, 1.0], (0.2, 0.4], (0.0, 0.2], (0.2, 0.4], (0.4, 0.6], (0.4, 0.6], (0.8, 1.0], (0.6, 0.8], (0.6, 0.8], (0.6, 0.8]]
          Categories (5, interval[float64]): [(0.0, 0.2] < (0.2, 0.4] < (0.4, 0.6] < (0.6, 0.8] < (0.8, 1.0]]
\end{Verbatim}
            
    \begin{Verbatim}[commandchars=\\\{\}]
{\color{incolor}In [{\color{incolor}196}]:} \PY{n}{pd}\PY{o}{.}\PY{n}{get\PYZus{}dummies}\PY{p}{(}\PY{n}{cats}\PY{p}{)}
\end{Verbatim}

\begin{Verbatim}[commandchars=\\\{\}]
{\color{outcolor}Out[{\color{outcolor}196}]:}    (0.0, 0.2]  (0.2, 0.4]  (0.4, 0.6]  (0.6, 0.8]  (0.8, 1.0]
          0           0           0           0           0           1
          1           0           1           0           0           0
          2           1           0           0           0           0
          3           0           1           0           0           0
          4           0           0           1           0           0
          5           0           0           1           0           0
          6           0           0           0           0           1
          7           0           0           0           1           0
          8           0           0           0           1           0
          9           0           0           0           1           0
\end{Verbatim}
            
    \begin{quote}
We create a random instance with seed=12345 to make the result
deterministic.
\end{quote}

    \hypertarget{string-manipulation}{%
\paragraph{String Manipulation}\label{string-manipulation}}

字符串操作

    Python has long been a popular raw data manipulation language in part
due to its ease of use for string and text processing. Many text
operations are made simple with the string object's built-in methods.
For more complex pattern matching and text manipulation, regular
expressions may be needed. *pandas adds to the mix by enabling us to
apply string and regular expressions concisely on whole arrays of data,
additionally handling the annoyance of missin data.

    \hypertarget{string-object-methods}{%
\paragraph{String Object Methods}\label{string-object-methods}}

    In many string munging and scripting applications, built-in methods are
sufficient. For example, a comma-separated string can be broken into
pieces with \emph{split}():

    \begin{Verbatim}[commandchars=\\\{\}]
{\color{incolor}In [{\color{incolor}197}]:} \PY{n}{val} \PY{o}{=} \PY{l+s+s1}{\PYZsq{}}\PY{l+s+s1}{a,b,    guido}\PY{l+s+s1}{\PYZsq{}}
\end{Verbatim}

    \begin{Verbatim}[commandchars=\\\{\}]
{\color{incolor}In [{\color{incolor}198}]:} \PY{n}{val}\PY{o}{.}\PY{n}{split}\PY{p}{(}\PY{l+s+s1}{\PYZsq{}}\PY{l+s+s1}{,}\PY{l+s+s1}{\PYZsq{}}\PY{p}{)}
\end{Verbatim}

\begin{Verbatim}[commandchars=\\\{\}]
{\color{outcolor}Out[{\color{outcolor}198}]:} ['a', 'b', '    guido']
\end{Verbatim}
            
    \emph{split}() oftem combined with \emph{strip}() to strim whitespace
(including line breaks):

    \begin{Verbatim}[commandchars=\\\{\}]
{\color{incolor}In [{\color{incolor}199}]:} \PY{n}{pieces} \PY{o}{=} \PY{p}{[}\PY{n}{x}\PY{o}{.}\PY{n}{strip}\PY{p}{(}\PY{p}{)} \PY{k}{for} \PY{n}{x} \PY{o+ow}{in} \PY{n}{val}\PY{o}{.}\PY{n}{split}\PY{p}{(}\PY{l+s+s1}{\PYZsq{}}\PY{l+s+s1}{,}\PY{l+s+s1}{\PYZsq{}}\PY{p}{)}\PY{p}{]}
\end{Verbatim}

    \begin{Verbatim}[commandchars=\\\{\}]
{\color{incolor}In [{\color{incolor}200}]:} \PY{n}{pieces}
\end{Verbatim}

\begin{Verbatim}[commandchars=\\\{\}]
{\color{outcolor}Out[{\color{outcolor}200}]:} ['a', 'b', 'guido']
\end{Verbatim}
            
    These strings could be concatenated together with a two-colons delimiter
using addition:

    \begin{Verbatim}[commandchars=\\\{\}]
{\color{incolor}In [{\color{incolor}201}]:} \PY{n}{first}\PY{p}{,} \PY{n}{second}\PY{p}{,} \PY{n}{third} \PY{o}{=} \PY{n}{pieces}
\end{Verbatim}

    \begin{Verbatim}[commandchars=\\\{\}]
{\color{incolor}In [{\color{incolor}202}]:} \PY{n}{first} \PY{o}{+} \PY{l+s+s2}{\PYZdq{}}\PY{l+s+s2}{::}\PY{l+s+s2}{\PYZdq{}} \PY{o}{+} \PY{n}{second} \PY{o}{+} \PY{l+s+s2}{\PYZdq{}}\PY{l+s+s2}{::}\PY{l+s+s2}{\PYZdq{}} \PY{o}{+} \PY{n}{third}
\end{Verbatim}

\begin{Verbatim}[commandchars=\\\{\}]
{\color{outcolor}Out[{\color{outcolor}202}]:} 'a::b::guido'
\end{Verbatim}
            
    But this isn't a practical general method. A faster and more Pythonic is
to pass a tuple or list to the \emph{join}() method on the string
``::'':

    \begin{Verbatim}[commandchars=\\\{\}]
{\color{incolor}In [{\color{incolor}203}]:} \PY{l+s+s2}{\PYZdq{}}\PY{l+s+s2}{::}\PY{l+s+s2}{\PYZdq{}}\PY{o}{.}\PY{n}{join}\PY{p}{(}\PY{n}{pieces}\PY{p}{)}
\end{Verbatim}

\begin{Verbatim}[commandchars=\\\{\}]
{\color{outcolor}Out[{\color{outcolor}203}]:} 'a::b::guido'
\end{Verbatim}
            
    另外一个主题是查找子串,可以使用Python的关键字\emph{in}判断一个子串是否在一个字符串中,当然,\emph{index}()和\emph{find}()也可以做到:

    \begin{Verbatim}[commandchars=\\\{\}]
{\color{incolor}In [{\color{incolor}204}]:} \PY{l+s+s1}{\PYZsq{}}\PY{l+s+s1}{guido}\PY{l+s+s1}{\PYZsq{}} \PY{o+ow}{in} \PY{n}{val}
\end{Verbatim}

\begin{Verbatim}[commandchars=\\\{\}]
{\color{outcolor}Out[{\color{outcolor}204}]:} True
\end{Verbatim}
            
    \begin{Verbatim}[commandchars=\\\{\}]
{\color{incolor}In [{\color{incolor}205}]:} \PY{n}{val}\PY{o}{.}\PY{n}{index}\PY{p}{(}\PY{l+s+s1}{\PYZsq{}}\PY{l+s+s1}{,}\PY{l+s+s1}{\PYZsq{}}\PY{p}{)}
\end{Verbatim}

\begin{Verbatim}[commandchars=\\\{\}]
{\color{outcolor}Out[{\color{outcolor}205}]:} 1
\end{Verbatim}
            
    \begin{Verbatim}[commandchars=\\\{\}]
{\color{incolor}In [{\color{incolor}206}]:} \PY{n}{val}\PY{o}{.}\PY{n}{find}\PY{p}{(}\PY{l+s+s2}{\PYZdq{}}\PY{l+s+s2}{:}\PY{l+s+s2}{\PYZdq{}}\PY{p}{)}
\end{Verbatim}

\begin{Verbatim}[commandchars=\\\{\}]
{\color{outcolor}Out[{\color{outcolor}206}]:} -1
\end{Verbatim}
            
    \begin{quote}
注意:\emph{index}()和\emph{find}()的不同之处在于,当找不到匹配的子串时,\emph{index}()方法会跑出异常,而相对应的\emph{find}()是返回-1。
\end{quote}

    Relatedly, \emph{count}() method returns the number of occurences of a
particular substring:

    \begin{Verbatim}[commandchars=\\\{\}]
{\color{incolor}In [{\color{incolor}207}]:} \PY{n}{val}\PY{o}{.}\PY{n}{count}\PY{p}{(}\PY{l+s+s1}{\PYZsq{}}\PY{l+s+s1}{,}\PY{l+s+s1}{\PYZsq{}}\PY{p}{)}
\end{Verbatim}

\begin{Verbatim}[commandchars=\\\{\}]
{\color{outcolor}Out[{\color{outcolor}207}]:} 2
\end{Verbatim}
            
    \emph{replace}() method will substitute (替换) the occurences of one
pattern for another. It is common to use delete patterns, too, by
passing am empty string:

    \begin{Verbatim}[commandchars=\\\{\}]
{\color{incolor}In [{\color{incolor}208}]:} \PY{c+c1}{\PYZsh{} 将val中的逗号\PYZsq{},\PYZsq{}替换为\PYZsq{}::\PYZsq{}}
          \PY{n}{val}\PY{o}{.}\PY{n}{replace}\PY{p}{(}\PY{l+s+s1}{\PYZsq{}}\PY{l+s+s1}{,}\PY{l+s+s1}{\PYZsq{}}\PY{p}{,} \PY{l+s+s1}{\PYZsq{}}\PY{l+s+s1}{::}\PY{l+s+s1}{\PYZsq{}}\PY{p}{)}
\end{Verbatim}

\begin{Verbatim}[commandchars=\\\{\}]
{\color{outcolor}Out[{\color{outcolor}208}]:} 'a::b::    guido'
\end{Verbatim}
            
    \begin{Verbatim}[commandchars=\\\{\}]
{\color{incolor}In [{\color{incolor}209}]:} \PY{c+c1}{\PYZsh{} 将val中的逗号删除,通过替换为空串\PYZsq{}\PYZsq{}实现}
          \PY{n}{val}\PY{o}{.}\PY{n}{replace}\PY{p}{(}\PY{l+s+s1}{\PYZsq{}}\PY{l+s+s1}{,}\PY{l+s+s1}{\PYZsq{}}\PY{p}{,} \PY{l+s+s1}{\PYZsq{}}\PY{l+s+s1}{\PYZsq{}}\PY{p}{)}
\end{Verbatim}

\begin{Verbatim}[commandchars=\\\{\}]
{\color{outcolor}Out[{\color{outcolor}209}]:} 'ab    guido'
\end{Verbatim}
            
    \begin{quote}
上述方法也可以使用正则表达式。
\end{quote}

    Python built-in string methods:\\
\includegraphics{attachment:image.png}

    \hypertarget{regular-expressions}{%
\paragraph{Regular Expressions}\label{regular-expressions}}

    Regular expressions probide a flexible way to search and match (often
more comlex) string pattern in text. Python's built-in module \emph{re}
is responsible for applying regular expression to strings.

    The \emph{re} module falls into three categories: pattern matching,
substitution, and splitting. Naturally these are all related; a regex
(short for regular expression) describes a pattern to locate in the
text, which can be used for many purposes. Let's look at an example:

    Suppose we want to split a string with a varaible number of whitespaces
(tabs, spaces and newlines). The regex describing one or more whitespace
characters is \emph{\s+}:

    \begin{Verbatim}[commandchars=\\\{\}]
{\color{incolor}In [{\color{incolor}210}]:} \PY{k+kn}{import} \PY{n+nn}{re}
\end{Verbatim}

    \begin{Verbatim}[commandchars=\\\{\}]
{\color{incolor}In [{\color{incolor}211}]:} \PY{n}{text} \PY{o}{=} \PY{l+s+s1}{\PYZsq{}}\PY{l+s+s1}{foo      bar}\PY{l+s+se}{\PYZbs{}t}\PY{l+s+s1}{  baz }\PY{l+s+se}{\PYZbs{}t}\PY{l+s+s1}{qux}\PY{l+s+s1}{\PYZsq{}}
\end{Verbatim}

    \begin{Verbatim}[commandchars=\\\{\}]
{\color{incolor}In [{\color{incolor}212}]:} \PY{n}{re}\PY{o}{.}\PY{n}{split}\PY{p}{(}\PY{l+s+s1}{\PYZsq{}}\PY{l+s+s1}{\PYZbs{}}\PY{l+s+s1}{s+}\PY{l+s+s1}{\PYZsq{}}\PY{p}{,} \PY{n}{text}\PY{p}{)}
\end{Verbatim}

\begin{Verbatim}[commandchars=\\\{\}]
{\color{outcolor}Out[{\color{outcolor}212}]:} ['foo', 'bar', 'baz', 'qux']
\end{Verbatim}
            
    \begin{quote}
注意:当我们调用\emph{re.split}(`\s+',
text)时,正则表达式会先编译,然后才会应用\emph{split}()方法到text上。我们也可以先使用\emph{re.compile}()方法编译正则表达式,得到一个可重复使用的regex对象:
\end{quote}

    \begin{Verbatim}[commandchars=\\\{\}]
{\color{incolor}In [{\color{incolor}213}]:} \PY{n}{regex} \PY{o}{=} \PY{n}{re}\PY{o}{.}\PY{n}{compile}\PY{p}{(}\PY{l+s+s1}{\PYZsq{}}\PY{l+s+s1}{\PYZbs{}}\PY{l+s+s1}{s+}\PY{l+s+s1}{\PYZsq{}}\PY{p}{)}
\end{Verbatim}

    \begin{Verbatim}[commandchars=\\\{\}]
{\color{incolor}In [{\color{incolor}214}]:} \PY{n}{regex}\PY{o}{.}\PY{n}{split}\PY{p}{(}\PY{n}{text}\PY{p}{)}
\end{Verbatim}

\begin{Verbatim}[commandchars=\\\{\}]
{\color{outcolor}Out[{\color{outcolor}214}]:} ['foo', 'bar', 'baz', 'qux']
\end{Verbatim}
            
    相对应地,如果我们想获得所有匹配到的模式,可以使用\emph{re.findall}()方法:

    \begin{Verbatim}[commandchars=\\\{\}]
{\color{incolor}In [{\color{incolor}215}]:} \PY{n}{regex}\PY{o}{.}\PY{n}{findall}\PY{p}{(}\PY{n}{text}\PY{p}{)}
\end{Verbatim}

\begin{Verbatim}[commandchars=\\\{\}]
{\color{outcolor}Out[{\color{outcolor}215}]:} ['      ', '\textbackslash{}t  ', ' \textbackslash{}t']
\end{Verbatim}
            
    Creating a regex object with \emph{re.compile}() method is highly
recommended if want to apply the same expression to many strings; doing
so will save many CPU cycles.

    \emph{match}()和\emph{search}()方法和\emph{findall}()关系较大,\emph{findall}()方法返回一个字符串中所有的匹配,而\emph{search}()只返回到第一个匹配,\emph{match}()只会去匹配一个字符串的开始位置。

    Let's consider a block of text and a regular expression capable of
identifying most email addresses:

    \begin{Verbatim}[commandchars=\\\{\}]
{\color{incolor}In [{\color{incolor}216}]:} \PY{n}{text} \PY{o}{=} \PY{l+s+s2}{\PYZdq{}\PYZdq{}\PYZdq{}}\PY{l+s+s2}{Dave dave@google.com}
          \PY{l+s+s2}{Steve steve@gmail.com}
          \PY{l+s+s2}{Rob rob@gmail.com}
          \PY{l+s+s2}{Ryan ryan@yahoo.com}
          \PY{l+s+s2}{\PYZdq{}\PYZdq{}\PYZdq{}}
\end{Verbatim}

    \begin{Verbatim}[commandchars=\\\{\}]
{\color{incolor}In [{\color{incolor}223}]:} \PY{n}{pattern} \PY{o}{=} \PY{l+s+sa}{r}\PY{l+s+s1}{\PYZsq{}}\PY{l+s+s1}{[A\PYZhy{}Z0\PYZhy{}9.\PYZus{}}\PY{l+s+s1}{\PYZpc{}}\PY{l+s+s1}{+\PYZhy{}]+@[A\PYZhy{}Z0\PYZhy{}9.\PYZhy{}]+}\PY{l+s+s1}{\PYZbs{}}\PY{l+s+s1}{.[A\PYZhy{}Z]}\PY{l+s+s1}{\PYZob{}}\PY{l+s+s1}{2,4\PYZcb{}}\PY{l+s+s1}{\PYZsq{}}
\end{Verbatim}

    \begin{Verbatim}[commandchars=\\\{\}]
{\color{incolor}In [{\color{incolor}224}]:} \PY{c+c1}{\PYZsh{} 编译}
          \PY{c+c1}{\PYZsh{} re.IGNORECASE makes the regex case\PYZhy{}insensitive}
          \PY{n}{regex} \PY{o}{=} \PY{n}{re}\PY{o}{.}\PY{n}{compile}\PY{p}{(}\PY{n}{pattern}\PY{p}{,} \PY{n}{flags}\PY{o}{=}\PY{n}{re}\PY{o}{.}\PY{n}{IGNORECASE}\PY{p}{)}
\end{Verbatim}

    使用\emph{findall}()方法找出所有的匹配项:

    \begin{Verbatim}[commandchars=\\\{\}]
{\color{incolor}In [{\color{incolor}225}]:} \PY{n}{regex}\PY{o}{.}\PY{n}{findall}\PY{p}{(}\PY{n}{text}\PY{p}{)}
\end{Verbatim}

\begin{Verbatim}[commandchars=\\\{\}]
{\color{outcolor}Out[{\color{outcolor}225}]:} ['dave@google.com', 'steve@gmail.com', 'rob@gmail.com', 'ryan@yahoo.com']
\end{Verbatim}
            
    \emph{searc}()返回一个特殊的代表第一个匹配到的邮箱的match对象,该对象告诉我们这个匹配项在字符串的起始和终止位置:

    \begin{Verbatim}[commandchars=\\\{\}]
{\color{incolor}In [{\color{incolor}226}]:} \PY{n}{m} \PY{o}{=} \PY{n}{regex}\PY{o}{.}\PY{n}{search}\PY{p}{(}\PY{n}{text}\PY{p}{)}
\end{Verbatim}

    \begin{Verbatim}[commandchars=\\\{\}]
{\color{incolor}In [{\color{incolor}227}]:} \PY{n}{m}
\end{Verbatim}

\begin{Verbatim}[commandchars=\\\{\}]
{\color{outcolor}Out[{\color{outcolor}227}]:} <re.Match object; span=(5, 20), match='dave@google.com'>
\end{Verbatim}
            
    \begin{Verbatim}[commandchars=\\\{\}]
{\color{incolor}In [{\color{incolor}228}]:} \PY{n}{text}\PY{p}{[}\PY{n}{m}\PY{o}{.}\PY{n}{start}\PY{p}{(}\PY{p}{)}\PY{p}{:}\PY{n}{m}\PY{o}{.}\PY{n}{end}\PY{p}{(}\PY{p}{)}\PY{p}{]}
\end{Verbatim}

\begin{Verbatim}[commandchars=\\\{\}]
{\color{outcolor}Out[{\color{outcolor}228}]:} 'dave@google.com'
\end{Verbatim}
            
    \begin{Verbatim}[commandchars=\\\{\}]
{\color{incolor}In [{\color{incolor}230}]:} \PY{n+nb}{print}\PY{p}{(}\PY{n}{regex}\PY{o}{.}\PY{n}{match}\PY{p}{(}\PY{n}{text}\PY{p}{)}\PY{p}{)}
\end{Verbatim}

    \begin{Verbatim}[commandchars=\\\{\}]
None

    \end{Verbatim}

    \emph{regex.match}() returns None, as it only matches if the pattern
occurs at the begining of the string.

    Relatedly, \emph{sub}() method will return a new string with occurences
of the string replaced by the new string:

    \begin{Verbatim}[commandchars=\\\{\}]
{\color{incolor}In [{\color{incolor}231}]:} \PY{n+nb}{print}\PY{p}{(}\PY{n}{regex}\PY{o}{.}\PY{n}{sub}\PY{p}{(}\PY{l+s+s2}{\PYZdq{}}\PY{l+s+s2}{redacted}\PY{l+s+s2}{\PYZdq{}}\PY{p}{,} \PY{n}{text}\PY{p}{)}\PY{p}{)}
\end{Verbatim}

    \begin{Verbatim}[commandchars=\\\{\}]
Dave redacted
Steve redacted
Rob redacted
Ryan redacted


    \end{Verbatim}

    假设我们想查找邮件地址,同时将邮件分割为三部分:用户名,域名,域名后缀。要做到这一点,我们只需在正则表达式中用圆括号(parenthesis)将这三部分的正则括起来:

    \begin{Verbatim}[commandchars=\\\{\}]
{\color{incolor}In [{\color{incolor}233}]:} \PY{n}{pattern} \PY{o}{=} \PY{l+s+sa}{r}\PY{l+s+s1}{\PYZsq{}}\PY{l+s+s1}{([A\PYZhy{}Z0\PYZhy{}9.\PYZus{}}\PY{l+s+s1}{\PYZpc{}}\PY{l+s+s1}{+\PYZhy{}]+)@([A\PYZhy{}Z0\PYZhy{}9.\PYZhy{}]+)}\PY{l+s+s1}{\PYZbs{}}\PY{l+s+s1}{.([A\PYZhy{}Z]}\PY{l+s+s1}{\PYZob{}}\PY{l+s+s1}{2,4\PYZcb{})}\PY{l+s+s1}{\PYZsq{}}
\end{Verbatim}

    \begin{Verbatim}[commandchars=\\\{\}]
{\color{incolor}In [{\color{incolor}237}]:} \PY{n}{regex} \PY{o}{=} \PY{n}{re}\PY{o}{.}\PY{n}{compile}\PY{p}{(}\PY{n}{pattern}\PY{p}{,} \PY{n}{flags}\PY{o}{=}\PY{n}{re}\PY{o}{.}\PY{n}{IGNORECASE}\PY{p}{)}
\end{Verbatim}

    A match object produced by this modified regex returns a tuple of the
pattern components with its \emph{group}() method:

    \begin{Verbatim}[commandchars=\\\{\}]
{\color{incolor}In [{\color{incolor}238}]:} \PY{n}{m} \PY{o}{=} \PY{n}{regex}\PY{o}{.}\PY{n}{match}\PY{p}{(}\PY{l+s+s1}{\PYZsq{}}\PY{l+s+s1}{tongzi@126.com}\PY{l+s+s1}{\PYZsq{}}\PY{p}{)}
\end{Verbatim}

    \begin{Verbatim}[commandchars=\\\{\}]
{\color{incolor}In [{\color{incolor}239}]:} \PY{n}{m}\PY{o}{.}\PY{n}{groups}\PY{p}{(}\PY{p}{)}
\end{Verbatim}

\begin{Verbatim}[commandchars=\\\{\}]
{\color{outcolor}Out[{\color{outcolor}239}]:} ('tongzi', '126', 'com')
\end{Verbatim}
            
    While \emph{findall}() returns a list of tuples when the pattern has
groups:

    \begin{Verbatim}[commandchars=\\\{\}]
{\color{incolor}In [{\color{incolor}240}]:} \PY{n}{regex}\PY{o}{.}\PY{n}{findall}\PY{p}{(}\PY{n}{text}\PY{p}{)}
\end{Verbatim}

\begin{Verbatim}[commandchars=\\\{\}]
{\color{outcolor}Out[{\color{outcolor}240}]:} [('dave', 'google', 'com'),
           ('steve', 'gmail', 'com'),
           ('rob', 'gmail', 'com'),
           ('ryan', 'yahoo', 'com')]
\end{Verbatim}
            
    \emph{sub}() also has access to groups in each match using special
synbols like \textbackslash{}1 and \textbackslash{}2. The symbol
\textbackslash{}1 corresponds to the first matched group,
\textbackslash{}2 corresponds to the second, and so forth:

    \begin{Verbatim}[commandchars=\\\{\}]
{\color{incolor}In [{\color{incolor}241}]:} \PY{n+nb}{print}\PY{p}{(}\PY{n}{regex}\PY{o}{.}\PY{n}{sub}\PY{p}{(}\PY{l+s+sa}{r}\PY{l+s+s1}{\PYZsq{}}\PY{l+s+s1}{User: }\PY{l+s+s1}{\PYZbs{}}\PY{l+s+s1}{1, Domain: }\PY{l+s+s1}{\PYZbs{}}\PY{l+s+s1}{2, Suffix: }\PY{l+s+s1}{\PYZbs{}}\PY{l+s+s1}{3}\PY{l+s+s1}{\PYZsq{}}\PY{p}{,} \PY{n}{text}\PY{p}{)}\PY{p}{)}
\end{Verbatim}

    \begin{Verbatim}[commandchars=\\\{\}]
Dave User: dave, Domain: google, Suffix: com
Steve User: steve, Domain: gmail, Suffix: com
Rob User: rob, Domain: gmail, Suffix: com
Ryan User: ryan, Domain: yahoo, Suffix: com


    \end{Verbatim}

    Table below provides a brief summay:\\
\includegraphics{attachment:image.png}

    \hypertarget{vectorized-string-functions-in-pandas}{%
\paragraph{Vectorized String Functions in
pandas}\label{vectorized-string-functions-in-pandas}}

    Cleaning up a messy dataset for analysis often requires a lot of string
munging and regularization. To complicate matters, a column containing
strings will sometimes have missing data:

    \begin{Verbatim}[commandchars=\\\{\}]
{\color{incolor}In [{\color{incolor}242}]:} \PY{n}{data} \PY{o}{=} \PY{p}{\PYZob{}}\PY{l+s+s1}{\PYZsq{}}\PY{l+s+s1}{Dave}\PY{l+s+s1}{\PYZsq{}}\PY{p}{:} \PY{l+s+s1}{\PYZsq{}}\PY{l+s+s1}{dave@google.com}\PY{l+s+s1}{\PYZsq{}}\PY{p}{,} \PY{l+s+s1}{\PYZsq{}}\PY{l+s+s1}{Steve}\PY{l+s+s1}{\PYZsq{}}\PY{p}{:} \PY{l+s+s1}{\PYZsq{}}\PY{l+s+s1}{steve@gmail.com}\PY{l+s+s1}{\PYZsq{}}\PY{p}{,}
          \PY{o}{.}\PY{o}{.}\PY{o}{.}\PY{o}{.}\PY{o}{.}\PY{p}{:} \PY{l+s+s1}{\PYZsq{}}\PY{l+s+s1}{Rob}\PY{l+s+s1}{\PYZsq{}}\PY{p}{:} \PY{l+s+s1}{\PYZsq{}}\PY{l+s+s1}{rob@gmail.com}\PY{l+s+s1}{\PYZsq{}}\PY{p}{,} \PY{l+s+s1}{\PYZsq{}}\PY{l+s+s1}{Wes}\PY{l+s+s1}{\PYZsq{}}\PY{p}{:} \PY{n}{np}\PY{o}{.}\PY{n}{nan}\PY{p}{\PYZcb{}}
\end{Verbatim}

    \begin{Verbatim}[commandchars=\\\{\}]
{\color{incolor}In [{\color{incolor}243}]:} \PY{n}{data} \PY{o}{=} \PY{n}{pd}\PY{o}{.}\PY{n}{Series}\PY{p}{(}\PY{n}{data}\PY{p}{)}
\end{Verbatim}

    \begin{Verbatim}[commandchars=\\\{\}]
{\color{incolor}In [{\color{incolor}244}]:} \PY{n}{data}
\end{Verbatim}

\begin{Verbatim}[commandchars=\\\{\}]
{\color{outcolor}Out[{\color{outcolor}244}]:} Dave     dave@google.com
          Steve    steve@gmail.com
          Rob        rob@gmail.com
          Wes                  NaN
          dtype: object
\end{Verbatim}
            
    \begin{Verbatim}[commandchars=\\\{\}]
{\color{incolor}In [{\color{incolor}245}]:} \PY{n}{data}\PY{o}{.}\PY{n}{isnull}\PY{p}{(}\PY{p}{)}
\end{Verbatim}

\begin{Verbatim}[commandchars=\\\{\}]
{\color{outcolor}Out[{\color{outcolor}245}]:} Dave     False
          Steve    False
          Rob      False
          Wes       True
          dtype: bool
\end{Verbatim}
            
    We can apply string and regular expression methods that can be applied
(passing a lambda or other function) to each value using
\emph{data.map}(), but it will fail on NA values. To code with this,
Series has array-oriented methods for string operations that skip NA
values. These methods are accessed through Series's \emph{str}
attribute. For example, we can check whether each email address has
`gmail' in it with \emph{str.contains}():

    \begin{Verbatim}[commandchars=\\\{\}]
{\color{incolor}In [{\color{incolor}246}]:} \PY{n}{data}\PY{o}{.}\PY{n}{str}\PY{o}{.}\PY{n}{contains}\PY{p}{(}\PY{l+s+s1}{\PYZsq{}}\PY{l+s+s1}{gmail}\PY{l+s+s1}{\PYZsq{}}\PY{p}{)}
\end{Verbatim}

\begin{Verbatim}[commandchars=\\\{\}]
{\color{outcolor}Out[{\color{outcolor}246}]:} Dave     False
          Steve     True
          Rob       True
          Wes        NaN
          dtype: object
\end{Verbatim}
            
    Regular expressions can be used, too, along with any \emph{re} options
like \emph{re.IGNORECASE}:

    \begin{Verbatim}[commandchars=\\\{\}]
{\color{incolor}In [{\color{incolor}248}]:} \PY{n}{pattern}
\end{Verbatim}

\begin{Verbatim}[commandchars=\\\{\}]
{\color{outcolor}Out[{\color{outcolor}248}]:} '([A-Z0-9.\_\%+-]+)@([A-Z0-9.-]+)\textbackslash{}\textbackslash{}.([A-Z]\{2,4\})'
\end{Verbatim}
            
    \begin{Verbatim}[commandchars=\\\{\}]
{\color{incolor}In [{\color{incolor}249}]:} \PY{n}{data}\PY{o}{.}\PY{n}{str}\PY{o}{.}\PY{n}{findall}\PY{p}{(}\PY{n}{pattern}\PY{p}{,} \PY{n}{flags}\PY{o}{=}\PY{n}{re}\PY{o}{.}\PY{n}{IGNORECASE}\PY{p}{)}
\end{Verbatim}

\begin{Verbatim}[commandchars=\\\{\}]
{\color{outcolor}Out[{\color{outcolor}249}]:} Dave     [(dave, google, com)]
          Steve    [(steve, gmail, com)]
          Rob        [(rob, gmail, com)]
          Wes                        NaN
          dtype: object
\end{Verbatim}
            
    There are a couple of ways to do vectorized element retrieval. Either
use \emph{str.get()} or index into the \emph{str} attribute:

    \begin{Verbatim}[commandchars=\\\{\}]
{\color{incolor}In [{\color{incolor}250}]:} \PY{n}{matches} \PY{o}{=} \PY{n}{data}\PY{o}{.}\PY{n}{str}\PY{o}{.}\PY{n}{match}\PY{p}{(}\PY{n}{pattern}\PY{p}{,} \PY{n}{flags}\PY{o}{=}\PY{n}{re}\PY{o}{.}\PY{n}{IGNORECASE}\PY{p}{)}
\end{Verbatim}

    \begin{Verbatim}[commandchars=\\\{\}]
{\color{incolor}In [{\color{incolor}260}]:} \PY{n}{matches}
\end{Verbatim}

\begin{Verbatim}[commandchars=\\\{\}]
{\color{outcolor}Out[{\color{outcolor}260}]:} Dave     True
          Steve    True
          Rob      True
          Wes       NaN
          dtype: object
\end{Verbatim}
            
    \begin{Verbatim}[commandchars=\\\{\}]
{\color{incolor}In [{\color{incolor}255}]:} \PY{n}{data}\PY{p}{[}\PY{l+s+s1}{\PYZsq{}}\PY{l+s+s1}{ceprei}\PY{l+s+s1}{\PYZsq{}}\PY{p}{]} \PY{o}{=} \PY{l+s+s1}{\PYZsq{}}\PY{l+s+s1}{cepreitest  software@ceprei.biz}\PY{l+s+s1}{\PYZsq{}}
\end{Verbatim}

    \begin{Verbatim}[commandchars=\\\{\}]
{\color{incolor}In [{\color{incolor}256}]:} \PY{n}{data}
\end{Verbatim}

\begin{Verbatim}[commandchars=\\\{\}]
{\color{outcolor}Out[{\color{outcolor}256}]:} Dave                      dave@google.com
          Steve                     steve@gmail.com
          Rob                         rob@gmail.com
          Wes                                   NaN
          ceprei    cepreitest  software@ceprei.biz
          dtype: object
\end{Verbatim}
            
    \begin{Verbatim}[commandchars=\\\{\}]
{\color{incolor}In [{\color{incolor}257}]:} \PY{n}{data}\PY{o}{.}\PY{n}{str}\PY{o}{.}\PY{n}{match}\PY{p}{(}\PY{n}{pattern}\PY{p}{,} \PY{n}{flags}\PY{o}{=}\PY{n}{re}\PY{o}{.}\PY{n}{IGNORECASE}\PY{p}{)}
\end{Verbatim}

\begin{Verbatim}[commandchars=\\\{\}]
{\color{outcolor}Out[{\color{outcolor}257}]:} Dave       True
          Steve      True
          Rob        True
          Wes         NaN
          ceprei    False
          dtype: object
\end{Verbatim}
            
    从上面的例子可以看出,ceprei这一项的第一个元素是cepreitest,所以\emph{data.str.match}()方法匹配失败,返回False。

    To access elements in the embedded lists, we can pass an index to either
of these functions:

    \begin{Verbatim}[commandchars=\\\{\}]
{\color{incolor}In [{\color{incolor}258}]:} \PY{c+c1}{\PYZsh{} matches是一个Series对象}
          \PY{n}{matches}\PY{o}{.}\PY{n}{str}\PY{o}{.}\PY{n}{get}\PY{p}{(}\PY{l+m+mi}{1}\PY{p}{)}
\end{Verbatim}

\begin{Verbatim}[commandchars=\\\{\}]
{\color{outcolor}Out[{\color{outcolor}258}]:} Dave    NaN
          Steve   NaN
          Rob     NaN
          Wes     NaN
          dtype: float64
\end{Verbatim}
            
    \begin{Verbatim}[commandchars=\\\{\}]
{\color{incolor}In [{\color{incolor}261}]:} \PY{n}{matches}\PY{o}{.}\PY{n}{str}\PY{p}{[}\PY{l+m+mi}{0}\PY{p}{]}
\end{Verbatim}

\begin{Verbatim}[commandchars=\\\{\}]
{\color{outcolor}Out[{\color{outcolor}261}]:} Dave    NaN
          Steve   NaN
          Rob     NaN
          Wes     NaN
          dtype: float64
\end{Verbatim}
            
    \begin{Verbatim}[commandchars=\\\{\}]
{\color{incolor}In [{\color{incolor}262}]:} \PY{n}{data}\PY{o}{.}\PY{n}{str}\PY{p}{[}\PY{p}{:}\PY{l+m+mi}{5}\PY{p}{]}
\end{Verbatim}

\begin{Verbatim}[commandchars=\\\{\}]
{\color{outcolor}Out[{\color{outcolor}262}]:} Dave      dave@
          Steve     steve
          Rob       rob@g
          Wes         NaN
          ceprei    cepre
          dtype: object
\end{Verbatim}
            
    Partial listing of vectorized string methods:\\
\includegraphics{attachment:image.png}\\
\includegraphics{attachment:image.png}

    本章完结!!


    % Add a bibliography block to the postdoc
    
    
    
    \end{document}
